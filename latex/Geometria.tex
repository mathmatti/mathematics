\documentclass[a4paper,10pt]{article}
\usepackage[utf8]{inputenc}
\usepackage{amsmath}
\usepackage{hyperref}

%opening
\title{Geometria}
\author{\href{http://www.baudo.hol.es}{giuseppe baudo}}

\begin{document}

\maketitle


\section{Introduzione}
Lo scopo di queste pagine è di mettere lo studente in grado di apprendere in maniera autonoma i concetti e gli strumenti della geometria.
Le pagine sono rivolte a studenti dei corsi di Matematica in quanto le conoscenze sono presentate sotto forma di definizioni, teoremi e relative dimostrazioni applicando le tecniche e gli strumenti della logica matematica.
I concetti, seppur presentati il più possibile in maniera formale, sono accompagnati da commenti scritti con il linguaggio dello studente.
Infatti, vengono associate al formalismo puro tipico della matematica moderna un linguaggio comune e comprensibile a tutti. Per questo si cerca di spiegare parola per
parola il significato dei termini delle definizioni, dei teoremi e di ogni altro enunciato. Le pagine di teoria sono accompagnate da
esercitazioni pratiche sia di esercizi ed esami svolti e sia di esercizi da svolgere in autonomia. Altro obiettivo è quello di contenere
in un'unica trattazione tutto il sapere matematico necessario per superare tutti gli esami di geometria proposti nei corsi universitari
di Matematica e di ogni altro corso di laurea che include anche lo studio della geometria. Le pagine sono strutturate in un indice in cui si trova
il syllabus (il nostro programma di studio) e di una pagina per ogni concetto. Le pagine sono scritte in latex e trasformate in html per la
visualizzazione. E' possibile convertire anche in formato pdf.

\subsection{I link}
Ogni pagina che espone un concetto contiene un certo numero di link. I link che mostrarno l'url, per esempio: \url{http://calvino.polito.it/~tedeschi/geometria/}, sono link 
esterni al documento. I link che non mostrano l'url, per esempio: \href{./GeometriaEuclidea.html}{Geometria euclidea}, sono link interni.
Se cliccate su un link esterno relativo a esercizi allora tutto bene ma se cliccate link esterni di teoria può darsi che la pagina non sia
completa o sia scritta in maniera non del tutto comprensibile. In quest'ultimo caso potete avvisare i partecipanti al progetto, aprendo un issue su github. 

\subsection{Appunti delle lezioni}

\section{Prerequisites}
Costanza nello studio, dedicare il giusto tempo, meglio un pò al giorno tutti i giorni.

\section{Geometria Euclidea}
  \begin{itemize}
   \item \href{./GeometriaEuclidea.html}{Geometria euclidea}	
   \item Punto
   \item Retta
   \item Rette ortogonali
   \item Rette sghembe
   \item Piano
   \item Fasci
   \item \href{./Vettore.html}{Vettore - probabilmente da spostare in algebra lineare}
   \item \href{./Reference.html}{Reference vs riferimento}
   \item \href{./SistemaRiferimento.html}{Sistema di riferimento}
   \item \href{./SistemaCartesiano.html}{Sistema cartesiano}   
   \item \href{./GeometriaEuclidea3d.html}{Spazio tridimensionale della geometria euclidea, geometria euclidea dello spazio, spazio euclideo}
   \item \href{./Segmento.html}{Segmento}
   \item \href{./InsiemeVettoriApplicati.html}{Insieme di tutti i vettori dello spazio applicati in un punto $O$}
   \item \href{./Piano.html}{Piano}
   \item \href{./PianoPassantePunto.html}{Piano passante per un punto e ortogonale ad un vettore}
   \item \href{./EquazioneCartesianaPiano.html}{Equazione cartesiana del piano}
   \item \href{./EquazioneCartesianaRettaPiano.html}{Equazione cartesiana della retta nel piano}
   \item \href{./EquazioneParametricaRettaSpazio.html}{Equazione parametrica della retta nello spazio}
   \item \href{./EquazioneCartesianaRetta.html}{Equazione cartesiana della retta nello spazio}
   \item \href{./RetteComplanari.html}{Rette complanari}
   \item \href{./Superficie.html}{Superficie}
  \end{itemize}

\section{Geometria Proiettiva}

\section{Coniche}

\section{Spazi topologici}
  \begin{itemize}
  	\item Spazi
  \end{itemize}
  
  
\section{Libri}
  \begin{itemize}
   \item \href{./libro001.html}{Geometria analitica di Silvio Greco e Paolo Valabrega}. Dettagli del libro su: \url{https://sol.unibo.it/SebinaOpac/Opac?action=search&thNomeDocumento=UBO2453492T}.
  \end{itemize}

\section{Temi d'esame}
  \begin{itemize}
   \item \url{http://calvino.polito.it/~tedeschi/geometria/}
   \item \url{http://calvino.polito.it/~rolando/Q_rette&piani.pdf}
   \item \url{http://calvino.polito.it/~rolando/2014-06-27.pdf}
  \end{itemize}

\section{Pagine simili}
\begin{itemize}
  \item \url{http://calvino.polito.it/~casnati/Geometria15BCG/dispensegeometria15BCG.htm}
\end{itemize}


\end{document}
