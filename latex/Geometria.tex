\documentclass[a4paper,10pt]{article}
\usepackage[utf8]{inputenc}
\usepackage{amsmath}
\usepackage{hyperref}

%opening
\title{Geometria}
\author{\href{http://www.baudo.hol.es}{giuseppe baudo}}

\begin{document}

\maketitle

\section*{Syllabus}

\subsection*{Fondamenti}
  \begin{itemize}
   \item \href{IntroduzioneGeometria.pdf}{Introduzione allo studio della Geometria}
  \end{itemize}  

\subsection*{Geometria Euclidea}
  \begin{itemize}   
   \item \href{./GeometriaEuclidea.pdf}{Geometria euclidea}	
   \item Punto
   \item Retta
   \item Rette ortogonali
   \item Rette sghembe
   \item Piano
   \item Fasci
   \item \href{./Vettore.pdf}{Vettore - probabilmente da spostare in algebra lineare}
   \item \href{./Reference.pdf}{Reference vs riferimento}
   \item \href{./SistemaRiferimento.pdf}{Sistema di riferimento}
   \item \href{./SistemaCartesiano.pdf}{Sistema cartesiano}   
   \item \href{./GeometriaEuclidea3d.pdf}{Spazio tridimensionale della geometria euclidea, geometria euclidea dello spazio, spazio euclideo}
   \item \href{./Segmento.pdf}{Segmento}
   \item \href{./InsiemeVettoriApplicati.pdf}{Insieme di tutti i vettori dello spazio applicati in un punto $O$}
   \item \href{./Piano.pdf}{Piano}
   \item \href{./PianoPassantePunto.pdf}{Piano passante per un punto e ortogonale ad un vettore}
   \item \href{./EquazioneCartesianaPiano.pdf}{Equazione cartesiana del piano}
   \item \href{./EquazioneCartesianaRettaPiano.pdf}{Equazione cartesiana della retta nel piano}
   \item \href{./EquazioneParametricaRettaSpazio.pdf}{Equazione parametrica della retta nello spazio}
   \item \href{./EquazioneCartesianaRetta.pdf}{Equazione cartesiana della retta nello spazio}
   \item \href{./RetteComplanari.pdf}{Rette complanari}
   \item \href{./Superficie.pdf}{Superficie}
  \end{itemize}

\subsection*{Forme Bilineari}
  \begin{itemize}
   \item Forme bilineari: Matrice associata a una forma bilineare. 
   \item Forme simmetriche e antisimmetriche. 
   \item Basi ortogonali. 
   \item Esistenza di basi ortogonali per le forme simmetriche. 
   \item Forme bilineari simmetriche reali. 
   \item Teorema di Sylvester. 
   \item Teorema spettrale reale. 
   \item Cenno alle forme hermitiane e al teorema spettrale complesso.  
   \item Coniche e quadriche di $R^n$ e loro classificazione affine ed euclidea.
  \end{itemize}
  
\subsection*{Geometria Proiettiva}

\subsection*{Coniche}  
  
\end{document}
