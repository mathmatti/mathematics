\documentclass[a4paper,10pt]{article}
\usepackage[utf8]{inputenc}
\usepackage{amsmath}
\usepackage{hyperref}

%opening
\title{Gruppo}
\author{www.baudo.hol.es}

\begin{document}

\maketitle

\section{Definizione}
Si chiama \textit{Gruppo} un qualunque insieme non vuoto $G$ in cui è possibile definire un'operazione da $G$ in $G$ che abbia le seguenti caratteristiche:
\begin{itemize}
 \item $a, b \in G$ implies that $a*b \in G$. (We describe this by saying that $G$ is \textit{closed} under $*$).
 \item Given $a, b, c \in G$, then $a * (b*c) = (a*b)*c$. (This is described by saying that the \textit{associative law} holds in $G$).
 \item There exists a special element $e \in G$ such that $a*e=e*a=a$ for all $a \in G$. ($e$ is called the \textit{identity} or \textit{unit element} of $G$).
 \item For every $a \in G$ there exists an element $b \in G$ such that $a*b=b*a=e$. (We write this element $b$ as $a^{-1}$ and call it the \textit{inverse} of $a$ in $G$).
\end{itemize}

\section{Esempio}
L'insieme $A(S)$ di tutte le permutazioni con l'operazione di composizione tra funzioni.
\end{document}
