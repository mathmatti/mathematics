\documentclass[a4paper,10pt]{article}
\usepackage[utf8]{inputenc}
\usepackage{amsmath}
\usepackage{hyperref}

%opening
\title{Calcolare le permutazioni di $S_{8}$, periodo e segno, inversa di una permutazione, Sottogruppo ciclico dell permutazioni}
\author{baudo81[at]gmail.com}

\begin{document}

\maketitle


\section{TESTO}
Si considerino le seguenti permutazioni di $S_{8}$:
\[
 \alpha = \left( \begin{array}{cccccccc}
                  1 & 2 & 3 & 4 & 5 & 6 & 7 & 8 \\
                  3 & 8 & 2 & 1 & 7 & 5 & 6 & 4 \\
                 \end{array}
 \right)
\]

\[
  \beta = \left( \begin{array}{cccccccc}
                  1 & 2 & 3 & 4 & 5 & 6 & 7 & 8 \\
                  7 & 3 & 6 & 4 & 2 & 5 & 1 & 8 \\
                 \end{array}
 \right)
\]

\[
  \gamma = \left( \begin{array}{cccccccc}
                  1 & 2 & 3 & 4 & 5 & 6 & 7 & 8 \\
                  1 & 7 & 2 & 8 & 4 & 6 & 3 & 5 \\
                 \end{array}
 \right)
\]

\begin{itemize}
 \item Calcolare il periodo ed il segno delle tre permutazioni.
 \item Calcolare l'inversa di ciascuna permutazione.
 \item Determinare esplicitamente gli elementi del sottogruppo ciclico generato da $\beta$.
\end{itemize}


\section{TEORIA}
  \begin{itemize}
   \item \href{./Permutazione.html}{Permutazione}
   \item \href{./PermutazioniCicli.html}{Ciclo di una permutazione}
   \item \href{./OrderOfGroup.html}{Order of a group}
   \item Segno di una permutazione
   \item Inversa di una permutazione
   \item Sottogruppo ciclico generato da una permutazione
  \end{itemize}

\section{SOLUZIONE}
\begin{itemize}
 \item $\alpha = (13284)(576)$, $\beta = (17)(2365)$, $\gamma = (273)(485)$  
 Siccome il segno di un prodotto di cicli è il prodotto dei segni e siccome il segno di un ciclo di lunghezza $s$ è $(-1)^{s-1}$, 
 abbiamo che il segno di $\alpha$, $\beta$ e $\gamma$ è sempre 1.
 \item  
 \[
 \alpha^{-1} = \left( \begin{array}{cccccccc}
                  1 & 2 & 3 & 4 & 5 & 6 & 7 & 8 \\
                  4 & 3 & 1 & 8 & 6 & 7 & 5 & 2 \\
                 \end{array}
 \right)
\]

\[
  \beta^{-1} = \left( \begin{array}{cccccccc}
                  1 & 2 & 3 & 4 & 5 & 6 & 7 & 8 \\
                  7 & 5 & 2 & 4 & 6 & 3 & 1 & 8 \\
                 \end{array}
 \right)
\]

\[
  \gamma^{-1} = \left( \begin{array}{cccccccc}
                  1 & 2 & 3 & 4 & 5 & 6 & 7 & 8 \\
                  1 & 3 & 7 & 5 & 8 & 6 & 2 & 4 \\
                 \end{array}
 \right)
\]
  
  \item
  Siccome il periodo di $\beta$ è 4 abbiamo che il sottogruppo generato da $\beta$ conterra $id$, $\beta$, $\beta^2$, $\beta^3$ e si avrà
  
  \[
  \beta^{2} = \left( \begin{array}{cccccccc}
                  1 & 2 & 3 & 4 & 5 & 6 & 7 & 8 \\
                  1 & 6 & 5 & 4 & 3 & 2 & 7 & 8 \\
                 \end{array}
 \right)
 \]
 
  \[
  \beta^{3} = \left( \begin{array}{cccccccc}
                  1 & 2 & 3 & 4 & 5 & 6 & 7 & 8 \\
                  7 & 5 & 2 & 4 & 6 & 3 & 1 & 8 \\
                 \end{array}
 \right)
 \].
 
 
\end{itemize}


\end{document}
