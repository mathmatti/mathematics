\documentclass[a4paper,10pt]{article}
\usepackage[utf8]{inputenc}
\usepackage{amsmath}
\usepackage{hyperref}

%opening
\title{Matrice diagonalizzabile}
\author{\href{http://www.baudo.hol.es}{giuseppe baudo}}

\begin{document}

\maketitle

\section{DEFINIZIONE}
Una matrice è diagonalizzabile se:
\begin{enumerate}
 \item Il numero degli autovalori, contati con la loro molteplicità, sia pari all'ordine della matrice.
 \item La molteplicità geometrica di ciascun autovettore coincida con la relativa molteplicità algebrica.
\end{enumerate}


\section{NOTAZIONE}

\section{ESEMPIO}

\section{APPROFONDIMENTI}
\begin{itemize}
 \item \url{http://www.youmath.it/lezioni/algebra-lineare/matrici-e-vettori/1581-matrice-diagonalizzabile.html}
\end{itemize}

\end{document}
