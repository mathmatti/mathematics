\documentclass[a4paper,10pt]{article}
\usepackage[utf8]{inputenc}
\usepackage{amsmath}
\usepackage{hyperref}

%opening
\title{Vettori ortogonali (perpendicolari)}
\author{\href{http://www.baudo.hol.es}{giuseppe baudo}}

\begin{document}

\maketitle

\section{DEFINIZIONE}
Due vettori si dicono ortogonali (o perpendicolari) se il loro prodotto scalare è uguale a zero.

\section{NOTAZIONE}

\section{ESEMPIO}

\section{APPROFONDIMENTI}
\begin{itemize}
 \item \url{http://www1.mat.uniroma1.it/people/garroni/pdf/Lezione3.pdf}
\end{itemize}

\end{document}
