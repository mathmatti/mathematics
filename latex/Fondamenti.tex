\documentclass[a4paper,10pt]{article}
\usepackage[utf8]{inputenc}
\usepackage{amsmath}
\usepackage{hyperref}

%opening
\title{TITLE}
\author{\href{http://www.baudo.hol.es}{giuseppe baudo}}

\begin{document}

\maketitle


\section{DA SISTEMARE}
\begin{itemize}
 \item \href{HowToStudy.html}{How to study mathematics}
  \item \href{Metamathematics.html}{Metamathematics}
  \item \href{Logics.html}{Logica}
  \item \href{QuantumTheory.html}{Quantum Theory}
  \item \href{history.html}{History} 
\end{itemize}

\section{Set Theory}

\begin{itemize}
 \item Insieme, famiglia, classe, collezione.
 \item Appartenenza
 \item \href{CartesianProduct.html}{Cartesian product}
 \item \href{Relation.html}{Relation between sets}
 \item inf e sup di un insieme.
 \item Unione tra insiemi
 \item Intersezione tra insiemi
 \item \href{OrdinamentoInsieme.html}{Ordinamento di un insieme (lemma di Zorn)}
 \item Significato di $A - B$ dove $A$ e $B$ sono insiemi.
\end{itemize}


\begin{itemize}
 \item \href{./Extensionality.html}{Axiom of extensionality}
 \item \href{./EmptySet.html}{Axiom of empty set}
 \item Axiom of pairing
 \item Axiom of union
 \item Axiom of infinity
 \item Axiom of power set
 \item Axiom of regularity
 \item Axiom schema of specification
 \item Axiom of choice
\end{itemize}


\section{Peano}
\begin{itemize}
 \item 
\end{itemize}

\section{Church's Postulates}
\begin{itemize}
 \item 
\end{itemize}

\section{Hilbert}
\begin{itemize}
 \item 
\end{itemize}

\end{document}
