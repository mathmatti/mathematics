\documentclass[a4paper,10pt]{article}
\usepackage[utf8]{inputenc}
\usepackage{amsmath}
\usepackage{amsthm}
\usepackage{hyperref}

%opening
\title{Operazione di prodotto tra matrici}
\author{\href{http://www.baudo.hol.es}{giuseppe baudo}}

\begin{document}
	
	\theoremstyle{definition}
	\newtheorem{definition}{Definition}[section]

\maketitle

\section{Introduzione}
Sotto talune condizione riguardo ai tipi di matrice, cio\'{e} sul numero delle righe e delle colonne che le matrici che si intendono moltiplicare devono avere, definiamo un'operazione
che chiamiamo \textit{operazione di prodotto tra matrici} o \textit{prodotto tra matrici} o \textit{prodotto riga per colonna}.
Bene, diciamolo subito, tra tutte le matrici dell'insieme $M_{m,n}$ \'{e} possibile moltiplicare tra di loro quelle in cui il numero di colonne della prima matrice corrisponde al
numero di righe della seconda matrice. Per questo motivo viene chiamato prodotto righe per colonne.

Altra cosa da tenere presente \'{e} il fatto che la forma della matrice risultato pu\'{o} essere diversa da quella delle matrici che abbiamo moltiplicato. Ed infatti, la matrice risultato
sar\'{a} composta da $m$ righe ed $n$ colonne, dove, $m$ \`{e} il numero di righe della prima matrice e $n$ \`{e} il numero di colonne della seconda matrice.

\begin{definition}
 Se $A$ \`{e} una matrice $m \times s$ e $B$ \`{e} una matrice $s \times n$, definiamo il prodotto $c_{ij}$ della riga $i$ di $A$ e della colonna $j$ di $B$ nel modo seguente:
 \[
  c_{ij} = \begin{pmatrix}
	      a_{i1} & ... & a_{is}
           \end{pmatrix}           
         = \begin{pmatrix}
	      b_{1j} \\
	      ... \\
	      b_{sj}
           \end{pmatrix}
         = a_{i1}b_{1j} + ... + a_{is}b_{sj}
         = \sum_{h=1}^s a_{ih}b_{hj}
 \]

\end{definition}

In altre parole, per ricavare l'elemento di posto $(i,j)$ bisogna eseguire il prodotto riga per colonna tra la riga $i$ e la colonna $j$.

Analogamente all'operazione di somma tra matrici, quello che \`{e} importante capire \`{e} quale formula \`{e} utilizzata per ricavare l'elemento di posto $(i,j)$. Solo che,
nel caso della somma tra matrici, l'operazione \`{e} banale ed immediata, qui invece l'operatore di sommatoria potrebbe dare qualche problema agli inizi, ma poi ci si fa l'abitudine.


\end{document}
