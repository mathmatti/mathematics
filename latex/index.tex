\documentclass[a4paper,10pt]{article}
\usepackage[utf8]{inputenc}
\usepackage{amsmath}
\usepackage{hyperref}

%opening
\title{Mathematics}
\author{\href{http://www.baudo.hol.es}{giuseppe baudo}}

\begin{document}
	
	\maketitle
	
	\section{Introduzione}		
	Mathematics è il tentativo di raccolta di appunti e materiale per studiare con profitto nei corsi di laurea in Matematica.
	
	Il programma di riferimento \`{e}, principalmente, quello dell'Universit\'{a}
	di Bologna, anche se, non essendo iscritto a nessuno corso, ho cercato di integrare i programmi (syllabus) con i corsi di altre Universit\'{a}. Diciamo che la matematica \'{e} tale.
	Comunque, nel confrontare i vari corsi proposti dalle varie universit\'{a} mi sono accorto che argomenti trattati in geometria vengano trattati in altri corsi denominati algebra lineare oppure, geometria e algebra.
	
	Pertanto, mi sono preso la libert\'{a} di organizzare tutti i concetti di tutte le materie in un unico corpo. Questo punto non \`{e} banale perch\'{e} apre alcune questioni oggi
	approfondite in corsi quali \textit{Interazione persona-computer}, \textit{Semantic web}, \textit{Intelligenza artificiale}, \textit{Logica}, \textit{Machine learning}, \textit{Mathematical Knowledge Management}, etc.
	
	Il materiale raccolto \'{e} ancora in fase embrionale. 
	
	
	
	\href{Syllabus.html}{Go to Syllabus}
	
\end{document}