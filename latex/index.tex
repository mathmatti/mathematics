\documentclass[a4paper,10pt]{article}
\usepackage[utf8]{inputenc}
\usepackage{amsmath}
\usepackage{hyperref}

%opening
\title{Mathematics}
\author{\href{http://www.baudo.hol.es}{giuseppe baudo}}

\begin{document}

\maketitle

\section{Introduzione}
Mathematics \`{e} il tentativo di raccolta di un certo tipo di sapere (quello matematico) capace di mettermi in grado di leggere un paper.
In realt\'{a} tale \`{e} uno degli scopi dei corsi di laurea in matematica delle universita del pianeta. Il programma di riferimento \`{e}, principalmente, quello dell'Universit\'{a}
di Bologna, anche se, non essendo iscritto a nessuno di tali corsi, ho cercato di integrare i programmi (syllabus) con i corsi di altre Universit\'{a}. Diciamo che la matematica \'{e} tale.
Comunque, nel confrontare i vari corsi proposti dalle varie universit\'{a} mi sono accorto che argomenti trattati in geometria vengano trattati in altri corsi denominati algebra lineare oppure, geometria e algebra.

Pertanto, mi sono preso la libert\'{a} di organizzare tutti i concetti di tutte le materie in un unico corpo. Questo punto non \`{e} banale perch\'{e} apre alcune questioni oggi
approfondite in corsi quali \textit{Interazione persona-computer}, \textit{Semantic web}, \textit{Intelligenza artificiale}, \textit{Logica}, \textit{Machine learning}, \textit{Mathematical Knowledge Management}.

Prima di leggere un paper, per\'{o} occorre dimistichezza con le modalit\'{a} di ragionamento che sono utilizzate nell'esposizione dei concetti, e dei relativi segni.



\section{INDICE}
\begin{itemize}
  \item \href{HowToStudy.html}{How to study mathematics}
  \item \href{SetTheory.html}{Set Theory}
  \item \href{AssiomiPeano.html}{Assiomi di Peano}
  \item \href{AlgebraIndex.html}{Algebra} 
  \item \href{Analisi.html}{Analisi} 
  \item \href{Aritmetica.html}{Aritmetica}
  \item \href{AlgebraLineare.html}{Algebra Lineare}
  \item \href{Geometria.html}{Geometria}
  \item \href{Metamathematics.html}{Metamathematics}
  \item \href{Logics.html}{Logica}
  \item \href{QuantumTheory.html}{Quantum Theory}
  \item \href{Journals.html}{Journals}
  \item \href{MatematicaComputazionale.html}{Matematica Computazionale}
  \item \href{Programming.html}{Programming}
  \item \href{finance.html}{Finance} 
  \item \href{history.html}{History} 
  \item \href{latex.html}{Latex} 
  \item \href{Eserciziario.html}{Eserciziario}
  
\end{itemize}

\end{document}
