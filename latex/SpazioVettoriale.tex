\documentclass[a4paper,10pt]{article}
\usepackage[utf8]{inputenc}
\usepackage{amsmath}
\usepackage{hyperref}

%opening
\title{Spazio Vettoriale}
\author{\href{http://www.baudo.hol.es}{giuseppe baudo}}

\begin{document}

\maketitle

\section{DEFINIZIONE}
Si chiama Spazio Vettoriale una qualunque struttura matematica che possiede determinate propriet\'{a}.

\section{Descrizione}
Affinch\'{e} si possa parlare di spazio vettoriale occorrono almeno due insiemi generalmente indicati con $V$ e $K$. Tra gli elementi di $V$ \'{e} definita
una funzione che associa ad ogni coppia di elementi di $V$ un elemento di $V$. Tale funzione viene chiamata operazione di somma.
\section{Propriet\'{a}}


\section{ESEMPIO}
L'insieme dei numeri reali con l'operazione di somma.

\section{APPROFONDIMENTI}
\begin{itemize}
 \item TITLE
\end{itemize}

\end{document}
