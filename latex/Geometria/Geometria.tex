\documentclass[a4paper]{book}
\usepackage{geometria}
\hypersetup{
	pdftitle={Geometria 1},
	pdfauthor={Davide Ferracin},
	pdflang={it},
	pdfsubject={Appunti del corso di Geometria 1},
	pdfcreator={Davide Ferracin},
	pdfcaptionwriter={Davide Ferracin}
	pdfcopyright={Creative Commons Attribution-ShareAlike 4.0 International},
	pdflicenseurl={http://creativecommons.org/licenses/by-sa/4.0/},
	pdfmetalang={it}
}

\title{
	{\sffamily\fontsize{35}{42}\selectfont Geometria 1}
}
\author{
	{\small a cura di}\\
	Davide Ferracin
}
\date{}

\pagestyle{headings}

\begin{document}

\frontmatter
\maketitle
\tableofcontents

\mainmatter
\chapter{Fondamenti}
	\documentclass[a4paper,10pt]{article}
\usepackage[utf8]{inputenc}
\usepackage{amsmath}
\usepackage{hyperref}

%opening
\title{Introduzione allo studio della Geometria}
\author{\href{http://www.baudo.hol.es}{giuseppe baudo}}

\begin{document}

\maketitle

\section{Introduzione}
Lo scopo di queste pagine è di mettere lo studente in grado di apprendere in maniera autonoma i concetti e gli strumenti della geometria.
Le pagine sono rivolte a studenti dei corsi di Matematica in quanto le conoscenze sono presentate sotto forma di definizioni, teoremi e relative dimostrazioni applicando le tecniche e gli strumenti della logica matematica.
I concetti, seppur presentati il più possibile in maniera formale, sono accompagnati da commenti scritti con il linguaggio dello studente.
Infatti, vengono associate al formalismo puro tipico della matematica moderna un linguaggio comune e comprensibile a tutti. Per questo si cerca di spiegare parola per
parola il significato dei termini delle definizioni, dei teoremi e di ogni altro enunciato. Le pagine di teoria sono accompagnate da
esercitazioni pratiche sia di esercizi ed esami svolti e sia di esercizi da svolgere in autonomia. Altro obiettivo è quello di contenere
in un'unica trattazione tutto il sapere matematico necessario per superare tutti gli esami di geometria proposti nei corsi universitari
di Matematica e di ogni altro corso di laurea che include anche lo studio della geometria. Le pagine sono strutturate in un indice in cui si trova
il syllabus (il nostro programma di studio) e di una pagina per ogni concetto. Le pagine sono scritte in latex e trasformate in html per la
visualizzazione. E' possibile convertire anche in formato pdf.

\subsection{I link}
Ogni pagina che espone un concetto contiene un certo numero di link. I link che mostrarno l'url, per esempio: \url{http://calvino.polito.it/~tedeschi/geometria/}, sono link 
esterni al documento. I link che non mostrano l'url, per esempio: \href{./GeometriaEuclidea.html}{Geometria euclidea}, sono link interni.
Se cliccate su un link esterno relativo a esercizi allora tutto bene ma se cliccate link esterni di teoria può darsi che la pagina non sia
completa o sia scritta in maniera non del tutto comprensibile. In quest'ultimo caso potete avvisare i partecipanti al progetto, aprendo un issue su github. 

\subsection{Appunti delle lezioni}

\section{Prerequisites}
Costanza nello studio, dedicare il giusto tempo, meglio un pò al giorno tutti i giorni.


\end{document}



\chapter{Geometria Euclidea}

	\documentclass[a4paper,10pt]{article}
\usepackage[utf8]{inputenc}
\usepackage{amsmath}
\usepackage{hyperref}

%opening
\title{Geometria euclidea}
\author{\href{http://www.baudo.hol.es}{giuseppe baudo}}

\begin{document}

\maketitle

\section{DEFINIZIONE}
E' la classica geometria che si studia alle scuole elementari, medie e superiori. E' quella che si basa sugli Elementi di Euclide dove
i concetti fondamentali (ovvero gli assiomi) sono quelli di punto, retta, etc.

\section{NOTAZIONE}

\section{ESEMPIO}

\section{APPROFONDIMENTI}


\end{document}
	
	\documentclass[a4paper,10pt]{article}
\usepackage[utf8]{inputenc}
\usepackage{amsmath}
\usepackage{hyperref}

%opening
\title{Vettore}
\author{\href{http://www.baudo.hol.es}{giuseppe baudo}}

\begin{document}

\maketitle

\section{DEFINIZIONE}
\subsection{Vettore colonna, vettore riga}
Un vettore in uno spazio $n$-dimensionale è un insieme ordinato formato da $n$ valori. 

\subsection{Vettore in geometria mono, bi e tri-dimensionale}
Un vettore è un oggetto che ha una direzione e una lunghezza. In questo caso si dimostrerà che un vettore può essere
rappresentato come da definizione precedente.

\section{Componenti di un vettore}

\section{Rappresentazione canonica}

\section{Lunghezza di un vettore in $R^n$}
The length of a vector $v$ in $R^n$ is the square root of the sum of the squares of tis components.
\[
 |v|=\sqrt{v^2_1+...+v^2_n}
\]

This is a natural generalization of the Pythagorean Theorem.

\section{dot product or scalar product}
The dot product (or inner product or scalar product) of two $n$-component real vectors is the linear combination of their components.
\[
 u \dotfill v = u_1v_1+...+u_nv_n
\]
squares of its components.


\section{NOTAZIONE}

\section{NOTE}
Le due definizioni sono equivalenti nel senso che si possono rappresentare i vettori della definizione 2 come vettori
della definizione 1.

Attenzione alla definizione di vettore libero.

Attenzione all'uguaglianza tra due vettori. Due vettori sono uguali quando hanno la stessa rappresentazione canonica.

\section{APPROFONDIMENTI}
\begin{itemize}
 \item \url{http://joshua.smcvt.edu/linearalgebra/book.pdf}
 \item \url{https://www.math10.com/en/geometry/vectors-operations/vectors-operations.html}
 \item \url{http://www.math.utah.edu/online/2210/notes/ch13.pdf}
 \item \url{http://www.ncert.nic.in/ncerts/l/lemh204.pdf}
\end{itemize}

\end{document}

	\documentclass[a4paper,10pt]{article}
\usepackage[utf8]{inputenc}
\usepackage{amsmath}
\usepackage{hyperref}

%opening
\title{Reference vs riferimento}
\author{\href{http://www.baudo.hol.es}{giuseppe baudo}}

\begin{document}

\maketitle

\section{DEFINIZIONE}


\section{NOTAZIONE}
reference as puntatore???

\section{ESEMPIO}

\section{APPROFONDIMENTI}
\begin{itemize}
 \item TITLE
\end{itemize}

\end{document}

	\section{Sistema di Riferimento}
\begin{definizione}
from wikipedia italiano:

Si definisce sistema di riferimento l'insieme dei riferimenti o coordinate utilizzate per individuare la posizione di un oggetto
nello spazio. A seconda del numero di riferimenti usati, si può parlare di: sistema di riferimento monodimensionale, bidimensionale, tridimensionale.
\end{definizione}

\begin{definizione}
from wikipedia in english:

In geometry, a coordinate system is a system which uses one or more numbers, or coordinates, to uniquely determine the position of
a point or other geometric element on a manifold such as Euclidean space...
\end{definizione}

\begin{osservazione}
Come dice il nome stesso, dobbiamo fissare un "punto" dal quale osservare i nostri "oggetti". Questo concetto deriva più dalla fisica
che dalla matematica. Vi sono tanti esempi che posso chiarificare. In qualche modo sistema di riferimento si potrebbe anche chiamare
punto di osservazione.

Da notare che per la definizione italiana, sistema di riferimento e sistema di coordinate sono la stessa cosa, sono sinonimi. Mentre
se cerchiamo reference system in inglese ci troviamo qualcosa come "frame of reference" che tratta argomenti di fisica. Quindi nel mondo
anglosassone il nostro sistema di riferimento prende il nome di coordinate system. Peccato però che le due definizioni sono diverse
non solo nella forma ma anche nella sostanza. La definizione inglese, che secondo me è da preferire, mette bene in risalto l'obiettivo
del "sistema delle coordinate" ovvero quello di rappresentare tramite dei numeri o altre strutture matematiche più complesse (per esempio vettori di $R^n$) la 
posizione degli elementi geometrici (punti, rette, solidi, etc).
\end{osservazione}

\subsection{ESEMPIO}
The simplest example of a coordinate system is the identification of points on a line with real numbers using the number line.
In this sytem, an arbitrary piont O (the origin) is chosen on a given line. The coordinate of a point P is defined as the signed
distance from O to P, where the signed distance is the distance taken as positive or negative depending on which side of the
line P lies. Each point is given a unique coordinate and each real numebr is the coordinate of a unique point.

\begin{osservazione}

\begin{itemize}
 \item \url{https://it.wikipedia.org/wiki/Sistema_di_riferimento}
 \item \url{https://en.wikipedia.org/wiki/Coordinate_system}
\end{itemize}
\end{osservazione}



	\documentclass[a4paper,10pt]{article}
\usepackage[utf8]{inputenc}
\usepackage{amsmath}
\usepackage{hyperref}

%opening
\title{Sistema di riferimento cartesiano}
\author{\href{http://www.baudo.hol.es}{giuseppe baudo}}

\begin{document}

\maketitle

\section{DEFINIZIONE}
Un sistema di riferimento cartesiano è un sistema di riferimento formato da n rette ortogonali, intersecantesi tutte in un punto
chiamato origine, su ciascuna delle quali si fissa un orientamento (sono quindi dette orientate) e per le quali si fissa
anche un'unità di misura (cioè si fissa una metrica di solito euclidea) che consente di identificare qualsiasi punto dell'insieme
mediante n numeri reali. In questo caso si dice che i punti di questo insieme sono in uno spazio di dimensione n.

\section{NOTE}


\section{ESEMPIO}

\section{APPROFONDIMENTI}
\begin{itemize}
 \item \href{https://it.wikipedia.org/wiki/Sistema_di_riferimento_cartesiano}{https://it.wikipedia.org/wiki/Sistema_di_riferimento_cartesiano}
\end{itemize}

\end{document}
   
	\documentclass[a4paper,10pt]{article}
\usepackage[utf8]{inputenc}
\usepackage{amsmath}
\usepackage{hyperref}

%opening
\title{Spazio tridimensionale della geometria euclidea, geometria euclidea dello spazio, spazio euclideo}
\author{\href{http://www.baudo.hol.es}{giuseppe baudo}}

\begin{document}

\maketitle

\section{DEFINIZIONE}
Uno spazio euclideo è uno spazio affine in cui valgono gli assiomi e i postulati della geometria euclidea.

\section{NOTAZIONE}

\section{ESEMPIO}

\section{APPROFONDIMENTI}
\begin{itemize}
 \item \url{https://it.wikipedia.org/wiki/Spazio_euclideo}
 \item \url{https://www.britannica.com/topic/Euclidean-space}
 \item \url{http://www.molwick.com/it/relativita/324-geometria-spazio.html}
 \item \url{http://www.dmmm.uniroma1.it/~giuseppe.accascina/Tesi_di_Laurea/2005-Piselli-Geometria_euclidea_dello_spazio/2005-Piselli-Tesi.pdf}
 \item \url{http://www.treccani.it/enciclopedia/tag/spazio-tridimensionale-euclideo/}
\end{itemize}

\end{document}

	\section{Segmento}
\begin{definizione}
Un segmento è un vettore applicato in un punto A e terminante in un punto B.
\end{definizione}

\begin{osservazione}
Il concetto di segmento è la corrispondenza 1 a 1 tra lo studio della geometria con gli strumenti tradizionali ed il concetto di vettore.
\end{osservazione}

\begin{osservazione}
\begin{itemize}
 \item \url{http://calvino.polito.it/~casnati/Geometria15BCG/GeometriaNuovo/GeometriaNuovo7.pdf}
\end{itemize}
\end{osservazione}



	\section{Insieme di Vettori Applicati}
\begin{definizione}
Fissiamo arbitrariamente nello spazio tridimensionale della geometria euclidea un punto $O$ e consideriamo l'insieme di tutti i vettori dello spazio 
applicati in $O$. Tale insieme lo chiameremo $V_{O}^{3}$ (si legge V con O cubo).
\end{definizione}

\begin{osservazione}
In questa definizione si cerca già di trovare un appiglio per passare dallo studio della geometria euclidea classica così come è stata
fatta dai tempi di Euclide ai giorni nostri ovvero nel modo in cui viene studiata nella scuola dell'obbligo alla geometria così come la
si studia nei corsi universitari di Matematica. Stiamo cercando di trovare un modo per studiare, rappresentare etc. la geometria con 
la notazione algebrica (o anche analitica) tipica della matematica pura. Per capire la differenza prendete un libro di geometria della
scuola dell'obbligo e confrontatelo con un libro universitario di geometria.

E' importante in questa definizione iniziare questo passaggio da geometria classica a geometria analitica studiata con gli strumenti
dell'algebra lineare etc.

Adesso fissate bene nella mente l'immagine classica dello spazio rappresentato con gli assi cartesiani e immaginate un qualsiasi punto.

Dimostreremo che è possibile passare dalla rappresentazione classica a quella di $V_{O}^{3}$ a quella di $R^3$.
\end{osservazione}

\begin{osservazione}
\begin{itemize}
 \item \href{http://progettomatematica.dm.unibo.it/GeomSpazio3/Sito/Pagine/tesi.html}{http://progettomatematica.dm.unibo.it/GeomSpazio3/Sito/Pagine/tesi.html}
\end{itemize}
\end{osservazione}


	\section{Piano}
\begin{definizione}
Si intende il piano euclideo quello che si studia nelle scuole dell'obbligo, vedi approfondimenti.
\end{definizione}

\begin{osservazione}
Daremo anche una definizione algebrica.
\end{osservazione}

\begin{osservazione}
\begin{itemize}
 \item \url{https://it.wikipedia.org/wiki/Geometria_euclidea}
 \item \url{https://it.wikipedia.org/wiki/Assioma}
 \item \url{http://www.astrofilibresciani.it/Biblioteca_UAB/Biblioteca/euclid_p.pdf}
 \item \url{http://www.webalice.it/francesco.odetti/EuclideSlidesCpct.pdf}
 \item \url{http://www.scienzaatscuola.it/euclide/index.html}
\end{itemize}
\end{osservazione}


	\documentclass[a4paper,10pt]{article}
\usepackage[utf8]{inputenc}
\usepackage{amsmath}
\usepackage{hyperref}

%opening
\title{Piano passante per un punto e ortogonale ad un vettore}
\author{\href{http://www.baudo.hol.es}{giuseppe baudo}}

\begin{document}

\maketitle

\section{DEFINIZIONE - ASSIOMA CHE DIVENTA DEFINIZIONE??!!!}
Sia $\pi$ il piano passante per un punto $P_0$ ortogonale ad un vettore $n \ne o$. Allora $\pi$ è il luogo
dei punti $P$ dello spazio tali che il vettore $P_0P$ è ortogonale al vettore $n$, ovvero:
\[
 \pi = \{ P \in S_3 | P_0P \cdot n = o  \} 
\]


\section{NOTA}
Qui stiamo facendo una cosa di estrema importanza stiamo formulando in linguaggio matematico moderno quello che per Euclide era un
concetto primitivo non dimostrabile (e non definibile! in un certo senso).

Vorrei capire però se $P_0P \cdot n$ è il prodotto scalare o vettoriale?

\section{ESEMPIO}

\section{APPROFONDIMENTI}
\begin{itemize}
 \item \url{http://calvino.polito.it/~salamon/P/G/alga11.pdf}
\end{itemize}

\end{document}

	\documentclass[a4paper,10pt]{article}
\usepackage[utf8]{inputenc}
\usepackage{amsmath}
\usepackage{hyperref}

%opening
\title{Equazione cartesiana del piano}
\author{\href{http://www.baudo.hol.es}{giuseppe baudo}}

\begin{document}

\maketitle

\section{ENUNCIATO}
Ogni equazione lineare in $x,y e z$ del tipo $ax+by+cz+d=0$ rappresenta, a meno di un fattore moltiplicativo non nullo,
l'equazione cartesiana di un piano nello spazio $S_3$ (chi essere $S_3$???).

\section{DIMOSTRAZIONE}

\section{NOTE}
In realtà qui la questione riguarda un teorema il cui risultato è utilizzatissimo nelle applicazioni pratiche.

Si dimostrerà che ogni equazione di primo grado in $x$, $y$ e $z$ del tipo:
\[
 ax+by+cz+d=0
\]

con $a,b,c,d \in R$ e $a,b,c$ non contemporaneamente tutti uguali a zero, $(a,b,c) \ne (0,0,0)$ rappresenta un piano. 
Viceversa, ogni piano dello spazio è rappresentabile tramite un'equazione lineare in $x,y,z$ del tipo suddetto.


\section{ESEMPIO}

\section{APPROFONDIMENTI}
\begin{itemize}
 \item http://progettomatematica.dm.unibo.it/GeomSpazio3/Sito/Pagine/indiceFRAME.html
 \item http://calvino.polito.it/~salamon/P/G/alga11.pdf
\end{itemize}

\end{document}

	\section{Equazione Cartesiana Retta sul Piano}

\begin{osservazione}
\begin{itemize}
 \item \url{https://it.wikipedia.org/wiki/Retta_nel_piano_cartesiano}
\end{itemize}
\end{osservazione}


	\documentclass[a4paper,10pt]{article}
\usepackage[utf8]{inputenc}
\usepackage{amsmath}
\usepackage{hyperref}

%opening
\title{Equazione parametrica della retta nello spazio}
\author{\href{http://www.baudo.hol.es}{giuseppe baudo}}

\begin{document}

\maketitle

\section{DEFINIZIONE (1)}
L'equazione parametrica di una retta parallela al vettore (non nullo) $(a,b,c)$ è passante per il punto $(x_0,y_0,z_0)$ è

 \begin{equation}
   \begin{cases}
   x=x_0 + ta \\
   y=y_0 + tb \\
   z=z_0 + tc
   \end{cases}
\end{equation}
con $t \in R$

\section{NOTE}
Trovare le equazioni parametriche e cartesiane di una retta nello spazio passante per due punti: 
\url{http://www.matematicamente.it/forum/retta-passante-per-due-punti-nello-spazio-t48348.html}

\section{NOTAZIONE}

\section{ESEMPIO}

\section{APPROFONDIMENTI}
\begin{itemize}
 \item \url{http://www.matematicamente.it/formulario-dizionario/formulario/geometria-analitica-nello-spazio-retta/}
 \item \url{}
\end{itemize}

\end{document}

	\section{Equazione Cartesiana della Retta}

\begin{osservazione}
\begin{itemize}
 \item \url{http://www.youmath.it/lezioni/algebra-lineare/geometria-dello-spazio/675-equazioni-cartesiane-della-retta-nello-spazio.html}
 \item \url{http://www1.mat.uniroma1.it/people/garroni/pdf/Lezione7.pdf}
 \item \url{http://science.unitn.it/~carrara/INGTN10_11/corres04.pdf}
\end{itemize}
\end{osservazione}


	\documentclass[a4paper,10pt]{article}
\usepackage[utf8]{inputenc}
\usepackage{amsmath}
\usepackage{hyperref}

%opening
\title{Rette complanari}
\author{\href{http://www.baudo.hol.es}{giuseppe baudo}}

\begin{document}

\maketitle

\section{DEFINIZIONE}
Due rette si dicono complanari se appartengono allo stesso piano.

\section{NOTE}
Calcolare se due rette sono complanari:
\begin{itemize}
 \item \url{http://www.youmath.it/domande-a-risposte/view/6183-rette-complanari.html}
\end{itemize}

\section{ESEMPIO}

\section{APPROFONDIMENTI}
\begin{itemize}
 \item \url{http://www.youmath.it/domande-a-risposte/view/6183-rette-complanari.html}
\end{itemize}

\end{document}

	\documentclass[a4paper,10pt]{article}
\usepackage[utf8]{inputenc}
\usepackage{amsmath}
\usepackage{hyperref}

%opening
\title{Superficie}
\author{\href{http://www.baudo.hol.es}{giuseppe baudo}}

\begin{document}

\maketitle

\section{DEFINIZIONE}

\section{NOTAZIONE}

\section{ESEMPIO}

\section{APPROFONDIMENTI}
\begin{itemize}
 \item TITLE
\end{itemize}

\end{document}



\chapter{Forme Bilineari}
\begin{itemize}
	\item Forme bilineari: Matrice associata a una forma bilineare. 
	\item Forme simmetriche e antisimmetriche. 
	\item Basi ortogonali. 
	\item Esistenza di basi ortogonali per le forme simmetriche. 
	\item Forme bilineari simmetriche reali. 
	\item Teorema di Sylvester. 
	\item Teorema spettrale reale. 
	\item Cenno alle forme hermitiane e al teorema spettrale complesso.  
	\item Coniche e quadriche di $R^n$ e loro classificazione affine ed euclidea.
\end{itemize}

\chapter{Geometria Proiettiva}

\chapter{Coniche}  

\chapter{Dati di lavoro}
da aggiungere in geometria euclidea:
\begin{itemize}
	\item Punto
	\item Retta
	\item Rette ortogonali
	\item Rette sghembe
	\item Piano
	\item Fasci
\end{itemize}



\backmatter
% bibliography, glossary and index would go here.

\end{document}