\section{Insieme di Vettori Applicati}
\begin{definizione}
Fissiamo arbitrariamente nello spazio tridimensionale della geometria euclidea un punto $O$ e consideriamo l'insieme di tutti i vettori dello spazio 
applicati in $O$. Tale insieme lo chiameremo $V_{O}^{3}$ (si legge V con O cubo).
\end{definizione}

\begin{osservazione}
In questa definizione si cerca già di trovare un appiglio per passare dallo studio della geometria euclidea classica così come è stata
fatta dai tempi di Euclide ai giorni nostri ovvero nel modo in cui viene studiata nella scuola dell'obbligo alla geometria così come la
si studia nei corsi universitari di Matematica. Stiamo cercando di trovare un modo per studiare, rappresentare etc. la geometria con 
la notazione algebrica (o anche analitica) tipica della matematica pura. Per capire la differenza prendete un libro di geometria della
scuola dell'obbligo e confrontatelo con un libro universitario di geometria.

E' importante in questa definizione iniziare questo passaggio da geometria classica a geometria analitica studiata con gli strumenti
dell'algebra lineare etc.

Adesso fissate bene nella mente l'immagine classica dello spazio rappresentato con gli assi cartesiani e immaginate un qualsiasi punto.

Dimostreremo che è possibile passare dalla rappresentazione classica a quella di $V_{O}^{3}$ a quella di $R^3$.
\end{osservazione}

\begin{osservazione}
\begin{itemize}
 \item \href{http://progettomatematica.dm.unibo.it/GeomSpazio3/Sito/Pagine/tesi.html}{http://progettomatematica.dm.unibo.it/GeomSpazio3/Sito/Pagine/tesi.html}
\end{itemize}
\end{osservazione}

