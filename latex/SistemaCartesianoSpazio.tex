\documentclass[a4paper,10pt]{article}
\usepackage[utf8]{inputenc}
\usepackage{amsmath}
\usepackage{hyperref}

%opening
\title{Sistema di riferimento cartesiano ortogonale monometrico nello spazio}
\author{\href{http://www.baudo.hol.es}{giuseppe baudo}}

\begin{document}

\maketitle

\section{DEFINIZIONE}
Un sistema di riferimento cartesiano ortogonale monometrico R.C.O. $(O,x,y,z)$ nello spazio è dato assegnando:
\begin{itemize}
 \item un punto $O$, detto origine,
 \item tre rette passanti per $O$, ortogonali a due a due, dette assi cartesiani o assi coordinati, su cui siano fissati rispettivamente tre sistemi di riferimento cartesiani R.C.$(O,x)$, R.C.$(O,y)$, R.C.$(O,z)$ aventi la stessa unità di misura.
\end{itemize}
Gli assi coordinati a due a due individuano tre piani, che sono detti piani coordinati.

\section{NOTE}
E' utilizzato per ottenere una corrispondenza biunivoca tra i punti dello spazio e le terne ordinate di numeri reali. Da questo momento in poi anzichè utilizzare segmenti utilizzeremo spazi vettoriali di dimensione 3!!!

\section{ESEMPIO}

\section{APPROFONDIMENTI}
\begin{itemize}
 \item \href{http://progettomatematica.dm.unibo.it/GeomSpazio3/Sito/Pagine/tesi.html}{http://progettomatematica.dm.unibo.it/GeomSpazio3/Sito/Pagine/tesi.html}
 \item \href{https://it.wikipedia.org/wiki/Sistema_di_riferimento_cartesiano}{https://it.wikipedia.org/wiki/Sistema_di_riferimento_cartesiano}
 \item \href{http://calvino.polito.it/~falletta/GEO_ALG/VETTORI_APPLICATI/vettori_applicati.pdf}{http://calvino.polito.it/~falletta/GEO_ALG/VETTORI_APPLICATI/vettori_applicati.pdf}
\end{itemize}

\end{document}
