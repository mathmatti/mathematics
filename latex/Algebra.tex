\documentclass[a4paper,10pt]{article}
\usepackage[utf8]{inputenc}
\usepackage{amsmath}
\usepackage{hyperref}

%opening
\title{Algebra}
\author{}

\begin{document}

\maketitle

\section{Syllabus}
\begin{itemize}
 \item Funzione
 \item Operazione
 \item Operazione di somma tra numeri
 \item Operazione di somma tra vettori
 \item Operazione di somma tra matrici
 \item \href{./FunzioneSuriettiva.html}{Funzione suriettiva (o surgettiva, o suriezione)} 
 \item \href{./Permutazione.html}{Permutazione}
 \item Insieme di tutte le permutazioni su un insieme $X$
 \item \href{./PermutazioneCicli.html}{Ciclo di una permutazione, l-ciclo}
 \item \href{./PermutazioniDisgiunte.html}{Cicli disgiunti}
 \item \href{./PermutazioneDecomposizioneCicli.html}{Decomposizione in cicli disgiunti}
 \item \href{./PermutazioneCicliPeriodo.html}{Ordine (o periodo) di un ciclo}
 \item \href{./PermutazionePeriodo.html}{Ordine (o periodo) di una permutazione}
 \item TEOREMA: Ogni permutazione si può scrivere come una composizione di cicli disgiunti in modo unico a meno dell'ordine dei cicli stessi
 \item \href{./Derangement.html}{Permutazione senza punti fissi, Derangement, partial derangement}
 \item \href{./PermutazioneTrasposizione.html}{Trasposizione di una permutazione}
 \item \href{./PermutazioneSegno.html}{Segno di una permutazione}
 \item \href{./PermutazioneOrbita.html}{Orbita di una permutazione}
 \item Sottogruppo ciclico del gruppo simmetrico $S_n$ - Sottogruppo ciclico generato da una permutazione
 \item \href{./CoefficientiBinomiali.html}{Coefficienti binomiali: definizione e proprietà}
 \item \href{./PrincipioInclusioneEsclusione.html}{Principio di inclusione-esclusione}
 
 \item \href{./Gruppo.html}{Gruppo}
 \item Elemento neutro di un gruppo
 \item Sottogruppo
 \item Gruppo abeliano
 \item \href{./OrderOfGroup.html}{Order of a group}
 \item \href{./TeoremaDiLagrange.html}{Teorema di Lagrange e sue conseguenze} 
 \item \href{./CyclicGroup.html}{Gruppo ciclico}
 
 \item Gruppo generale lineare
 \item Gruppo simmetrico 
 \item Gruppo classi resto modulo n rispetto alla somma
 \item Gruppo di permutazioni
 
 \item Omomorfismo
 \item Isomorfismo
 \item Omomorfismo/Isomorfismo tra ...
 \item \href{./OmomorfismoGruppi.html}{Omomorfismo di gruppi}
 \item Nucleo, nucleo di una funzione, nucleo di applicazione lineare
 \item \href{Nucleo.html}{Nucleo di omomorfismo di gruppi}
 \item Insieme degli omomorfismo di gruppi ( dati due gruppi $G$ e $H$ in simboli: $Hom(G,H)$)
 
 \item Anelli
 \item Anelli commutativi: 0-divisori, elementi nilpotenti, unità.
 \item Domini, campi. 
 \item Estensioni quadratiche. 
 \item Gli interi di Gauss. 
 \item Domini euclidei:  ideali principali, esistenza del massimo comun divisore, scomposizione in fattori irriducibili. 
 \item Morfismi di anelli; il morfismo da Z ad un anello; la caratteristica. 
 \item Ideali e anelli quozienti; ideale generato da un sottoinsieme. 
 \item La fattorizzazione di un morfismo di anelli. 
 \item Il campo dei quozienti di un dominio; Q, K(X). 
 \item Divisibilità in un anello. 
 \item Ideali primi e massimali.
 \item Il campo dei numeri complessi;
 \item rappresentazione geometrica dei numeri complessi; 
 \item teorema di De Moivre; 
 \item l'insieme delle radici n-esime di un numero complesso come laterale rispetto al sottogruppo delle radici n-esime dell'unità.
 \item L'anello dei polinomi in una indeterminata a coefficienti in un anello, 
 \item serie formali, 
 \item funzioni polinomiali, 
 \item grado di un polinomio e sue proprieta', 
 \item se A e' un dominio anche A[x] lo e'. 
 \item Polinomi a coefficienti in un campo: zeri e fattori lineari; 
 \item il lemma di divisione e le sue conseguenze (K[X] è un dominio euclideo, a fattorizzazione unica, ad ideali principali). 
 \item Il teorema fondamentale dell'algebra. 
 \item La derivata di un polinomio, molteplicità di una radice. 
 \item Polinomi reali.
 \item Quozienti di K[X]; forma ridotta. 
 \item Estensioni di campi; 
 \item elementi algebrici e trascendenti;
 \item polinomio minimo; 
 \item il sottocampo K(u) di un campo F generato dal sottocampo K di F e dall'elemento u. 
 \item Il grado di una estensione finita; 
 \item ogni elemento di una estensione finita è algebrico; 
 \item il grado della composizione di due estensioni finite; 
 \item l'insieme dei numeri algebrici è algebricamente chiuso. 
 \item Campo di spezzamento: esistenza e unicità. 
 \item Esistenza e unicità del campo con $p^n$ elementi; questi sono gli unici campi finiti.
 \item Teoria di Galois
 \item Teoria dei Moduli
\end{itemize}

\end{document}
