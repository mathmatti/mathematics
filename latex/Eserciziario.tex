\documentclass[a4paper,10pt]{article}
\usepackage[utf8]{inputenc}
\usepackage{amsmath}
\usepackage{hyperref}

%opening
\title{Eserciziario}
\author{\href{http://www.baudo.hol.es}{giuseppe baudo}}

\begin{document}

\maketitle

\section{Algebra}
\begin{itemize}
 \item \href{./esercizio1.html}{Esercizio 1: Dimostrare che una matrice è sottogruppo di $GL_{n}$}
 \item \href{./esercizio2.html}{Esercizio 2: Dimostrare che una funzione è omomorfismo di gruppi}
 \item \href{./esercizio3.html}{Esercizio 3: Trovare il nucleo di un omomorfismo di gruppi}
 \item \href{./esercizio4.html}{Esercizio 4: Applicazione delle formule per il derangement, partial derangement}
 \item \href{./esercizio5.html}{Esercizio 5: Calcolare le permutazioni di $S_{8}$, periodo e segno, inversa di una permutazione, Sottogruppo ciclico dell permutazioni}
 \item \href{./esercizio6.html}{Esercizio 6: Si provi o si confuti la seguente affermazione: Date due matrici $A, B \in M_n(R)$ risulta $(AB)^t = B^tA^t$ (dove $A^t$ indica la matrice trasposta di $A$)}
\end{itemize}

\section{Algebra lineare}
\begin{itemize}
 \item \href{./esercizio6.html}{Esercizio 6: Si provi o si confuti la seguente affermazione: Date due matrici $A, B \in M_n(R)$ risulta $(AB)^t = B^tA^t$ (dove $A^t$ indica la matrice trasposta di $A$)}
\end{itemize}


\end{document}
