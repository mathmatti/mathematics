\documentclass[a4paper,10pt]{book}
\usepackage[utf8]{inputenc}
\usepackage{amsmath}
\usepackage{hyperref}

\begin{document}
\href{IndagineLeggiPensiero.pdf}{pdf}
\chapter*{PREFAZIONE}
Il lavoro che segue non \'{e} la ristampa di un trattato precedente dello stesso autore, intitolato \textit{L'analisi matematica della logica}. Le sue prime parti sono, \'{e} vero,
dedicate allo stesso argomento, e il libro incomincia stabilendo il medesimo sistema di leggi fondamentali, ma i suoi metodi sono pi\'{u} generali e il campo delle sue applicazioni
\'{e} di gran lunga pi\'{u} ampio. Il libro espone i risultati, maturati in anni di studio e di riflessione, di un principio d'indagine relativo alle operazioni dell'intelletto, la cui
prima esposizione fu stesa in poche settimane da che fu concepita l'idea.

\textbf{Questo e il secondo lavoro di boole ma e da considerare quello principale. Per fare un analogia struts... Bisogna prima di leggere questo trattato
  leggere altri due trattati che erano le basi per poter leggere il libro di Boole nel 1850.  
  \textit{Elementi di logica} dell'arcivescovo Whately oppure \textit{Lineamenti delle leggi del pensiero} del signor Thomson.
}

\chapter{Natura e scopo dell'opera}
\section{}
Scopo di questo trattato \'{e} d'indagare le leggi fondamentali di quelle operazioni della mente per mezzo delle quali si attua il ragionamento; di dar loro espressione nel linguaggio
simbolico di un calcolo e d'istituire, su questo fondamento, la scienza della logica costruendone il metodo; di fare, di questo stesso metodo, la base di un metodo generale per
l'applicazione della dottrina matematica della probabilit\'{a} e, in ultimo, di ricavare dai diversi elementi di verit\'{a} portati alla luce nel corso di queste indagini alcune
indicazioni probabili sulla natura e la costituzione della mente umana.

\section{}
Non c'\'{e} quasi bisogno di ricordare che questo progetto non \'{e} del tutto originale, e tutti sanno che i filosofi hanno dedicato una parte considerevole della loro attenzione
a quelle che sono, dal punto di vista pratico, le sue suddivisioni principali: la logica e la teoria della probabilit\'{a}. Nella forma che le fu data dagli antichi e dagli Scolastici,
la logica \'{e} quasi esclusivamente associata al grande nome di Aristotele: e fino ai giorni nostri, salvo alcuni cambiamenti inessenziali, \'{e} rimasta praticamente tal quale fu
presentata all'antica Grecia nelle disquisizioni in parte tecniche e in parte metafisiche dell'\textit{Organon}. Dal canto suo, l'indirizzo della ricerca originale si \'{e} orientato
principalmente verso questioni di filosofia generale le quali, pur essendo sorte fra le dispute dei logici, sono andate oltre quello che erano all'origine, dando alle epoche successive
della speculazione la loro inclinazione e il loro carattere particolari. Le et\'{a} di Porfirio e Proclo, di Anselmo e Abelardo, di Ramus e Descartes, conclusesi con la contestazione
di Bacone e Locke, stanno davanti alla nostra mente come esempi delle pi\'{u} remote influenze che questo studio ha esercitato sul cammino del pensiero umano: in parte perch\'{e} hanno
suggerito fecondi argomenti di discussione, in parte perch\'{e} hanno dato luogo alle critiche contro le sue pretese illegittime. Dall'altra parte, la storia della teoria della
probabilit\'{a} \'{e} stata contraddistinta in misura molto maggiore da quel costante sviluppo che costituisce la caratteristica propria della scienza. Il genio precoce di Pascal alle origini
di questa disciplina, le pi\'{u} profonde tra le speculazioni matematiche di Laplace nelle sue fasi pi\'{u} mature (e qui non faccio menzione di altri nomi, non meno noti di questi)
furono impegnati nel perfezionamento di questa teoria. Como lo studio della logica ha esercitato la propria influenza sul pensiero per le questioni di metafisica, ad esso affini,
cui ha dato occasione, cos\'{i} quello della teoria della probabilit\'{a} deve ritenersi importante per lo sviluppo che ha impresso alle parti pi\'{u} astratte della scienza matematica.
Si \'{e} inoltre ritenuto giustamente che ciascuna di queste discipline avesse di mira, oltre che fini pratici, anche fini teorici. L'oggetto della logica, infatti, non \'{e} solo
quello di metterci in grado di trarre inferenze corrette da premesse date, n\'{e} l'unica pretesa della teoria della probabilit\'{a} \'{e} quella di insegnarci come fondare su solide
basi il mestiere di assicuratore sulla vita o di raccogliere in formule i dati significativi delle innumerevoli osservazioni che si compiono in astronomia, in fisica, o in quel campo delle ricerche
sociali che oggi va rapidamente acquistando importanza. Entrambi questi studi presentano anche un interesse di altro genere, derivante dalla luce che gettano sui poteri dell'intelletto.
C'insegnano in qual modo il linguaggio e il numero servano da strumento e da ausilio ai processi del ragionamento; ci rivelano, in certa misura, la connessione esistente fra i diversi
poteri del nostro comune intelletto; mettono davanti a noi, nei due domin\^{i} della conoscenza dimostrativa e di quella probabile, i modelli essenziali della verit\'{a} e della
correttezza: modelli che non sono stati ricavati dall'esterno, ma sono profondamente radicati nella costituzione delle facolt\'{a} dell'uomo. Questi fini speculativi non cedono n\'{e}
in dignit\'{a}, n\'{e}, si potrebbe aggiungere, in importanza, agli scopi pratici con il perseguimento dei quali sono stati spesso associati nel corso della loro storia. Lo svelare le
leggi e le relazioni pi\'{u} nascoste di quelle facolt\'{a} superiori del pensiero grazie alle quali giungiamo a possedere, o portiamo a compimento, tutto ci\'{o} che va oltre la pura
e semplice conoscenza percettiva del mondo e di noi stessi, \'{e} un fine la cui dignit\'{a} non ha certo bisogno di essere raccomandata a uno spirito raziocinante.

\chapter{Dei segni in generale e dei segni adatti alla scienza della logica in particolare; delle leggi alle quali \'{e} sottoposta quest'ultima classe di segni}

\end{document}
