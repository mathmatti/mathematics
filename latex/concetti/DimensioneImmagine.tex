\documentclass[a4paper,10pt]{article}
\usepackage[utf8]{inputenc}
\usepackage{amsmath}
\usepackage{hyperref}

%opening
\title{Dimensione dell'immagine di un'applicazione lineare}
\author{\href{http://www.baudo.hol.es}{giuseppe baudo}}

\begin{document}

\maketitle

\section{DEFINIZIONE}
La dimensione dell'imagine di un'applicazione lineare è uguale al rango 
della matrice associata all'applicazione lineare.

\section{NOTAZIONE}

\section{ESEMPIO}

\section{APPROFONDIMENTI}
\begin{itemize}
 \item TITLE
\end{itemize}

\end{document}
