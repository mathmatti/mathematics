\documentclass[a4paper,10pt]{article}
\usepackage[utf8]{inputenc}
\usepackage{amsmath}
\usepackage{hyperref}

%opening
\title{Order of a group}
\author{baudo81[at]gmail.com}

\begin{document}

\maketitle

\section{DEFINIZIONE}
The order of a group is its cardinality, i.e., the number of elements in its set. Also, the order, sometimes period, of an element
$a$ of a group is the smallest positive integer $m$ such that $a^m=e$ (where $e$ denotes the identity element of the group, and $a^m$
denotes the product of $m$ copies of $a$). If no such $m$ exists, $a$ is said to have infinite order.

NB: La stessa definizione può essere data con la notazione additiva.

\section{NOTAZIONE}
The order of a group $G$ is denoted by $ord(G)$ or $|G|$ and the order of an element $a$ is denoted by $ord(a)$ or $|a|$. 

\section{APPROFONDIMENTI}
\begin{itemize}
 \item EN.WIKIPEDIA.ORG: Order (group theory) \cite{orderofgroup1}
 \item EN.WIKIPEDIA.ORG: Lagrange's theorem (group theory) \cite{orderofgroup2}
\end{itemize}


\bibliography{AlgebraIndex}
\bibliographystyle{plain}
\end{document}
