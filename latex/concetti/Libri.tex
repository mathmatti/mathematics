\documentclass[a4paper,10pt]{article}
\usepackage[utf8]{inputenc}
\usepackage{amsmath}
\usepackage{hyperref}

%opening
\title{Libri}
\author{\href{http://www.baudo.hol.es}{giuseppe baudo}}

\begin{document}

\maketitle

\section*{Nota}
Il materiale contenuto nella cartella di google drive è ad accesso limitato. Per accedere cliccare sul link seguente e seguire le istruzioni per ottenere
l'accesso alla cartella. \url{https://drive.google.com/drive/folders/0Bx2fZ0r5vhSSSDdvWkVjNG9YQjQ}

\section*{Introduzione}
I libri che trovate nella \href{https://drive.google.com/drive/folders/0Bx2fZ0r5vhSSSDdvWkVjNG9YQjQ}{folder}, sono stati raccolti seguendo la bibliografia proposta dai Docenti di Università
italiane e straniere. Si trovano i classici di algebra, analisi, geometria, etc. Inoltre, ho selezionato alcuni libri perchè hanno una data stampa risalenete al massimo
agli ultimi tre anni che in genere sono fatti bene perchè raccolgono le esperienze maturate studiando i testi che li hanno preceduti.

Oltre a libri, troverete dispense, papers, etc. Al momento non ho fatto distinzione tra libri o altro pdf ma ho semplicemente suddiviso il materiale seguendo
più o meno le materie indicate nel \href{Syllabus.html}{Syllabus} e quindi troverete le seguenti cartelle: Algebra, Analisi, Topologia, Geometria, etc.

\section*{Come utilizzare il materiale pdf}
Esistono buoni articoli che ci danno una panoramica su come utilizzare un libro di testo o una dispensa. Alcuni professori, specie nelle lezioni introduttive, danno
informazioni riguardanti lo studio della materia nel suo complesso e con esso anche un accenno sull'utilizzo dei libri di testo.

\end{document}
