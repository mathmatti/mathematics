\documentclass[a4paper,10pt]{article}
\usepackage[utf8]{inputenc}
\usepackage{amsmath}
\usepackage{hyperref}

%opening
\title{Algebra}
\author{}

\begin{document}

\maketitle

\section*{Syllabus}


\subsubsection*{Permutazioni, Coefficienti binomiali}
\begin{itemize}
 \item \href{Permutazione.pdf}{Permutazione}
 \item \href{InsiemePermutazioni.pdf}{Insieme di tutte le permutazioni su un insieme $X$}
 \item \href{PermutazioneCicli.pdf}{Ciclo di una permutazione, l-ciclo}
 \item \href{PermutazioniDisgiunte.pdf}{Cicli disgiunti}
 \item \href{PermutazioneDecomposizioneCicli.pdf}{Decomposizione in cicli disgiunti}
 \item \href{PermutazioneCicliPeriodo.pdf}{Ordine (o periodo) di un ciclo}
 \item \href{PermutazionePeriodo.pdf}{Ordine (o periodo) di una permutazione}
 \item TEOREMA: Ogni permutazione si può scrivere come una composizione di cicli disgiunti in modo unico a meno dell'ordine dei cicli stessi
 \item \href{Derangement.pdf}{Permutazione senza punti fissi, Derangement, partial derangement}
 \item \href{PermutazioneTrasposizione.pdf}{Trasposizione di una permutazione}
 \item \href{PermutazioneSegno.pdf}{Segno di una permutazione}
 \item \href{PermutazioneOrbita.pdf}{Orbita di una permutazione}
 \item \href{CoefficientiBinomiali.pdf}{Coefficienti binomiali: definizione e proprietà}
 \item \href{PrincipioInclusioneEsclusione.pdf}{Principio di inclusione-esclusione}
\end{itemize}

\subsection*{Teoria dei Gruppi}
  \subsubsection*{Gruppi}
  \begin{itemize}
    \item \href{./Gruppo.pdf}{Gruppo}
    \item Elemento neutro di un gruppo
    \item \href{FiniteGroup.pdf}{Finite Group}
    \item \href{./OrderOfGroup.pdf}{Order of a group}  
    \item Ordine di un elemento di un gruppo (vedi Order of a group)
    \item \href{Sottogruppo.pdf}{Sottogruppo}
    \item Gruppo abeliano    
    \item \href{./TeoremaDiLagrange.pdf}{Teorema di Lagrange e sue conseguenze} 
    \item \href{./CyclicGroup.pdf}{Gruppo ciclico}
        
    \item Gruppo generale lineare
    \item Gruppo simmetrico 
    \item Gruppo classi resto modulo n rispetto alla somma
    \item Gruppo di permutazioni
    \item Sottogruppo ciclico del gruppo simmetrico $S_n$ - Sottogruppo ciclico generato da una permutazione
    
    \item Omomorfismo
    \item Isomorfismo
    \item Omomorfismo/Isomorfismo tra ...
    \item \href{./OmomorfismoGruppi.pdf}{Omomorfismo di gruppi}
    \item Nucleo, nucleo di una funzione, nucleo di applicazione lineare
    \item \href{Nucleo.pdf}{Nucleo di omomorfismo di gruppi}
    \item Insieme degli omomorfismo di gruppi ( dati due gruppi $G$ e $H$ in simboli: $Hom(G,H)$)
  \end{itemize}



 \subsubsection*{Anelli e Campi}
 \begin{itemize}
   \item Anelli
 \item Anelli commutativi: 0-divisori, elementi nilpotenti, unità.
 \item Domini, campi. 
 \item Estensioni quadratiche. 
 \item Gli interi di Gauss. 
 \item Domini euclidei:  ideali principali, esistenza del massimo comun divisore, scomposizione in fattori irriducibili. 
 \item Morfismi di anelli; il morfismo da Z ad un anello; la caratteristica. 
 \item Ideali e anelli quozienti; ideale generato da un sottoinsieme. 
 \item La fattorizzazione di un morfismo di anelli. 
 \item Il campo dei quozienti di un dominio; Q, K(X). 
 \item Divisibilità in un anello. 
 \item Ideali primi e massimali.
 \item Il campo dei numeri complessi;
 \item rappresentazione geometrica dei numeri complessi; 
 \item teorema di De Moivre; 
 \item l'insieme delle radici n-esime di un numero complesso come laterale rispetto al sottogruppo delle radici n-esime dell'unità.
 \item L'anello dei polinomi in una indeterminata a coefficienti in un anello, 
 \item serie formali, 
 \item funzioni polinomiali, 
 \item grado di un polinomio e sue proprieta', 
 \item se A e' un dominio anche A[x] lo e'. 
 \item Polinomi a coefficienti in un campo: zeri e fattori lineari; 
 \item il lemma di divisione e le sue conseguenze (K[X] è un dominio euclideo, a fattorizzazione unica, ad ideali principali). 
 \item Il teorema fondamentale dell'algebra. 
 \item La derivata di un polinomio, molteplicità di una radice. 
 \item Polinomi reali.
 \item Quozienti di K[X]; forma ridotta. 
 \item Estensioni di campi; 
 \item elementi algebrici e trascendenti;
 \item polinomio minimo; 
 \item il sottocampo K(u) di un campo F generato dal sottocampo K di F e dall'elemento u. 
 \item Il grado di una estensione finita; 
 \item ogni elemento di una estensione finita è algebrico; 
 \item il grado della composizione di due estensioni finite; 
 \item l'insieme dei numeri algebrici è algebricamente chiuso. 
 \item Campo di spezzamento: esistenza e unicità. 
 \item Esistenza e unicità del campo con $p^n$ elementi; questi sono gli unici campi finiti.
 \end{itemize}




\subsection*{Teoria di Galois}
\begin{itemize}
 \item Teoria di Galois
 \item Richiami sui polinomi e le loro radici. 
 \item Campo di spezzamento (di un polinomio). 
 \item Funzioni simmetriche.
 \item Estensioni normali di campi. 
 \item Estensioni separabili di campi. 
 \item Teorema dell'elemento primitivo. 
 \item Estensioni di Galois. 
 \item Il gruppo di Galois di un'estensione. 
 \item La corrispondenza di Galois. 
 \item Il gruppo di Galois di un polinomio. 
 \item Il discriminante di un polinomio.
 \item Radici multiple, separabilità.
 \item Gruppo di Galois. 
 \item Corrispondenza di Galois. 
 \item Teorema 90 di Hilbert.
 \item Risolubilità per radicali. 
 \item Estensioni ciclotomiche.
 \item Costruzioni con riga e compasso.
 \item Il gruppo simmetrico come gruppo di Galois.
 \item Chiusura algebrica.
\end{itemize}

\subsection*{Teoria dei Moduli}
\begin{itemize}
	\item Anelli commutativi, ideali primi e massimali. 
	\item Moduli: definizioni di base. 
	\item Prodotto e somma diretta di moduli. 
	\item Moduli liberi. 
	\item Moduli noetheriani e artiniani. 
	\item Anelli noetheriani. 
	\item Teorema della base di Hilbert. 
	\item Moduli finitamente generati e loro presentazione. 
	\item Concetto di algebra. 
	\item Moduli su anelli euclidei e PID. 
	\item Decomposizione di un modulo su un anello euclideo e sue conseguenze. 
	\item Forme canoniche di un endomorfismo. 
	\item Prodotto tensoriale di moduli.
\end{itemize}
\end{document}
