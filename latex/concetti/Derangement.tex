\documentclass[a4paper,10pt]{article}
\usepackage[utf8]{inputenc}
\usepackage{amsmath}
\usepackage{hyperref}

%opening
\title{Derangement, Partial Derangement}
\author{baudo81[at]gmail.com}

\begin{document}

\maketitle

\section{DEFINITION}
A derangement is a permutation of the elements of a set, such that no element appears in its original 
position. \cite{derangement}

\section{NOTATION}
The number of derangement of a set of size $n$, usually written $D_{n}$, $d_{n}$, or $!n$, is called the "derangement number" or
"de Montmort number". (These numbers are generalized to rencontres numbers). \cite{derangement}

The number of derangements of an n-element set is called the nth derangement number or rencontres number, or the subfactorial
of n and is sometimes denoted $!n$ or $D_{n}$

\section{FORMULA DERANGEMENT}
\[
 d_{n} = n!\sum^{n}_{i=0} \frac{(-1)^i}{i!}
\]

\section{FORMULA PARTIAL DERANGEMENT}
La formula precendente è utilizzata quando vogliamo il numero delle permutazioni (o casi favorevoli, a volte negli esercizi) che hanno fixed point uguale a 0.
In generale per $k>0$ dove $k$ rappresenta il numero di fixed point, la formula diventa:

\[
 d_{n,k} = \frac{n!}{k!}\sum^{n}_{i=0} \frac{(-1)^i}{i!}
\]

\section{NOTE}
In altre parole, il derangment è un sottoinsieme dell'insieme delle permutazioni formato dalle permutazioni che non hanno punti fissi, cioè 
in cui nessun elemento è al suo posto.

\section{HISTORY}
The problem of counting derangements was first considered by Pierre Raymond de Montmort in 1708; he solved it in 1713, as did
Nicholas Bernoulli at about the same time. \cite{derangement}

\section{APPROFONDIMENTI}
\begin{itemize}
 \item \href{https://en.wikipedia.org/wiki/Derangement}{WIKIPEDIA: Derangement}
 \item \href{./pdf/DERANGEMENT/derangementLucchini.pdf}{DISPENSA: Derangement.pdf}
 \item \href{http://oeis.org/wiki/Number_of_derangements}{OEIS: Number of derangement}
\end{itemize}


\bibliographystyle{alpha}
\bibliography{AlgebraIndex}
\end{document}
