\documentclass[a4paper,10pt]{article}
\usepackage[utf8]{inputenc}
\usepackage{amsmath}
\usepackage{hyperref}

%opening
\title{Fondamenti di Matematica}
\author{\href{http://www.baudo.hol.es}{giuseppe baudo}}

\begin{document}

\maketitle

\section{Logica, Storia e Filosofia}
\begin{itemize}
 \item \href{HowToStudy.pdf}{How to study mathematics}
  \item \href{Metamathematics.pdf}{Metamathematics}
  \item \href{Logics.pdf}{Logica}
  \item \href{QuantumTheory.pdf}{Quantum Theory}
  \item \href{history.pdf}{History} 
\end{itemize}

\section{Set Theory}
\subsection*{Basics}
\begin{itemize}
 \item Insieme, famiglia, classe, collezione.
 \item Appartenenza
 \item Unione tra insiemi
 \item Intersezione tra insiemi 
 \item inf e sup di un insieme
 \item Significato di $A - B$ dove $A$ e $B$ sono insiemi.
\end{itemize}
 
\subsection*{Assiomi della teoria degli insiemi}
\begin{itemize}
 \item \href{./Extensionality.pdf}{Axiom of extensionality}
 \item \href{./EmptySet.pdf}{Axiom of empty set}
 \item Axiom of pairing
 \item Axiom of union
 \item Axiom of infinity
 \item Axiom of power set
 \item Axiom of regularity
 \item Axiom schema of specification
 \item Axiom of choice
\end{itemize}

\subsection*{Funzioni, applicazioni, mappe, operazioni, relazioni}
\begin{itemize}
 \item \href{CartesianProduct.pdf}{Cartesian product}
 \item \href{Relation.pdf}{Relation between sets}
 \item \href{EquivalenceRelation.pdf}{Equivalence relation}
 \item \href{OrdinamentoInsieme.pdf}{Ordinamento di un insieme (lemma di Zorn)}
 \item \href{Function.pdf}{Funzione}
 \item Applicazione
 \item Mappa
 \item \href{Operazione.pdf}{Operazione}
 \item Operazione di somma tra numeri
 \item Operazione di somma tra vettori
 \item Operazione di somma tra matrici
 \item \href{./FunzioneSuriettiva.pdf}{Funzione suriettiva (o surgettiva, o suriezione)}
 \item Funzione iniettiva
 \item Funzione bigettiva
 \item Cardinality (of a group for example)
\end{itemize}

\subsection*{Teoria delle Funzioni}
\begin{itemize}
 \item Dominio di una funzione
 \item Maggiorante di una funzione
 \item Minorante di una funzione
 \item Estremo superiore di una funzione
 \item \href{FunzioneMassimoMinimo.pdf}{Massimo e minimo di una funzione}
\end{itemize}

\section{Assiomi di Peano}
\begin{itemize}
 \item 
\end{itemize}

\section{Church's Postulates}
\begin{itemize}
 \item 
\end{itemize}

\section{Hilbert's Axioms}
\begin{itemize}
 \item 
\end{itemize}




\end{document}
