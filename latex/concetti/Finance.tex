\documentclass[a4paper,10pt]{article}
\usepackage[utf8]{inputenc}
\usepackage{amsmath}
\usepackage{hyperref}

%opening
\title{Finance}
\author{\href{http://www.baudo.hol.es}{giuseppe baudo}}

\begin{document}

\maketitle

\section{Introduzione}
Coming soon...

\section{Generale}
\begin{itemize}
 \item Contabilit\'{a}
 \item Bilanci
 \item Controllo di gestione
 \item Logistica
 \item Cash Flow
\end{itemize}

\section{A risk Perspective}
\begin{itemize}
	\item RISK AND RISK MANAGEMENT
	\item THE INVESTMENT INDUSTRY
	\item BASEL - Overview and Scope
	\item MIFID - A Risk Perspective
	\item UCITS - Key aspects
	\item MARKET RISK
	\item OPERATIONAL RISK
	\item CREDIT RISK
	\item ESMA (ex CESR) - Role and scope
	\item ESMA Risk Dashboard: monitoring the Risks
	\item EMIR - Scope and main obbligation
	\item CCPs - Scope and key aspects
	\item ISDA and Collateral management
\end{itemize}

\section{Applicazioni Big Data in finanza}
\begin{itemize}
	\item Introduzione, obiettivi e strumenti
	\item Introduzione ai problemi di classificazione
	\item Metodi di ricampionamento: cross-validation, Monte Carlo e bootstrap
	\item Metodi di regolarizzazione: Ridge, Lasso e tecniche collegate
	\item Modelli non lineari
	\item Metodi basati su alberi di classificazione e regressione
\end{itemize}

\section{Calcolo stocastico per la finanza}
\begin{itemize}
	\item Spazi di probabilit\'{a}, variabili aleatorie e distribuzioni
	\item Indipendenza, prodotto di misure e distribuzione congiunta
	\item Teorema di Radon-Nikodym, cambio di misura di probabilit\'{a} e attesa condizionata
	\item Processi stocastici, moto Browniano e martingale
	\item Il modello binomiale
	\item Integrale stocastico e calcolo di Ito multidimensionale. Equazioni differenziali stocastiche e risoluzione numerica
	\item Teorema di rappresentazione delle martingale e Teorema di Girsanov
	\item Modello di Black\&Scholes (B\&S): equazioni differenziali paraboliche, valutazione neutrale al rischio e copertura di derivati nel modello B\&S
	\item Analisi della volatilit\'{a}: volatilit\'{a} storica e implicita, effetto smile e struttura a termine della volatilit\'{a}. Cenni ad estensioni del modello di B\&S: modelli CEV, shifted lognormal, volatilit\'{a} locale, path-dependent e stocastica
	\item Cenni a metodi di approssimazione numerica: metodo Monte Carlo e metodo delle differenze finite
	\item Definizione e propriet\'{a} dei processi di L\'{e}vy
	\item Processi di L\'{e}vy esponenziali e valutazione di derivati
	\item Metodi di Fourier
\end{itemize}

\section{Counterparty Credit Risk}
The aim of the course is to introduce the basic concepts underlying Counterparty Credit Risk (CCR) in terms of:

\begin{itemize}
	\item Regulatory requirements
	\item Economic nature of the risk
	\item Risk mitigation
	\item Risk monitoring
	\item Pricing
\end{itemize}

The course will threat the main risk management aspects related to both: i) modeling credit exposure and ii) pricing counterparty risk.

\section{Equazioni alle derivate parziali e metodi di approssimazione numerica}
Il corso fornisce le competenze di base per trattare le equazioni differenziali di Black \& Scholes relative alle opzioni europee, alle opzioni americane e alle opzioni asiatiche. Verranno adeguatamente approfonditi gli aspetti dell'implementazione numerica.

Gli argomenti trattati sono i seguenti:

Teoria generale delle equazioni differenziali alle derivate parziali di diffusione e metodi di approssimazione numerica della soluzione. Applicazioni all'equazione di Black \& Scholes e ad alcuni modelli a volatilità stocastica;
Equazioni differenziali di tipo diffusione-trasporto: teoria generale e metodi numerici. Applicazioni alle opzioni asiatiche e ad un modello per le opzioni europee dove la volatilità dipendente dalla storia del titolo sottostante;
Problemi relativi alle equazioni differenziali con ostacolo. Applicazioni alle opzioni americane, sia per il classico modello di Black \& Scholes, sia per le opzioni che dipendano dalla storia del titolo sottostante.
Il materiale didattico verrà distribuito nel corso delle lezioni.

\section{Excel e VBA per la finanza}
Il corso offre una panoramica di Microsoft Excel ® utilizzato come strumento di analisi finanziaria.

In particolare, dopo una prima parte dedicata alle funzioni di base del software, verranno affrontati i problemi applicativi inerenti all'ambito matematico-finanziario, utilizzando anche le potenzialità offerte dal linguaggio Visual Basic integrato in Excel (VBA).

Gli argomenti del corso saranno:

Presentazione di Excel ®
Funzioni di base del foglio elettronico
Creazione di grafici e tabelle
Tabelle pivot e reportistica
Implementazione di procedure automatiche
Applicazioni alla matematica finanziaria: in particolare simulazioni Montecarlo per il pricing di prodotti derivati.


\section{Finanza Computazionale}
Metodi numerici per la valutazione di derivati finanziari:

Introduzione alla programmazione con Matlab. Grafica in Matlab. Algebra lineare in Matlab con cenni sulla vettorializzazione del codice.
Cenni di calcolo numerico (errori in aritmetica finita).
Risoluzione numerica di equazioni differenziali alle derivate parziali. PDE paraboliche: differenze finite, metodi implicito, esplicito, Crank-Nicholson e theta-method.
Alberi Binomiali: valutazione di opzioni europee di opzioni europee ed americane. Riduzione degli errori tramite tecnica "control-variate".
Metodo Monte Carlo: generazione di scenari, approssimazione diequazioni differenziali stocastiche, metodi di riduzione della varianza.
Metodi basati sulla trasformata di Fourier discreta. Trasformata di Fourier e sue proprietà. Metodi per la valutazioni di opzioni europee.
Calibrazione di modelli a dati di mercato.
- Tutti gli esercizi verranno svolti in Matlab(R).


Testi di riferimento:

- Monte Carlo Methods in Financial Engineering, Paul Glasserman, Springer, 2004

- Numerical Methods in Finance and Economics (seconda edizione), Paolo Brandimarte, Wiley, 2006
- Numerical Methods and Optimization in Finance, Manfred Gilli, Dietmar Maringer and Enrico Schumann, Academic Press, 2011

\section{Fixed Income Trading}
Il corso ha l'obiettivo di approfondire le peculiarità del mercato dei tassi di interesse, con approfondimenti sul mercato obbligazionario e sugli strumenti derivati. Lo scopo è quello di spiegare le relazioni esistenti tra mercati segmentati e a volte inefficienti e di costruire modelli di pricing che tengano conto di queste caratteristiche. Verranno analizzati casi concreti per documentare come i modelli matematici teorici vengano applicati nella realtà del trading, con quali limiti e con quali opportunità.

Gli argomenti trattati sono i seguenti:

Tassi risk free
Bootstrapping
Tassi spot e tassi forward
Pricing di obbligazioni
Building bloks
Determinazione del "Fair Price"
Analisi di sensitività
Relative value (Asset swap vs Credit Default Swap)
Trading
Inefficienze e opportunità di mercato
Arbitraggi e algorithmic trading

\section{Introduzione all'ambiente e al linguaggio Matlab}
Introduzione all'ambiente Matlab.
Linguaggio di programmazione Matlab: uso di array, costrutti di programmazione, funzioni.
Funzioni predefinite Matlab. Funzioni matematiche, manipolazione di array, funzioni grafiche 2D e 3D.
Esempi ed esercizi guidati in laboratorio.


\section{Matematica finanziaria e strumenti di mercato}
Il corso si propone di fornire le nozioni elementari per la comprensione delle leggi che regolano i mercati finanziari. Verranno introdotti i principali strumenti del mercato dei capitali, della liquidità e del credito.

Il corso si concluderà con una breve introduzione ai principali modelli di asset allocation.

Gli argomenti trattati sono i seguenti:

Introduzione alla matematica finanziaria: Regimi di capitalizzazione, valore attuale e sconto, tassi equivalenti e basi temporali. Rendite e costituzione di un capitale. Ammortamento di un prestito, tassi reali, tassi nominali, tasso di inflazione.
Introduzione al mercato dei capitali: Libor, Euribor, Eonia, Depositi, Repo. Derivati di tasso (future, opzioni e IRS). Bond and loans (tasso fisso, zero coupon, tasso variabile, inflation linked).
Introduzione agli strumenti azionari e derivati: Azioni ed asset pricing, opzioni europee, relazioni di parità, strategie. Opzioni esotiche. Derivati sul credito: CDS e Asset Swaps.
Introduzione all'asset allocation: definizione di indici e benchmark. Modello di Markowitz, Sharpe. Modello di allocazione alla Black-Litterman.
- Il materiale didattico verrà distribuito nel corso delle lezioni.


\section{Metodi di calibrazione}
Gli argomenti trattati sono i seguenti:

Identificazione di parametri in equazioni differenziali paraboliche. Problema dei minimi quadrati non lineari.
Risoluzione numerica di un problema di minimi quadrati non lineari: condizioni di ottimalità. Metodi di discesa: ordine di convergenza.
Metodi del gradiente e metodi quasi Newton.
Funzione Matlab LSQNONLIN.
Esempi di applicazione in ambito finanziario in Matlab.


\section{Metodi econometrici in finanza}
Introduzione, obiettivi e strumenti
Rendimenti finanziari: definizioni e proprietà
Strumenti statistici per l’analisi descrittiva e grafica dei dati
Distribuzioni di probabilità univariate: definizioni e proprietà
Distribuzioni di probabilità multivariate: definizioni e proprietà
Modello di regressione lineare e minimi quadrati ordinari
Capital Asset Pricing Model
Modello di regressione non lineare e minimi quadrati non lineari
Estensioni del modello di regressione
Modelli lineari per serie storiche
Modelli GARCH
Cointegrazione.

\section{Misurazione del rischio finanziario}
Il corso si propone di introdurre le principali idee e tecniche che stanno alla base dell'attività di misurazione del rischio finanziario, con particolare riferimento al rischio di mercato e di liquidità.

Gli argomenti trattati sono:

Tipologie di rischi finanziari.
Fattori di rischio e variabile Profit\&Loss (PL).
Approccio "sensitivities", approssimazioni Delta e Delta-Gamma Quantili, Value-at-Risk e Expected Shortfall.
Metodo storico, semplice e pesato.
Metodo analitico: PL lineari con 1 e con più fattori di rischio.
PL non-lineari, metodo analitico e metodo Monte Carlo.
Back-testing di VaR e ES Approccio degli "stress-test".
Normativa di Basilea III: principali aspetti operativi.
Microstruttura dei mercati e misurazione del rischio di liquidità.

\section{Misurazione del rischio in Solvency 2}
Il corso ha per oggetto la presentazione della nuova normativa di vigilanza prudenziale Solcency II prevista per le compagnie assicurative a partire dal 1 gennaio 2016 con particolare focus sulla modalità di misurazione del rischio.

In particolare verranno descritti gli elementi di base necessari per la determinazione del market consistent balancesheet Solvency II e degli own funds e verrà descritta la modalità di misurazione del rischio e del Solcency Capital Requirement (SCR) previsti dalla normativa in base alla formula standard.

Inoltre, verranno presentati i requisiti previsti dalla normativa per le imprese che decidono di dotarsi di un modello interno per la misurazione dei rischi e verrà riportata una breve descrizione del modello interno sviluppato presso il Gruppo Unipol.


\section{Misure di rischio finanziario: sviluppi recenti}
Durante il corso, che si pone come naturale complemento a quello di "Misurazione del rischio finanziario", verranno presentati alcuni recenti sviluppi in tema di misure di rischio che stanno avendo un significativo impatto operativo.

Gli argomenti trattati saranno:

Coerenza di una misura di rischio
Forecasting, backtesting ed "elicitability" di una misura di rischio
Rischio di stima e di modello.


\section{Modern Interest Rates}
- Interest rate basics

Dimensions and units in finance and other disciplines
Interest rates definition and conventions
Multiple types of interest rates
- Interest rate market

Deposits
The money market: central banks, interbank, retail
Libor/Euribor/Eonia/Repo interest rates
How the market changed: stylized facts and overview of market data
Symmetry breaking and market segmentation after the credit crunch
Credit and liquidity components
Counterparty risk and collateral
From Libor to OIS discounting
- Modern interest rate modelling

Notation and basic assumptions
Short rate, bank account, Zero Coupon Bond, probability measure
Feynman-Kac and Girsanov theorems
Replication
Black-Scholes-Merton, modern perspective
Multiple funding sources
Funding and funding value adjustment (FVA)
Collateral: perfect, partial, hedge collateral
Stochastic funding rates, multiple currencies
- Pricing of linear interest rate derivatives

A simple credit model to explain multiple interest rates
Spot, forward and instantaneous forward rates
Forward Rate Agreement
Futures
Overnight Indexed Swap
Swap, forward swap measure
Basis Swap
Bond
- Multiple-curve framework

Modern multiple curve pricing \& hedging market practice
Multiple curves construction
Selection of bootstrapping instruments, market data
Bootstrapping formulas
Interpolation
Handling negative rates
Exogenous bootstrapping
Turn of year effect
Multiple curves, multiple deltas, multiple hedging
Performance, Sanity checks
Lab session: yield curve bootstrapping implementation
- Forward rate modelling: single rates

Black's model
Beyond the Black's model
Stochastic volatility SABR model
Handling negative rates, shifted Black, shifted SABR
- Pricing of interest rate volatility product

Cap/Floor
Swaption, cash vs physical settlement
Market quotations
- Multiple volatility cubes

Modern multiple curve, multiple volatility market practice
Main issues
Swaptions volatility cube
Caps/Floors volatility cube
Handling multiple rate tenors, Kienitz model
Lab session: SABR implementation
- Convexity adjustment and Constant Maturity Swaps

IRS convexity adjustment
Constant Maturity Swaps
CMS convexity adjustment
SABR calibration to Swaptions and CMSs
CMS Options
CMS Spread Options
Bootstrapping implied correlations
- Short rate modelling and forward rate modelling

Instantaneous forward rates: the HJM model
Short rate models, Vasicek and Hull-White models
Libor Market Model (LMM)
Dealing with multiple curves and negative rates
- Conclusions and references



\section{Rischio di credito}
Nel corso ci si propone di fornire modelli e nozioni matematico-probabilistiche di base per studiare i problemi legati al rischio di credito, in particolare, il prezzaggio di prodotti sensibili al rischio di credito (defaultable zero-coupon bonds,credit default swaps, ecc.). Particolare attenzione sarà portata ai modelli a struttura affine.

Si studierà dapprima il caso di una sola controparte fallimentare per poi estendere lo studio anche al rischio di credito di portafoglio.

- Testo di riferimento: A.J. McNeil, R.Frey, P.Embrechts, "Quantitative Risk Management", Princeton Series in Finance, Princeton University Press 2005, edizione rivista 2015. Capitolo 10.



Altri riferimenti:

- T.R. Bielecki, M.Rutkowski, " Credit Risk: Modeling, Valuation and Hedging", Springer Finance 2004.
- D. Filipovic, "Term Structure Models", Springer Verlag 2009. Capitolo 12.
- D. Brigo, F. Mercurio, "Interest Rate Models - Theory and Practice", Springer Verlag. Capitoli 21,22,23 della seconda edizione 2006.


\section{Struttura a termine dei tassi}
Durante il corso verrà fornita un'introduzione alla modellistica della struttura a termine dei tassi (modelli per il tasso spot; impostazione secondo Heath-Jarrow-Morton per i tassi istantanei a termine; modelli di mercato), così come delle tecniche di base utilizzate nello studio dei problemi legati ai tassi di interesse (tecniche legate alla struttura a termine affine; tecniche di cambio di numerario).

Accenni verranno fatti anche ai problemi di calibrazione dei modelli ai dati di mercato e verrà studiato il prezzaggio dei principali derivati dei tassi (FRAs, Caps e Floors, Swaptions). Per facilitare la comprensione, ci si concentrerà principalmente sulla situazione pre-crisi di modelli a curva singola con accenno anche ai modelli muti-curva che sono stati introdotti dopo la grande crisi.

Testi di riferimento:

- T. Bjoerk, "Arbitrage Theory in Continuous Time", Oxford University Press; 3a edizione 2009.
- D. Brigo e F. Mercurio, "Interest Rate Models, Theory and Practice", Springer-Verlag, 2da edizione 2006.
- Z. Grbac e W.J. Runggaldier, "Interest rate modeling: post-crisis challenges and approaches", Springer Briefs in Quantitative Finance, 2015.




\section{RESOURCES}
\begin{itemize}
 \item \url{https://ocw.mit.edu/courses/mathematics/18-s096-topics-in-mathematics-with-applications-in-finance-fall-2013/video-lectures/lecture-1-introduction-financial-terms-and-concepts/}, sono arrivato al minuto 26
 \item \url{http://www.dm.unibo.it/finanza/}
\end{itemize}

\end{document}
