\documentclass[a4paper,10pt]{article}
\usepackage[utf8]{inputenc}
\usepackage{amsmath}
\usepackage{hyperref}

%opening
\title{Esercizio 2 - Dimostrare che una funzione è omomorfismo di gruppi}
\author{}

\begin{document}

\maketitle

\begin{abstract}



Sia  
\[
  G = \left\{ \left[ {\begin{array}{cc}
   a & b \\
   0 & a \\    
   \end{array} } 
   \right] ; a, b \in R, a \ne 0 \right\} 
\]

\begin{itemize}
 \item Dimostrare che la funzione $f:G\longrightarrow R^{*}$ definita da \[
  f \left( \left[ {\begin{array}{cc}
   a & b \\
   0 & a \\    
   \end{array} } 
   \right] \right) = a
\]
 
è un omomorfismo del gruppo $G$ nel gruppo moltiplicativo $R^{*}$.



\end{itemize}

\end{abstract}

\section{TEORIA}

\begin{enumerate}
 \item Definizione di \href{./OmomorfismoGruppi.html}{omomorfismo di gruppi} in quanto per dimostrare l'esercizio uso la definizione. fine.
 \item Cosa si intende per gruppo moltiplicativo $R^{*}$? Scritto in altro modo è più chiaro $(R,*)$.

\end{enumerate}





\section{SOLUZIONE}


\end{document}
