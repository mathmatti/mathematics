\documentclass[a4paper,10pt]{article}
\usepackage[utf8]{inputenc}
\usepackage{amsmath}
\usepackage{hyperref}

%opening
\title{Introduzione alle applicazioni lineari}
\author{\href{http://www.baudo.hol.es}{giuseppe baudo}}

\begin{document}

\maketitle

\section{Introduzione tratta dal libro "Introduzione all'algebra lineare" Prof. Fioresi-Morigi}
Le applicazioni lineari sono funzioni tra spazi vettoriali che ne rispettano la struttura, cioè sono compatibili con le operazioni di somma tra vettori e moltiplicazione di un
vettore per uno scalare. Come vedremo le applicazioni lineari si rappresentano in modo molto efficace attraverso le matrici. Lo scopo di questo capitolo è quello di introdurre
il concetto di applicazione lineare e capire come sia possibile associare univocamente una matrice ad ogni applicazione lineare tra $R^n$ e $R^m$, una volta fissata in entrambi
gli spazi la base canonica. Studieremo poi il nucleo e l'immagine di un'applicazione lineare fino ad arrivare al Teorema della dimensione, che rappresenta uno dei risultati più
importanti della teoria sugli spazi vettoriali di dimensione finita.


\end{document}
