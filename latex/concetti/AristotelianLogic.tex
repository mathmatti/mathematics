\documentclass[a4paper,10pt]{article}
\usepackage[utf8]{inputenc}
\usepackage{amsmath}
\usepackage{hyperref}

\bibliographystyle{alpha}
%opening
\title{Aristotelian logic}
\author{\href{http://www.baudo.hol.es}{giuseppe baudo}}

\begin{document}

\maketitle

\section{DEFINIZIONE}
Aristotelian or traditional logic is a “subject-predicate” logic and is therefore concerned only with a portion of the sum total of logical truth.

It confines itself to the four forms of categorical proposition 
known as the A, E, I, and O forms.

In the second place, it treats subalternation as a valid form of inference. 
That is, it assigns (tacitly at least) conventional meanings, different from those employed in modern 
(i.e. symbolic or mathematical) logic, to the four categorical forms, meanings such that subalternation holds.

These two characteristics, however, are not sufficient to define Aristotelian logic. 
For an indefinite number of systems could be devised which possess these two properties. 
Since further distinguishing traits of Aristotelian logic will appear only in the course of our investigation, 
a definition of Aristotelian logic framed in terms of its essential properties cannot be given at the outset.
\bibliography{bibliography}

\end{document}
