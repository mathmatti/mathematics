\documentclass[a4paper,10pt]{article}
\usepackage[utf8]{inputenc}
\usepackage{amsmath}
\usepackage{hyperref}

%opening
\title{Spazio Topologico}
\author{\href{http://www.baudo.hol.es}{giuseppe baudo}}

\begin{document}

\maketitle

\section{Definizione (1)}
Si chiama \textit{spazio topologico} una struttura costituita da un insieme non vuoto $S$, per ogni elemento (o punto) del quale sia assegnata una famiglia non vuota
$F(x)$ di sottoinsiemi di $S$, in modo tale che siano soddisfatti i seguenti postulati:

\section{Symbol}
\[
 (S, \bar{U})
\]

\section{Esempio}

\section{APPROFONDIMENTI}
\begin{itemize}
 \item TITLE
\end{itemize}

\end{document}
