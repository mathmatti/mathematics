\documentclass[a4paper,10pt]{article}
\usepackage[utf8]{inputenc}
\usepackage{amsmath}
\usepackage{hyperref}

%opening
\title{Matrice Invertibile}
\author{}

\begin{document}

\maketitle

\begin{abstract}

\end{abstract}

\section{Definizione}
Una matrice quadrata $A$ $n \times n$ si dice invertibile se esiste una matrice quadrata $B$ tale che $A \cdot B = B \cdot A = \href{./MatriceIdentica.html}{I_{n}}$

\section{Note}
La matrice $B$ se esiste è unica, cioè la matrice $A$ non può avere due inverse diverse. Ovviamente, qui in questo contesto la matrice inversa è $B$

L'unica inversa si scrive come $A^{-1}$.

Una matrice non invertibile si dice singolare.

Attenzione le matrici non quadrate non hanno l'inversa !!! da verificare

\section{Data una matrice, trovare se esiste la sua inversa}

\end{document}
