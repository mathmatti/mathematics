\documentclass[a4paper,10pt]{article}
\usepackage[utf8]{inputenc}
\usepackage{amsmath}
\usepackage{hyperref}

%opening
\title{Algebra Lineare}

\begin{document}

\maketitle
\href{AlgebraLineare.pdf}{pdf} 
\section*{Syllabus}
\subsection{Spazi Vettoriali}
\begin{itemize}
 \item Vettore
 \item \href{CoordinateVettore.pdf}{Coordinate di un vettore}
 \item \href{SpazioVettoriale.pdf}{Spazio Vettoriale}
 \item Sottospazio vettoriale
 \item Combinazione lineare
 \item Spazio generato dai vettori $v_1, ..., v_k$
 \item Sistema di generatori
 \item Versore
 \item Interdipendenza lineare (Indipendenza lineare)
 \item \href{Base.pdf}{Base di uno spazio vettoriale}
 \item Base canonica
 \item Matrice di cambiamento di base
 \item Criterio di indipendenza
 \item Estrazione di una base
 \item Completamento a una base
 \item \href{Dimensione.pdf}{Dimensione di uno spazio vettoriale}
 \item Componenti
 \item Somma diretta e somma 
 \item Spazi quozienti
 \item Duale
\end{itemize}

\subsection{Matrici}
\begin{itemize}
   \item \href{Matrice.pdf}{Matrice}
   \item Matrici identit\'{a} (identica)
   \item Matrice nulla
   \item Matrice opposta 
   \item \href{InsiemeDelleMatrici.pdf}{Insieme delle matrici $m \times n$: $M_{m,n}(K)$}
   \item Matrice trasposta
   \item Properties of Transpose Matrices: \url{http://www.math.nyu.edu/~neylon/linalgfall04/project1/dj/proptranspose.htm}
   \item Help with proving that the transpose of the product of any number of matrices is equal to the product of their transposes in reverse.
   \item La trasposta della trasposta \'{e} la matrice stessa
   \item La trasposta della somma di due matrici è uguale alla somma delle due matrici trasposte
   \item L'ordine delle matrici si inverte per la moltiplicazione
   \item Se $c$ \'{e} uno scalare, la trasposta di uno scalare \'{e} lo scalare invariato
   \item Nel caso di matrici quadrate, il determinante della trasposta \'{e} uguale al determinante della matrice iniziale
   \item La trasposta di una matrice invertibile \'{e} ancora invertibile e la sua inversa \'{e} la trasposta dell'inversa della matrice iniziale
   \item Se $A$ \'{e} una matrice quadrata, allora i suoi autovalori sono uguali agli autovalori della sua trasposta
   \item \href{SommaMatrici.pdf}{Operazione di somma tra matrici}
   \item \href{ProdottoMatrici.pdf}{Operazione di prodotto tra matrici}
   \item Properties of matrix multiplication: \url{https://www.khanacademy.org/math/precalculus/precalc-matrices/properties-of-matrix-multiplication/a/properties-of-matrix-multiplication}
   \item Operazione di prodotto tra uno scalare ed una matrice
   \item \href{MatriceQuadrata.pdf}{Matrice Quadrata}
   \item \href{OrdineMatrice.pdf}{Ordine di una matrice quadrata}
   \item \href{DeterminanteMatrice.pdf}{Determinante di una matrice quadrata}
   \item \href{PolinomioMatrice.pdf}{Polinomio caratteristico di una matrice}
   \item \href{AutovaloriMatrice.pdf}{Autovalori di una matrice quadrata}
   \item \href{Autovalore.pdf}{Autovalore}
   \item Autovettore
   \item Molteplicità algebrica e geometrica
   \item Matrice diagonale
   \item \href{MatriceDiagonalizzabile.pdf}{Matrice diagonalizzabile}
\end{itemize}

\subsection{Sistemi lineari}
  \begin{itemize}
   \item Sistema di che cosa?
   \item Sistema lineare e matrici.
   \item Sistema lineare
   \item Sistema omogeneo
   \item Risoluzione
  \end{itemize}
  
\subsection{Applicazioni lineari}
  \begin{itemize}
   \item \href{IntroAppLineari.pdf}{Introduzione alle applicazioni lineari}
   \item \href{ApplicazioneLineare.pdf}{Applicazione lineare}
   \item Nucleo di un'applicazione lineare
   \item Immagine di un'applicazione lineare
   \item Base di un'applicazione linearea
   \item Cambio di base per un'applicazione lineare
   \item \href{ApplicazioneIniettiva.pdf}{Applicazione lineare iniettiva}
   \item \href{DimensioneImmagine.pdf}{Dimensione dell'immagine di un'applicazione lineare}
   \item Teorema della dimensione
   \item \href{MatriceApplicazione.pdf}{Matrice associata ad un'applicazione lineare tra spazi vettoriali}
   \item Isomorfismo di applicazioni lineari
   \item Calcolo del nucleo e dell'immagine
   \item Diagonalizzazione
   \item Applicazione lineare diagonalizzabile
  \end{itemize}

\end{document}
