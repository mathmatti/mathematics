\documentclass[a4paper,10pt]{article}
\usepackage[utf8]{inputenc}
\usepackage{amsmath}
\usepackage{hyperref}

%opening
\title{Topologia}
\author{\href{http://www.baudo.hol.es}{giuseppe baudo}}

\begin{document}

\maketitle

\section{Introduzione alla Topologia Generale}
La topologia generale è il linguaggio con il quale è scritta una parte consistente della Matematica. Non a caso il nome originario \textit{topologia analitica} è stato
rimpiazzato da \textit{topologia generale}, anche a significare che si tratta della topologia che è usata dalla grande maggioranza dei matematici e che è necessaria a molti
settori della matematica. Quali?

In pratica, è quella situazione in cui si prendono gli assiomi della teoria degli insiemi e su di essi si costruisce tutta la matematica. I semplici concetti di insieme, punto, etc.
combinati insieme con le operazioni tra elementi e insiemi (unione intersezione, etc.) formano delle strutture matematiche che sono state immaginate come modello di rappresentazione
di una realtà oggetto di studio.

E' innegabile che la topologia generale ha un notevole valore formativo, in quanto costringe ed abitua la mente a lavorare con oggetti estremamente astratti, definiti 
esclusivamente per conto di assiomi. 

E' un dato di fatto che il modo migliore di studiare è quello di frequentare (partecipare attivamente) le lezioni, o in alternativa studiare i libri, cercando di capire bene
le definizioni, i teoremi ed i collegamenti esistenti tra loro, e contemporaneamente svolgere gli esercizi, senza paura di sbagliare, confrontando successivamente le proprie
soluzioni con quelle proposte dal libro, dal docente, dai compagni di corso, dai siti internet eccetera. 

Utilizzeremo la teoria ingenua degli insiemi, cioè eviteremo, con una sola eccezione, di dare impostazioni assiomatiche e lasceremo che ognuno usi le nozioni di insieme
e di appartenenza che gli sono suggerite dal senso comune e che gli hanno permesso di superare gli esami di algebra, analisi e geometria: l'eccezione riguarderà l'assioma
della scelta, il cui contenuto è meno evidente dal punto di vista del pensiero comune e della logica elementare.

Adotteremo volutamente la strategia dello struzzo per non vedere i paradossi a cui questo approccio può portare. Una sana regola che costa poco e che aiuta ad evitare
i più classici paraddosi è questa: non dire \textit{l'insieme degli insiemi tali che ...} ma preferire \textit{la famiglia degli insiemi ...}; questo eviterà inoltre
monotone ripetizioni. La stessa regola si applica alle famiglie e quindi diremo: \textit{la classe delle famiglie ...}, \textit{la collezione delle classi ...} e così via.

Gli spazi topologici sono le strutture più generali nelle quali ha senso introdurre e studiare il concetto di funzione continua. Per questo motivo, sovente, i corsi di analisi
matematica contengono una introduzione alla topologia, in genere della retta reale.

\end{document}
