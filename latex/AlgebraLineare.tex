\documentclass[a4paper,10pt]{article}
\usepackage[utf8]{inputenc}
\usepackage{amsmath}
\usepackage{hyperref}

%opening
\title{Algebra Lineare}

\begin{document}

\maketitle

\section{Spazi Vettoriali}
\begin{itemize}
 \item \href{SpazioVettoriale.html}{Spazio Vettoriale}
 \item Sottospazio vettoriale
 \item Combinazione lineare
 \item Spazio generato dai vettori v_1, ..., v_k
 \item Sistema di generatori
 \item Interdipendenza lineare (Indipendenza lineare)
 \item Applicazione lineare
 \item \href{Base.html}{Base di uno spazio vettoriale}
 \item Base canonica
 \item Criterio di indipendenza
 \item Estrazione di una base
 \item Completamento a una base
 \item \href{Dimensione.html}{Dimensione di uno spazio vettoriale}
 \item \href{DimensioneImmagine.html}{Dimensione dell'immagine di un'applicazione lineare}
 \item Componenti
 \item Somma diretta e somma 
 \item \href{ApplicazioneIniettiva.html}{Applicazione lineare iniettiva} 
 \item Spazi quozienti
 \item Duale
 \item \href{MatriceApplicazione.html}{Matrice associata ad un'applicazione lineare tra spazi vettoriali}
\end{itemize}

\section{Matrici}
\begin{itemize}
   \item \href{MatriceQuadrata.html}{Matrice Quadrata}
   \item \href{OrdineMatrice.html}{Ordine di una matrice quadrata}
   \item \href{DeterminanteMatrice.html}{Determinante di una matrice quadrata}
   \item \href{PolinomioMatrice.html}{Polinomio caratteristico di una matrice}
   \item \href{AutovaloriMatrice.html}{Autovalori di una matrice quadrata}
   \item \href{MatriceDiagonalizzabile.html}{Matrice diagonalizzabile}
\end{itemize}

\end{document}
