\documentclass[a4paper,10pt]{article}
\usepackage[utf8]{inputenc}
\usepackage{amsmath}
\usepackage{hyperref}

%opening
\title{Transfinito}
\author{\href{http://www.baudo.hol.es}{giuseppe baudo}}

\begin{document}

\maketitle

\section{DEFINIZIONE}
Che va al di l\'{a} del finito.

\section{NOTE}
Transfinito dunque è sinonimo del termine infinito che si usa oggi in matematica, anche se secondo me il termine transfinito è qualcosa di più ampio
nel senso che l'infinito è un transfinito perch\'{e} va oltre il finito ma transfinito potrebbe anche essere qualcos'altro.

Per quanto riguarda la logica matematica il problema \'{e} capire qual'\'{e} la dimostrazione della sua esistenza e della sua coerenza.

\section{APPROFONDIMENTI}
\begin{itemize}
 \item \url{http://www.treccani.it/enciclopedia/transfinito/}
\end{itemize}

\end{document}
