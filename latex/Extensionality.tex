\documentclass[a4paper,10pt]{article}
\usepackage[utf8]{inputenc}
\usepackage{amsmath}
\usepackage{hyperref}

%opening
\title{Axiom of Extensionality}
\author{\href{http://www.baudo.hol.es}{giuseppe baudo}}

\begin{document}

\maketitle
\href{Extensionality.pdf}{pdf}

\section{Introduzione}
Dovrebbe essere abbastanza evidente che, l'uguaglianza tra due strutture matematiche, \'{e} un fatto "quasi" arbitrario e quindi in un certo senso devo definire che cosa si intende
per uguaglianza tra due insiemi. Ebbene, Zermelo identifica tale concetto di uguaglianza con il concetto di estenzionalit\'{a}, cio\'{e} di estensione. Dietro questo assiome
c\'{e} dell'altro. In un certo senso, si potrebbe dire che \'{e} il meccanismo del contare all'interno della teoria degli insiemi. Quindi dall'assioma non ricavo "quanti" sono
gli elementi dei due insiemi ma sono certo del fatto che i due insiemi contengono lo "stesso numero" di elementi.

\section{NOTAZIONE}
\[
  \forall A \forall B (\forall X (X \in A \Leftrightarrow X \in B ) \Rightarrow A = B)
\]
\end{document}
