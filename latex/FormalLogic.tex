\documentclass[a4paper,10pt]{article}
\usepackage[utf8]{inputenc}
\usepackage{amsmath}
\usepackage{hyperref}

%opening
\title{Formal Logic}
\author{\href{http://www.baudo.hol.es}{giuseppe baudo}}

\begin{document}

\maketitle

\section{DEFINIZIONE}
Formal logic e symbolic logic sono la stessa cosa. Aristotele per esempio, non usava simboli per studiare la logica formale.

\section{NOTAZIONE}

\section{ESEMPIO}

\begin{thebibliography}{100}
  \bibitem{k1} \url{https://www.britannica.com/topic/formal-logic}
  \bibitem{k2} \url{https://math.stackexchange.com/questions/1812585/what-is-the-difference-between-symbolic-and-formal-logic}
\end{thebibliography}

\end{document}
