\documentclass[a4paper,10pt]{article}
\usepackage[utf8]{inputenc}
\usepackage{amsmath}
\usepackage{hyperref}

%opening
\title{Sottogruppo}
\author{\href{http://www.baudo.hol.es}{giuseppe baudo}}

\begin{document}

\maketitle

\section{Definizione}
A nonempty subset, $H$, of a group $G$ is called a $subgroup$ of $G$ if, relative to the product in $G$, $H$ itself forms a group.

\section{Note}
\subsection{relative to the product in $G$}
We stress the phrase "relative to the product in $G$". Take, for instance, the subset $A = \{1, -1\}$ in $Z$, the set of integers. Under the multiplication of integers,
$A$ is a group. But $A$ is not a subgroup of $Z$ viewed as a group with respect to $+$.

\subsection{From intro to paragraph about Subgroup From Abstract Algebra by Herstein}
In order for us to find out more about the makeup of a given group $G$, it may be too much of a task to tackle all of $G$ head-on. It might be desiderable to focus our
attention on appropriate pieces of $G$, which are smaller, over which we have some control, and are such that the information gathered about them can be used ot get 
relevant information and insight about $G$ itself. The question then becomes: What should serve as suitable pieces for this kind of dissection of $G$? Clearly, whatever
we choose as such pieces, we want them to reflect the fact that $G$ is a group, not merely any old set.

A group is distinguished from an ordinary set by the fact that it is endowed with a well-behaved operation. It is thus natural to demand that such pieces above behaved
reasonably with respect to the operation of $G$. Once this is granted, we are led almost immediately to the concept of a subgroup of a group.



\end{document}
