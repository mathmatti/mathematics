\documentclass[a4paper,10pt]{article}
\usepackage[utf8]{inputenc}
\usepackage{amsmath}
\usepackage{hyperref}

%opening
\title{Interpretation}
\author{\href{http://www.baudo.hol.es}{giuseppe baudo}}

\begin{document}

\maketitle

\section{DEFINIZIONE (1)}
An interpretation is an assignment of meaning to the symbols of a formal language.  

assignment = ???  
symbol = ???  
formal language = ???

\section{DEFINIZIONE (2)}
An interpretation $I$ is a function which assigns to any atomic formula $p_i$ a truth value

\[
I(p_i) \in {0,1}
\]

If $I(p_i)=1$ then $p_i$ is called true under the interpretation $I$  
If $I(p_i)=0$ then $p_i$ is called false under the interpretation $I$

\section{DEFINIZIONE (2)}
An interpretation $A$ is a structure

\section{NOTAZIONE}

\section{ESEMPIO}

\section{APPROFONDIMENTI}
\begin{itemize}
 \item \url{https://en.wikipedia.org/wiki/Interpretation_%28logic%29}
 \item Da qui prendi def2: \url{http://cgi.csc.liv.ac.uk/~frank/teaching/comp118/lecture2.pdf}
\end{itemize}

\end{document}
