\documentclass[a4paper,10pt]{article}
\usepackage[utf8]{inputenc}
\usepackage{amsmath}
\usepackage{hyperref}

%opening
\title{History}
\author{\href{http://www.baudo.hol.es}{giuseppe baudo}}

\begin{document}

\maketitle

\section{SYLLABUS}
\begin{itemize}
 \item Logic: \url{http://cgi.csc.liv.ac.uk/~frank/teaching/comp118/history.pdf}
 \item Funzione
 \item Derivata
 \item \href{People.html}{People}
 \item Cronologia generale: \url{http://www-groups.dcs.st-and.ac.uk/~history/Chronology/index.html}
 \item Biografie: \url{http://www-groups.dcs.st-and.ac.uk/~history/BiogIndex.html}
\end{itemize}

\section{Funzione}
La parola "funzione" appare per la prima volta verso la fine del XVII secolo, nella corrispondenza tra Leibniz e Johann Bernoulli; tuttavia \'{e} solo con l'opera di Eulero
che questo concetto si afferma come uno dei principali strumenti dell'analisi.

\section{Derivata}
Il calcolo differenziale nasce soprattutto per affrontare problemi di geometria, in particolare quello delle tangenti.

\end{document}
