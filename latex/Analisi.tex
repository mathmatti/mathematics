\documentclass[a4paper,10pt]{article}
\usepackage[utf8]{inputenc}
\usepackage{amsmath}
\usepackage{hyperref}

%opening
\title{Analisi matematica}
\author{}

\begin{document}

\maketitle

\section{Metric Spaces}
\begin{itemize}
 \item \href{Distanza.html}{Distanza}
 \item \href{MetricSpace.html}{Metric Space}
 \item \href{PointMetricSpace.html}{Point of a Metric Space}
 \item \href{SubsetMetricSpace.html}{Subset of a Metric Space}
 \item \href{DiametroSpazioMetrico.html}{Diametro di uno spazio metrico} 
 \item Spazio metrico limitato (è quello con diametro finito).
 \item Spazio metrico illimitato (è quello con diametro infinito). 
 \item Palla aperta (se l'intervallo contiene gli estremi)
 \item Palla chiusa (se l'intervallo non contiene gli estremi)
 \item Insiemi aperti, chiusi. 
 \item Unione e intersezione di insiemi aperti o chiusi. 
 \item Definizione di interno, chiusura e frontiera di un insieme e loro proprietà. 
 \item Spazi metrici e insiemi connessi. 
 \item Insiemi connessi in R. 
 \item Poligonale.
 \item Insiemi aperti connessi in C. 
 \item Successioni convergenti, punti limite. 
 \item La chiusura di un insieme coincide con i suoi punti limite. 
 \item Insiemi densi. 
 \item Successioni di Cauchy.
 \item Le successioni convergenti sono di Cauchy. 
 \item Una successione di Cauchy che ammette una sottosuccessione convergente è convergente. 
 \item Spazi metrici e insiemi completi.
 \item Completezza di C (assumendo R completo). 
 \item Un sottoinsieme di uno spazio metrico completo è completo se e solo se è chiuso. 
 \item Spazi metrici e insiemi (sequenzialmente) compatti. 
 \item Uno spazio metrico compatto è completo. 
 \item Spazi metrici totalmente limitati. 
 \item Uno spazio metrico totalmente limitato è limitato. 
 \item Uno spazio metrico è compatto se e solo se è completo e totalmente limitato. 
 \item Un sottoinsieme di R n è compatto se e solo se è chiuso e limitato. 
 \item Intorno
 \item \href{LimitPoint.html}{Limit point (Punto di accumulazione)}
 \item \href{InsiemeCompatto.html}{Insieme compatto}
\end{itemize}


\section{Syllabus}
\begin{itemize}
 \item inf e sup di cose: funzioni, insiemi, etc.
 \item Principio di induzione
 \item Teoria degli insiemi, funzioni, applicazione, prodotto cartesiano.
 \item Topologia
 \item \href{Function.html}{Funzione}
 \item \href{Tangent.html}{Tangent}
 \item \href{Arcsine.html}{Arcsine}
 \item Valore assoluto, esponenziali, logaritmi, radici, equazioni e disequazioni
 \item \href{./ValoreAssoluto.html}{Valore assoluto}
 \item \href{./FunzioneEsponenziale.html}{Funzione esponenziale}
 \item Dominio di una funzione
 \item Maggiorante di una funzione
 \item Minorante di una funzione
 \item Estremo superiore di una funzione
 \item \href{FunzioneMassimoMinimo.html}{Massimo e minimo di una funzione}
 \item \href{FunzioneContinua.html}{Funzione continua}
 \item \href{Limite.html}{Limite}
 \item \href{Weierstrass.html}{TEOREMA di Weierstrass. Una funzione continua in un insieme $E$ compatto ha massimo e minimo.}
 \item Successione
 \item \href{Derivata.html}{Derivata} 
 \item Integrale
 \item Lebensque
\end{itemize}

\section{Temi d'esame}
\begin{itemize}
 \item \url{http://www.math.unipd.it/~marson/didattica/Analisi1/temiAnalisi1.html}
 \item \href{http://www.math.unipd.it/~colombo/didattica/analisi1/}{http://www.math.unipd.it/~colombo/didattica/analisi1/}
 \item \url{http://www.uniba.it/docenti/mininni-michele/attivita-didattica/tracce/istituzioni-di-analisi-matematica-analisi-mat.-1}
 \item \url{http://paola-gervasio.unibs.it/Appelli_AM1/appelli.html}
 \item \href{http://calvino.polito.it/~terzafac/Corsi/analisi1/materiale.html}{http://calvino.polito.it/~terzafac/Corsi/analisi1/materiale.html}
 \item \url{http://calvino.polito.it/~lancelotti/didattica/analisi1_new/analisi1_new_temi.html}
 \item \url{http://www.dmi.units.it/~omari/Analisi_matematica_1_(2010-11)/Esercizi/Anex1.pdf}
 \item \url{http://users.dma.unipi.it/gobbino/Home_Page/Files/HP_AD/E99_CS.pdf}
 \item \href{http://www.dima.unige.it/~demari/Eser.pdf}{http://www.dima.unige.it/~demari/Eser.pdf}
\end{itemize}

\end{document}
