\documentclass[a4paper,10pt]{article}
\usepackage[utf8]{inputenc}
\usepackage{amsmath}
\usepackage{hyperref}

%opening
\title{Piano}
\author{\href{http://www.baudo.hol.es}{giuseppe baudo}}

\begin{document}

\maketitle

\section{DEFINIZIONE}
Si intende il piano euclideo quello che si studia nelle scuole dell'obbligo, vedi approfondimenti.

\section{NOTE}
Daremo anche una definizione algebrica.

\section{ESEMPIO}

\section{APPROFONDIMENTI}
\begin{itemize}
 \item \url{https://it.wikipedia.org/wiki/Geometria_euclidea}
 \item \url{https://it.wikipedia.org/wiki/Assioma}
 \item \url{http://www.astrofilibresciani.it/Biblioteca_UAB/Biblioteca/euclid_p.pdf}
 \item \url{http://www.webalice.it/francesco.odetti/EuclideSlidesCpct.pdf}
 \item \url{http://www.scienzaatscuola.it/euclide/index.html}
\end{itemize}

\end{document}
