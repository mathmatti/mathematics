\documentclass[a4paper,10pt]{article}
\usepackage[utf8]{inputenc}
\usepackage{amsmath}
\usepackage{hyperref}

%opening
\title{Sistema di riferimento cartesiano}
\author{\href{http://www.baudo.hol.es}{giuseppe baudo}}

\begin{document}

\maketitle

\section{DEFINIZIONE}
Il sistema cartesiano \`{e} un \href{SistemaRiferimento.html}{sistema di riferimento} formato da n rette ortogonali, intersecantesi tutte in un punto
chiamato origine, su ciascuna delle quali si fissa un orientamento (sono quindi dette orientate) e per le quali si fissa
anche un'unità di misura (cioè si fissa una metrica di solito euclidea) che consente di identificare qualsiasi punto dell'insieme
mediante n numeri reali. In questo caso si dice che i punti di questo insieme sono in uno spazio di dimensione n.

\subsection{from wikipedia in english}
A Cartesian coordinate system is a coordinate system that specifies each point uniquely in a plain by a pair of numerical coordinates,
which are the signed distances to the point from two fixed perpendicular directed lines, measured in the same unit of length.
Each reference line is called coordinate axis or just axis of the system, and the point where they meet is its origin, usually at
ordered pair $(0, 0)$. The coordinates can also be defined as the positions of the perpendicular projections of the point onto the
two axes, expressed as signed distances from the origin.

\section{NOTE}
Ogni punto è identificato da una coppia di numeri $(x,y)$ dove il primo numero rappresenta la distanza dall'asse y mentre il secondo
numero rappresenta la distanza dall'asse x. Pensateci è proprio così! Il fatto è che se parliamo di distanza, allora dobbiamo definire
che cosa si intende per distanza.

Per definizione (che vuol dire?), esiste una corrispondenza biunivoca fra i punti del piano cartesiano e le coppie ordinate di 
numeri reali. L'insieme di tutte le coppia di numeri reali, $R^2$ è un $R$-spazio vettoriale.

\section{ESEMPIO}

\section{APPROFONDIMENTI}
\begin{itemize}
 \item \url{https://it.wikipedia.org/wiki/Sistema_di_riferimento_cartesiano}
 \item \url{https://en.wikipedia.org/wiki/Cartesian_coordinate_system}
\end{itemize}

\end{document}
