\documentclass[a4paper,10pt]{article}
\usepackage[utf8]{inputenc}
\usepackage{amsmath}
\usepackage{hyperref}

%opening
\title{equivalence relation}
\author{\href{http://www.baudo.hol.es}{giuseppe baudo}}

\begin{document}

\maketitle

\section{Definizione}
A relation $\textasciitilde{}$ on a set $S$ is called an \textit{equivalence relation} if, for all $a, b, c \in S$, it satisfies:
\begin{itemize}
 \item $a \textasciitilde{} a$ (reflexivity)
 \item $a \textasciitilde{} b$ implies that $b \textasciitilde{} a$ (symmetry)
 \item $a \textasciitilde{} b$, $b \textasciitilde{} c$ implies that $a \textasciitilde{} c$ (transitivity)
\end{itemize}

\section{Symbol}
\textasciitilde{}



\section{Esempio}
L'uguaglianza, $=$, è una relazione di equivalenza.

\end{document}
