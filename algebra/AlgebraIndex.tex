\documentclass[a4paper,10pt]{article}
\usepackage[utf8]{inputenc}
\usepackage{amsmath}
\usepackage{hyperref}

%opening
\title{Algebra - Algebra 1 - Algebra 2 - Complementi di algebra - index}
\author{}

\begin{document}

\maketitle



\section{Prerequisites}
\begin{itemize}
 \item Teoria degli insiemi, funzioni, applicazione, prodotto cartesiano.
 \item \href{./CalcProb.html}{Elementi di calcolo delle probabilità}
 \item \href{./AssiomiProb.html}{Assiomatizzazione del calcolo delle probabilità}
 \item Matrici
\end{itemize}

\section{Syllabus}
\begin{itemize}
 \item \href{./FunzioneSuriettiva.html}{Funzione suriettiva (o surgettiva, o suriezione)}
 
 \item \href{./Permutazione.html}{Permutazione}
 \item \href{./PermutazioneCicli.html}{Ciclo di una permutazione}
 \item \href{./PermutazioniDisgiunte.html}{Cicli disgiunti}
 \item \href{./PermutazioneDecomposizioneCicli.html}{Decomposizione in cicli disgiunti}
 \item \href{./PermutazioneCicliPeriodo.html}{Ordine (o periodo) di un ciclo}
 \item \href{./PermutazionePeriodo.html}{Ordine (o periodo) di una permutazione}
 \item TEOREMA: Ogni permutazione si può scrivere come una composizione di cicli disgiunti in modo unico a meno dell'ordine dei cicli stessi
 \item \href{./Derangement.html}{Permutazione senza punti fissi, Derangement, partial derangement}
 \item Segno di una permutazione
 \item Orbita ???
 
 \item \href{./CoefficientiBinomiali.html}{Coefficienti binomiali: definizione e proprietà}
 \item \href{./PrincipioInclusioneEsclusione.html}{Principio di inclusione-esclusione}
 
 \item \href{./Gruppo.html}{Gruppo}
 \item Elemento neutro di un gruppo
 \item Sottogruppo
 \item Gruppo abeliano
 \item \href{./OrderOfGroup.html}{Order of a group}
 \item \href{./TeoremaDiLagrange.html}{Teorema di Lagrange e sue conseguenze} 
 \item \href{./CyclicGroup.html}{Gruppo ciclico}
 
 \item ESEMPIO GRUPPO: Gruppo generale lineare
 \item ESEMPIO GRUPPO: Gruppo simmetrico 
 \item ESEMPIO GRUPPO: classi resto modulo n rispetto alla somma
 \item ESEMPIO GRUPPO: Gruppo di permutazioni ??? non so se andrà qui.
 
 \item \href{./OmomorfismoGruppi.html}{Omomorfismo di gruppi}
 \item Nucleo, nucleo di una funzione, nucleo di applicazione lineare
 \item \href{./Nucleo.html}{Nucleo di omomorfismo di gruppi}
 \item Insieme degli omomorfismo di gruppi ( dati due gruppi $G$ e $H$ in simboli: $Hom(G,H)$)
 
\end{itemize}

\section{Esercizi}
\begin{itemize}
 \item \href{./esercizio1.html}{Esercizio 1: Dimostrare che una matrice è sottogruppo di $GL_{n}$}
 \item \href{./esercizio2.html}{Esercizio 2: Dimostrare che una funzione è omomorfismo di gruppi}
 \item \href{./esercizio3.html}{Esercizio 3: Trovare il nucleo di un omomorfismo di gruppi}
 \item \href{./esercizio4.html}{Esercizio 4: Applicazione delle formule per il derangement, partial derangement}
 \item \href{./esercizio5.html}{Esercizio 5: Calcolare le permutazioni di $S_{8}$, periodo e segno, inversa di una permutazione, Sottogruppo ciclico dell permutazioni}
\end{itemize}

\end{document}
