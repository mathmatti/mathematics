\documentclass[a4paper,10pt]{article}
\usepackage[utf8]{inputenc}
\usepackage{amsmath}
\usepackage{hyperref}
\usepackage[english]{babel}
\usepackage{natbib}

%opening
\title{Nucleo di un omomorfismo di gruppi}
\author{}

\begin{document}

\maketitle

\section{Definizione}
Sia $f:G \longrightarrow G'$ un omomorfismo di gruppi.
Chiamiamo nucleo di $f$, e lo indichiamo con $ker f$ (oppure $ker(f)$), il sottoinsieme $f^{-1}({1'})$, dove $1'$ è l'elemento netro di $G'$. \cite{progmat1}

In questa definizione occorre notare che per trovare il $ker$ bisogna trovare l'insieme degli elementi neutri di $G'$ e tramite l'inversa risalire agli elementi del ker.

\section{Prerequisites}
\begin{itemize}
 \item Elemento neutro
\end{itemize}



\bibliographystyle{plain} 
\bibliography{AlgebraIndex}
\end{document}
