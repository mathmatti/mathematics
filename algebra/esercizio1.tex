\documentclass[a4paper,10pt]{article}
\usepackage[utf8]{inputenc}
\usepackage{amsmath}
\usepackage{hyperref}

%opening
\title{ALGEBRA 1 - Esercizio 1}
\author{baudo81[at]gmail.com}

\begin{document}

\maketitle

\section{TESTO}
Sia  
\[
  G = \left\{ \left[ {\begin{array}{cc}
   a & b \\
   0 & a \\    
   \end{array} } 
   \right] ; a, b \in R, a \ne 0 \right\} 
\]

\begin{itemize}
 \item Dimostrare che $G$ è un sottogruppo di $GL_{2}(R)$.
 \item Dimostrare che la funzione $f:G\longrightarrow R^{*}$ definita da \[
  f \left( \left[ {\begin{array}{cc}
   a & b \\
   0 & a \\    
   \end{array} } 
   \right] \right) = a
\]
 
è un omomorfismo del gruppo $G$ nel gruppo moltiplicativo $R^{*}$.

\item Determinare il nucleo $ker(f)$.

\end{itemize}



\section{TEORIA}

\begin{enumerate}
 \item Definizione di \href{./sottogruppo.html}{sottogruppo} perchè devo far vedere che l'insieme dato soddisfa le proprietà della definizione di sottogruppo.
 \item \href{./GL.html}{Chi è $GL_{2}(R)$}
 \item \href{./MatriceInvertibile.html}{Inversa di una matrice quadrata}
\end{enumerate}





\section{SOLUZIONE}

\end{document}
