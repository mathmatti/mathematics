\documentclass[a4paper,10pt]{article}
\usepackage[utf8]{inputenc}
\usepackage{amsmath}
\usepackage{hyperref}

%opening
\title{Segno di una permutazione}
\author{\href{https://www.baudo.hol.es}{baudo}}

\begin{document}

\maketitle

\section{INTRODUZIONE}
Ogni permutazione di $S_n$, $n>2$, è prodotto di trasposizioni. Osserviamo però che tali trasposizioni possono non essere
disgiunte ed inoltre la rappresentazione di una permutazione como prodotto di trasposizioni non è unica. Ad esempio, la permutazione
$\alpha=(123)$, si può scrivere come: $\alpha=(13)(12)=(12)(23)=(23)(13)$. Il teorema del segno di una permutazione ci dice
però che la parità (ovvero il segno) di una permutazione rimane la stessa.

\section{DEFINIZIONE}
Sia $\alpha \in S_n$, $n \geq 2$. Si dice che $\alpha$ è pari se è prodotto di un numero pari di trasposizioni, dispari se è prodotto di un 
numero dispari di trasposizioni.

Inoltre si dice che il segno di $\alpha$, $sgn(\alpha)$, è 1 se $\alpha$ è pari, -1 se $\alpha$ è dispari.

\section{NOTAZIONE}

\section{ESEMPIO}

\section{APPROFONDIMENTI}
\begin{itemize}
 \item \href{http://progettomatematica.dm.unibo.it/Permutazioni/fr6.htm}{http://progettomatematica.dm.unibo.it/Permutazioni/fr6.htm}
\end{itemize}

\end{document}
