\documentclass[a4paper,10pt]{article}
\usepackage[utf8]{inputenc}
\usepackage{amsmath}
\usepackage{hyperref}

%opening
\title{Equazione parametrica della retta nello spazio}
\author{\href{http://www.baudo.hol.es}{giuseppe baudo}}

\begin{document}

\maketitle

\section{DEFINIZIONE (1)}
L'equazione parametrica di una retta parallela al vettore (non nullo) $(a,b,c)$ è passante per il punto $(x_0,y_0,z_0)$ è

 \begin{equation}
   \begin{cases}
   x=x_0 + ta \\
   y=y_0 + tb \\
   z=z_0 + tc
   \end{cases}
\end{equation}
con $t \in R$

\section{NOTE}
Trovare le equazioni parametriche e cartesiane di una retta nello spazio passante per due punti: 
\url{http://www.matematicamente.it/forum/retta-passante-per-due-punti-nello-spazio-t48348.html}

\section{NOTAZIONE}

\section{ESEMPIO}

\section{APPROFONDIMENTI}
\begin{itemize}
 \item \url{http://www.matematicamente.it/formulario-dizionario/formulario/geometria-analitica-nello-spazio-retta/}
 \item \url{}
\end{itemize}

\end{document}
