\documentclass[a4paper,10pt]{article}
\usepackage[utf8]{inputenc}
\usepackage{amsmath}
\usepackage{hyperref}


%opening
\title{Logica e Fondamenti}
\author{\href{http://www.baudo.hol.es}{giuseppe baudo}}

\begin{document}

\maketitle

\section{Introduction}
La logica si divide in logica dei predicati e logica delle proposizioni. La logica dei predicati è la logica delle proposizioni con l'aggiunta dei quantificatori
per ogni ($\forall$) ed esiste ($\exists$). Quando si passa alla logica dei predicati sembra che le proposizioni vengano rappresentate come delle funzioni

\section{Syllabus}
\begin{itemize}
 \item Notion
 \item Symbol
 \item Variable (symbol)
 \item Reasoning
 \item Magnitude
 \item Law of excluded middle
 \item Propositional function
 \item \href{Appartenenza.html}{Appartenenza}
 \item Inclusione
 \item Uguaglianza (tra insiemi)
 \item \href{TeoriaIngenuaInsiemi.html}{Teoria ingenua degli insiemi}
 \item Cantor = teoria dei numeri e teoria (ingenua) degli insiemi
 \item \href{./Esistenza.html}{Esistenza}
 \item \href{./Transfinito.html}{Transfinito}
 \item Coerenza
 \item Indipendenza
 \item \href{../history/history.html}{History}
 \item \href{./SviluppiDecimali.html}{Sviluppi decimali}
 \item Funziona caratteristica di un sottoinsieme.
 \item Rappresentazione posizionale dei numeri
 \item \href{./Interpretation.html}{Interpretation}
 \item \href{./Passaggio.html}{Passaggio da linguaggio naturale a linguaggio simbolico (linguaggio formale?)}
 \item Classical Logic
 \item \href{FormalLogic.html}{Formal Logic}
 \item \href{./IntuitiveLogic.html}{Intuitive Logic}
 \item Constructive Logic
 \item \href{ConstructiveMathematics.html}{Constructive Mathematics}
 \item Symbolic Logic
 \item \href{Entscheidungsproblem.html}{Entscheidungsproblem}
 \item \href{./RecursiveFunctionTheory.html}{Recursive Function Theory}
\end{itemize}

\section{Libri}
  \begin{itemize}
   \item cite{1}
   \item cite{2}
  \end{itemize}

\section{Temi d'esame}
  \begin{itemize}
   \item 1
  \end{itemize}

\section{Dispense di partenza}
\begin{itemize}
  \item Dispense Prof. Placci Unibo
  \item \url{http://www.settheory.net/}
  \item \url{http://www.mafy.lut.fi/study/LogicAndDiscreteMethods/Lectures/Lecture2.pdf}
  \item \url{http://cse.unl.edu/~choueiry/F07-235/files/PredicatesQuantifiers.pdf}
  \item \url{http://cgi.csc.liv.ac.uk/~frank/teaching/comp118/lecture2.pdf}
  \item \url{http://emilkirkegaard.dk/en/wp-content/uploads/0415400678.Routledge.Logic_.An_.Introduction.Dec_.2005.pdf}
\end{itemize}

\section{GOOGLE SEARCHES}
\begin{itemize}
 \item predicate logic solved examples
\end{itemize}


\end{document}
