\documentclass[a4paper,10pt]{article}
\usepackage[utf8]{inputenc}
\usepackage{amsmath}
\usepackage{hyperref}

\bibliographystyle{alpha}


%opening
\title{Intuitive Logic}
\author{\href{http://www.baudo.hol.es}{giuseppe baudo}}

\begin{document}

\maketitle

\section{DEFINIZIONE}
Intuitionistic logic, sometimes more generally called constructive logic, refers to systems of symbolic logic that differ from the systems used for classical logic 
by more closely mirroring the notion of constructive proof. 
In particular, systems of intuitionistic logic do not include the law of the excluded middle and double negation elimination, 
which are fundamental inference rules in classical logic. \cite{k1}

Intuitionistic logic encompasses the principles of logical reasoning which were used by L. E. J. Brouwer in developing his intuitionistic mathematics, 
beginning in [1907]. Because these principles also underly Russian recursive analysis and the constructive analysis of E. Bishop and his followers, 
intuitionistic logic may be considered the logical basis of constructive mathematics. \cite{k2}

\bibliography{LogicsBibTex}
\begin{thebibliography}{100}
  \bibitem{k1} \url{https://en.wikipedia.org/wiki/Intuitionistic_logic}
  \bibitem{k2} \url{https://plato.stanford.edu/entries/logic-intuitionistic/}
  \bibitem{k3} \url{https://plato.stanford.edu/entries/brouwer/}
\end{thebibliography}

\end{document}
