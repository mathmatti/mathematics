\documentclass[a4paper,10pt]{article}
\usepackage[utf8]{inputenc}
\usepackage{amsmath}
\usepackage{hyperref}

%opening
\title{Teoria ingenua degli insiemi}
\author{\href{http://www.baudo.hol.es}{giuseppe baudo}}

\begin{document}

\maketitle

\section{DEFINIZIONE}
La teoria ingenua degli insiemi è un modello matematico che si pone alla base di tutta la matematica.  
Putroppo successivamente si scoprirono dei bug ovvero
il modello portava a dei paradossi e quindi fu aggiustata da Zermelo, Fraenkel e Skolem (ZFS). 

Ecco il modello ingenuo:  
Un insieme è una collezione di elementi distinti con due particolarità:  
\begin{itemize}
 \item gli elementi dell'insieme possono essere, a loro volta, insiemi.
 \item esiste un insieme composto da nessun insieme, detto insieme vuoto.
\end{itemize}

\section{APPROFONDIMENTI}
\begin{itemize}
 \item \url{http://www.lidimatematici.it/blog/2010/12/09/cantor-e-la-teoria-ingenua-degli-insiemi/}
\end{itemize}

\end{document}
