\documentclass[a4paper,10pt]{article}
\usepackage[utf8]{inputenc}
\usepackage{amsmath}
\usepackage{hyperref}

%opening
\title{Aritmetica}

\begin{document}

\maketitle


\section{Introduzione}
Aritmetica, teoria dei numeri...
\section{Syllabus}
\begin{itemize}
 \item item1
\end{itemize}

\section{Primi materiali da vedere}
\begin{itemize}
 \item libro: \url{http://gen.lib.rus.ec/book/index.php?md5=D2DC7AD499257DF413CAD18252837065}
 \item \url{https://en.wikipedia.org/wiki/Divisor_function}
 \item \url{https://en.wikipedia.org/wiki/Arithmetic_function#.CE.A9.28n.29.2C_.CF.89.28n.29.2C_.CE.BDp.28n.29_.E2.80.93_prime_power_decomposition}
 \item \url{https://oeis.org/A001221}
 \item \url{https://it.wikipedia.org/wiki/Teorema_di_fattorizzazione_di_Weierstrass}
 \item \url{https://en.wikipedia.org/wiki/Prime_factor}
 \item \url{https://math.stackexchange.com/questions/689546/show-that-the-set-of-all-subsets-of-an-infinite-enumerable-set-is-not-enumerable}
\end{itemize}

\end{document}