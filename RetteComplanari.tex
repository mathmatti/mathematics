\documentclass[a4paper,10pt]{article}
\usepackage[utf8]{inputenc}
\usepackage{amsmath}
\usepackage{hyperref}

%opening
\title{Rette complanari}
\author{\href{http://www.baudo.hol.es}{giuseppe baudo}}

\begin{document}

\maketitle

\section{DEFINIZIONE}
Due rette si dicono complanari se appartengono allo stesso piano.

\section{NOTE}
Calcolare se due rette sono complanari:
\begin{itemize}
 \item \url{http://www.youmath.it/domande-a-risposte/view/6183-rette-complanari.html}
\end{itemize}

\section{ESEMPIO}

\section{APPROFONDIMENTI}
\begin{itemize}
 \item \url{http://www.youmath.it/domande-a-risposte/view/6183-rette-complanari.html}
\end{itemize}

\end{document}
