\documentclass[a4paper,10pt]{article}
\usepackage[utf8]{inputenc}

%opening
%\title{}
%\author{}

\begin{document}

%\maketitle

\section{Limit point of a set}
\subsection{Definition (1)}
Let $ X $ be a metric space. All points and sets mentioned below
are understood to be elements and subsets of  $ X $.

A point $ p $ is a \textit{limit point} of the set $ E $ if \textit{every} neighborhood of $ p $ 
contains a point $ q \ne p $ such that $ q \in E $.

\subsection{Definition (2)}
In what follows, $R$ is the reference space, that is all the sets are subsets of $R$.

Let $S \subseteq R$, and let $x \in R$.

1. $x$ is a limit point or an accumulation point or a cluster point of $S$
if $\foreach \delta >0, (x-\delta,x+delta) \cap S\{x}\ne \o{o} $

2. The set of limit points of a set S is denoted L(S)


\subsection{Remarks}
Scriviamo la definizione di neighborhood in maniera estesa: 

$ N_{r}(p)=\{q \in X : d(p,q)<r, r>0 \} $

La definizione dice che ogni $ N_{r}(p) $ contiene un $ q \ne p $ 

L'insieme $ N_{r}(p) $ puo' essere vuoto? NO almeno deve contenere $ p $

Se l'insieme $ N_{r}(p) $ non può essere vuoto e contiene $ q $ vuol dire che ogni insieme contiene almeno due elementi!?

Intuitively speaking, a limit point of a set $S$ in a space $X$ is a point of $X$ which 
can be approximated by points of $S$ other than $x$ as well as one pleases.

The notion of limit point is an extension of the notion of being "close" to a
set in the sense that it tries to measure how crowded the set is. To be a limit
point of a set, a point must be surrounded by an infinite number of points of 
the set.
\end{document}
