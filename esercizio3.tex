\documentclass[a4paper,10pt]{article}
\usepackage[utf8]{inputenc}
\usepackage{amsmath}
\usepackage{hyperref}

%opening
\title{Esercizio 3: Trovare il nucleo di un omomorfismo di gruppi}
\author{baudo81[at]gmail.com}

\begin{document}

\maketitle

\section{TESTO}
Sia  
\[
  G = \left\{ \left[ {\begin{array}{cc}
   a & b \\
   0 & a \\    
   \end{array} } 
   \right] ; a, b \in R, a \ne 0 \right\} 
\]

\begin{itemize}
 \item Dimostrare che $G$ è un sottogruppo di $GL_{2}(R)$.
 \item Dimostrare che la funzione $f:G\longrightarrow R^{*}$ definita da \[
  f \left( \left[ {\begin{array}{cc}
   a & b \\
   0 & a \\    
   \end{array} } 
   \right] \right) = a
\]
 
è un omomorfismo del gruppo $G$ nel gruppo moltiplicativo $R^{*}$.

\item Determinare il nucleo $ker(f)$.

\end{itemize}

\section{TEORIA}
\begin{itemize}
 \item Teoria degli insiemi
 \item Nucleo di un omomorfismo di gruppi
\end{itemize}


\section{SOLUZIONE}
\[
 ker(f) = \left\{ A \in G, tali che f(A) = 1 \right\} = \left\{ \left[ \begin{array}{cc}
   1 & b \\
   0 & 1 \\    
   \end{array} \right]; b \in R \right\} 
\]

\section{IDEA A BASE DELLA SOLUZIONE}
Ho individuato facilmente gli elementi neutri di $R^{*}$ e cioè 1!!! dopodichè ho cercato l'elemento di 
$G$ che applicato tramite la $f$ mi dia come risultato 1. Lo si può fare a occhio.
\end{document}
