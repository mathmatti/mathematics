\documentclass[a4paper,10pt]{article}
\usepackage[utf8]{inputenc}
\usepackage{amsmath}
\usepackage{hyperref}

%opening
\title{Matrice associata ad un applicazione lineare tra spazi vettoriali}
\author{\href{http://www.baudo.hol.es}{giuseppe baudo}}

\begin{document}

\maketitle

\section{DEFINIZIONE}
La matrice associata ad un'applicazione lineare tra spazi vettoriali si ottiene mettendo nelle colonne della matrice
le immagini dei vettori della base del dominio (o di altra base data).

\section{NOTAZIONE}

\section{ESEMPIO}

\section{APPROFONDIMENTI}
\begin{itemize}
 \item \url{http://www.dm.unibo.it/~ida/NoteGeometria1-25-5-16.pdf}
 \item \url{http://joshua.smcvt.edu/linearalgebra/book.pdf}
 \item \url{http://www.youmath.it/lezioni/algebra-lineare/applicazioni-lineari/771-calcolare-dimensione-e-base-di-nucleo-e-immagine.html}
\end{itemize}

\end{document}
