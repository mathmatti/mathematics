\chapter{Limiti e continuità}
\section{Limiti di funzioni in spazi metrici}
\begin{definizione}
\label{d:limite}
Sia $f\colon E\subseteq(X_1,d_1)\to(X_2,d_2)$ una funzione, un'applicazione tra differenti spazi metrici\footnote{Le successioni sono particolari applicazioni di questo tipo in cui $(X_1,d_1)=\N$.}.
Siano $p\in E'$ e $\ell\in X_2$. Si dice che la funzione $f$ ha limite $\ell$ per $x\to p$ se, fissato arbitrariamente un intorno di $\ell$, esiste in corrispondenza un intorno di $p$ tale che tutti i valori compresi nell'intersezione di $E$ con l'intorno di $p$ hanno come immagine un valore compreso nell'intorno di $\ell$.
\[
\forall\epsilon>0\;\exists\delta>0\colon\forall x\in E\cap B(p,\delta)\text{ con }x\neq p\longrightarrow f(x)\in B(\ell,\epsilon).
\]
\end{definizione}
\begin{figure}
	\tikzsetnextfilename{limite-funzione}
	\centering
	\begin{tikzpicture}
		\begin{axis}[standard,xmin=0,ymin=0,xmax=6.5,ymax=40,xtick={4,5,6},xticklabels={$x_0-\delta$,$x_0$,$x_0+\delta$},ytick={16,25,36},yticklabels={$\ell-\epsilon$,$\ell$,$\ell+\epsilon$},xlabel=$x$,ylabel=$y$]
			\addplot[samples=200,domain=0:6.5]{x^2};
			\addplot[dotted] coordinates {(4,0) (4,16)};
			\addplot[dotted] coordinates {(6,0) (6,36)};
			\addplot[dotted] coordinates {(0,16) (4,16)};
			\addplot[dotted] coordinates {(0,36) (6,36)};
			\addplot[dashed] coordinates {(5,0) (5,25)};
			\addplot[dashed] coordinates {(0,25) (5,25)};
		\end{axis}
	\end{tikzpicture}
	\caption{Limite di una funzione (non necessariamente si ha $\ell=f(x_0)$).}
	\label{fig:limite}
\end{figure}


\begin{teorema}
Sia $f\colon E\subseteq(X_1,d_1)\to(X_2,d_2)$ con $p\in E'$ e $\ell\in X_2$. Le seguenti affermazioni sono equivalenti:
\begin{enumerate}
\item$\lim_{x\to p}f(x)=\ell$.
\item$\forall\{x_n\}\subset E$ con $x_n\to p$ si ha $f(x_n)\to\ell$.
\end{enumerate}
\end{teorema}
\begin{proof}
\begin{enumerate}
\item Sia data la successione $\{x_n\}$: se essa tende a $p$, allora fissato un $\delta>0$ la distanza tra $x_n$ e $p$ è definitivamente minore di $\delta$. Quindi poiché vale la 1), queste immagini di $x_n$ ``vicine'' a $p$, cioè $f(x_n)$, sono contenute nell'intorno $B(\ell,\epsilon)$, quindi $f(x_n)\to\ell$.
\item Sia per assurdo che esista $\epsilon>0$ tale per cui $\forall\delta>0$ $\exists \tilde{x}\in B(p,\delta)$, con $\tilde{x}\neq p$, tale che $f(\tilde{x})\notin B(\ell,\epsilon)$, cioè che $d_2\big(f(\tilde{x}),\ell\big)>\epsilon$ ossia $f(x)\not\to\ell$.
Si consideri $\delta$ come una successione, ad esempio $\delta=1/n$. Per ogni valore di $\delta$ si trova in corrispondenza un $x_n$ per cui $f(x_n)\notin B(\ell,\epsilon)$. Poiché $\delta\to0$, allora $x_n\to p$, ma allo stesso tempo la successione delle immagini non è definitivamente contenuta in $B(\ell,\epsilon)$, vale a dire che se $f(x)\not\to\ell$ allora nemmeno $f(x_n)\to\ell$ il che è assurdo poiché contrasta con la 2). Allora se $x\to p$ deve essere che $f(x)\to\ell$.\qedhere
\end{enumerate}
\end{proof}
Per questo teorema allora si possono applicare alle funzioni tutti i teoremi per i limiti di successioni già visti.
\begin{teorema}[di permanenza del segno]
\label{t:permanenza_segno-f}
Sia $(X,d)$ uno spazio metrico e siano $E\subseteq X$, $p\in E'$. Sia $f\colon E\to\R$ una funzione tale che per $x\to p$, $f(x)\to\ell$.
\begin{enumerate}
\item Se $\ell$ è positivo, allora $\exists B(p,\delta)$ tale che la funzione sia positiva in $B(p,\delta)\cap E$, con $x\neq p$;
\item Se $\exists B(p,\delta)$ per cui la funzione sia positiva in $B(p,\delta)\cap E$, con $x\neq p$, allora $\ell\geq 0$;
\end{enumerate}
\end{teorema}
\begin{proof}
Sia $\ell>0$ il limite a cui converge $f(x)$ per $x\to p$: esiste un $\epsilon$ tale che $\ell-\epsilon>0$. Per la convergenza della funzione in $B(p,\delta)\cap E$ si ha che i valori della funzione sono contenuti nell'intorno $(\ell-\epsilon,\ell+\epsilon)$, cioè sono sempre maggiori di $\ell-\epsilon$ che è positivo, quindi sono positivi.

Sia ora che la funzione è positiva in $B(p,\delta)\cap E$. Se $\ell$ fosse negativo, allora per il punto precedente nell'intorno la funzione dovrebbe assumere valori negativi, che è assurdo: allora $\ell$ non è negativo.
\end{proof}
Analogamente si dimostra che se la funzione ha un limite negativo nell'intorno, allora i valori della funzione sono ivi negativi, e che se una funzione è negativa nell'intorno il limite può essere solo negativo o nullo.
\begin{teorema}[del confronto]
\begin{enumerate}
\item Siano $f,g,h$ tre funzioni definite in $E\subseteq(X,d)$ a valori reali, e $p\in E'$. Se $\lim_{x\to p}f(x)=\lim_{x\to p}h(x)=\ell$ ed esiste un intorno $B(p,\delta)$ per cui $\forall x\in B(p,\delta)\cap E$ si ha $f(x)\leq g(x)\leq h(x)$, allora anche $g(x)$ ammette limite per $x\to p$, e tale limite è $\ell$.
\item Siano $f,g$ due funzioni definite in $E\subseteq(X,d)$ a valori reali, e $p\in E'$. Se esiste un intorno $B(p,\delta)$ per cui $\forall x\in B(p,\delta)\cap E$ si ha $f(x)\leq g(x)$:
\begin{itemize}
\item se $f(x)\to+\infty$ per $x\to p$, allora anche $g$ ammette limite, che è $+\infty$;
\item se $g(x)\to-\infty$ per $x\to p$, allora anche $f$ ammette limite, che è $-\infty$. 
\end{itemize}
\end{enumerate}
\end{teorema}
\begin{proof}
\begin{enumerate}
\item Poiché $f$ e $h$ convergono a $\ell$, si ha in $B(p,\delta)\cap E$ che $\forall\epsilon>0$:
\begin{itemize}
\item $\exists\delta_1\colon\forall x\in B(p,\delta_1)\then\ell-\epsilon<f(x)<\ell+\epsilon$;
\item $\exists\delta_2\colon\forall x\in B(p,\delta_2)\then\ell-\epsilon<h(x)<\ell+\epsilon$.
\end{itemize}
Allora, per un qualunque $\tilde{\delta}<\min\{\delta,\delta_1,\delta_2\}$, per l'ipotesi deve valere
\[
\ell-\epsilon<f(x)\leq g(x)\leq h(x)<\ell+\epsilon,
\]
cioè $\ell-\epsilon<g(x)<\ell+\epsilon$, che significa che nell'intorno $B$, quindi per $x\to p$, si ha $g(x)\to\ell$.
\item Poiché $f$ diverge a $+\infty$, si ha che in $B(p,\delta)\cap E$ che $\forall M>0$, $\exists\delta_3\colon\forall x\in B(p,\delta_3)$ $f(x)>M$. Allora per un qualunque $\tilde{\delta}<\min\{\delta,\delta_3\}$ vale $g(x)\geq f(x)>M$, cioè $g(x)\to+\infty$ per $x\to p$.\qedhere
\end{enumerate}
\end{proof}

\section{Limiti di funzioni reali a valori reali}
\begin{definizione}
Siano $f\colon A\subseteq\R\to\R$, $p\in A'$ e $\ell\in\R$. Si dice che $\lim_{x\to p}f(x)=\ell$ se
\[
\forall\epsilon>0\;\exists\delta>0\colon\forall x\in A\text{ con }x\neq p\text{ e }\abs{x-p}<\delta\text{ si ha che }\abs{f(x)-\ell}<\epsilon.
\]
\end{definizione}
Estendendo $\ell$ a $\Rex$ si ha che
\begin{itemize}
\item$\forall M>0\;\exists\delta>0\colon\forall x\in A$ con $x\neq p$ e $\abs{x-p}<\delta$ si ha che $f(x)>M$;
\item$\forall M>0\;\exists\delta>0\colon\forall x\in A$ con $x\neq p$ e $\abs{x-p}<\delta$ si ha che $f(x)<-M$.
\end{itemize}
Nel primo caso si ha che $\ell=+\infty$, nel secondo $\ell=-\infty$. In ogni caso, la retta $x=p$ è un asintoto verticale per la funzione $f(x)$.
Estendendo inoltre anche $p$ a $\Rex$, si ha che se $\lim_{x\to\pm\infty}f(x)=\ell$:
\begin{itemize}
\item se $A$ non è inferiormente limitato, allora $\forall\epsilon>0\;\exists k\colon\forall x\in A$ con $x>k$ si ha che $\abs{f(x)-\ell}<\epsilon$;
\item se $A$ non è superiormente limitato, allora $\forall\epsilon>0\;\exists k\colon\forall x\in A$ con $x<k$ si ha che $\abs{f(x)-\ell}<\epsilon$.
\end{itemize}
Nel primo caso è $p\to+\infty$, nel secondo $p\to-\infty$.
Si definiscono anche, con le opportune ipotesi, i limiti del tipo $\lim_{x\to\pm\infty}=\pm\infty$.

Se, in particolare, non esiste il limite di $f(x)$ per $x\to p$, si può definire comunque il limite per difetto (``da sinistra'') o per eccesso (``da destra''), rispettivamente calcolati sull'intorno sinistro e destro di $p$, ossia per $x\to p^-$, in $(p-\delta,p]$, e per $x\to p^+$, in $[p,p+\delta)$. Quando il limite per $x\to p$ esiste, questi due limiti coincidono.

Tutti i teoremi e le proprietà dei limiti per le successioni valgono anche per i limiti di funzioni reali a valori reali.

\section{Asintoti}
Può accadere che il grafico di una funzione tenda ad approssimarsi ad una retta all'avvicinarsi della variabile verso un punto di accumulazione: ciò vuol dire che la distanza tra la funzione e tale retta, detta \emph{asintoto}, è infinitesima.
\begin{definizione} \label{d:asintoto}
Sia $f$ una funzione a valori reali definita, almeno, in un intorno di un suo punto di accumulazione $p$. La retta $r$ si dice \emph{asintoto} di $f$ pe $x\to p$ se
\[
\lim_{x_\to p}d\big(f(x),r\big)=0.
\]
\end{definizione}
Ovviamente qualsiasi retta può essere asintoto di una funzione: sono particolari gli asintoti verticali e orizzontali:
\begin{itemize}
\item La retta $y=a$ è un asintoto orizzontale per la funzione $f(x)$ se $f(x)\to a$ per $x\to-\infty$ o $x\to+\infty$, o anche entrambi. Nei primi casi, si chiama rispettivamente asintoto orizzontale sinistro e destro.
\item La retta $x=b$ è un asintoto verticale per la funzione $f(x)$ se per almeno $x\to b^-$ o per $x\to b^+$ (ovviamente anche se valgono entrambi), si ha $f(x)\to\infty$.
\end{itemize}
Gli asintoti obliqui invece si possono individuare, in assenza di asintoti orizzontali, con le equazioni
\[
m=\lim_{x\to\infty}\frac{f(x)}{x}\qtext{e} q=\lim_{x\to\infty}\big[f(x)-mx\big],
\]
diversificando per $+\infty$ e $-\infty$. L'asintoto obliquo è la retta $y=mx+q$ soltanto se entrambi i limiti esistono finiti e se $m\neq 0$. Notare che il risultato $m=0$ \emph{non} significa che l'asintoto è orizzontale, e nemmeno il fatto che la derivata tende a zero per $x\to\pm\infty$: basta vedere la funzione $f(x)=\sqrt{x}$, per cui si avrebbe con questi ultimi due metodi $m=0$ o una derivata infinitesima nell'intorno di $+\infty$, ma chiaramente non esiste alcun asintoto per $x\to+\infty$.
Un altro metodo per individuarli è riuscire ad esprimere la funzione nella forma $f(x)=mx+q+o(1)$ dove $o(1)$ è un generico infinitesimo per $x\to\pm\infty$.
Ovviamente gli asintoti obliqui e gli asintoti orizzontali sono mutuamente esclusivi: una funzione non può avere entrambi per $x\to-\infty$ o per $x\to+\infty$.

\section{Continuità}
\begin{definizione} \label{d:continuita}
Sia $f\colon E\subseteq (X_1,d_1)\to(X_2,d_2)$, e $p$ un punto appartenente ad $E$. Si dice che la funzione $f$ è continua in $p$ se per ogni $\epsilon>0$ esiste in corrispondenza un $\delta>0$ tale che per ogni punto $x$ nell'insieme $E$ e in un intorno di $p$ di raggio $\delta$ si ha che l'immagine di $x$ appartiene all'intorno di $f(p)$ di raggio $\epsilon$.
\begin{equation} \label{eq:def-cont}
\forall\epsilon>0\;\exists\delta>0\colon\forall x\in E\cap B(p,\delta)\text{ si ha che } f(x)\in B\big(f(p),\epsilon\big).
\end{equation}
\end{definizione}
A differenza della definizione di limite, che è simile a questa, la definizione di continuità impone che $p$ appartenga all'insieme di definizione $E$, inoltre non è necessario che sia un punto di accumulazione.
Infatti se un punto è isolato, scelto opportunamente il raggio $\delta$, nessun altro valore di $x$ oltre a $p$, nell'intorno $B(p,\delta)$, soddisfa la definizione di continuità, e ovviamente la sua immagine $f(p)$ è contenuta nell'intorno $B\big(f(p),\epsilon\big)$ per qualunque valore di $\epsilon$; allora ogni punto isolato soddisfa ``automaticamente'' la definizione di continuità. Quindi segue che tutte le successioni sono continue: l'insieme $\N$ infatti è composto da soli punti isolati.
Se invece il punto è di accumulazione, allora $f$ è continua se e solo se $\lim_{x\to p} f(x)=f(p)$.
Ovviamente una funzione è continua in tutto un insieme se e solo se è continua in tutti i punti che lo compongono.

\begin{teorema}
Siano $f\colon E\subseteq (X_1,d_1)\to(X_2,d_2)$ e $g\colon X_2\to (X_3,d_3)$, e $p$ un punto di $E$. Se $f$ è continua in $p$, e $g$ è continua in $f(p)$, allora la funzione $g\circ f$ è continua in $p$.
\end{teorema}
\begin{teorema}
La funzione $f\colon E\subseteq (X_1,d_1)\to(X_2,d_2)$ è continua in tutto $E$ se e solo se per ogni insieme $A$ aperto in $X_2$ la sua controimmagine $f^{-1}(A)$ è aperta in $X_1$.
\end{teorema}
\begin{proof}
Siano $A$ un insieme aperto in $X_2$, e il punto $p\in f^{-1}(A)$. La sua immagine è $f(p)\in A$, e dato che $A$ è aperto allora $p$ è necessariamente interno, quindi $\forall\epsilon>0$ $\exists B_2\big(f(p),\epsilon\big)\subseteq A$, cioè $f(p)$ è interno ad $A$. Dato che $f$ è continua in $p$, dato tale $\epsilon$ esiste un raggio $\delta$ che definisce un intorno $B_1(p,\delta)$ per cui $f(B_1)\subseteq B_2\big(f(p),\epsilon\big)\subseteq A$.  Poiché allora $f(B_1)\subseteq A$, passando alle controimmagini risulta che $B_1(p,\delta)\subseteq f^{-1}(A)$, che quindi è aperto.

Sia ora $p\in E$: poiché la funzione è definita in $E$, esiste l'immagine $f(p)$. Scelto un $\epsilon>0$, l'intorno $B_2\big(f(p),\epsilon\big)$ è per definizione aperto. Sia $A$ tale intorno, per ipotesi si ha che $f^{-1}\big(B_2\big(f(p),\epsilon\big)\big)\equiv f^{-1}(A)$ è aperto in $X_1$ e certamente $p$, poiché è la controimmagine di $f(p)$, appartiene a questo insieme. Allora $p$ è interno, quindi esiste un $\delta$ per cui $B_1(p,\delta)\subseteq f^{-1}(A)$, vale a dire $f\big(B_1(p,\delta)\big)\subseteq B_2\big(f(p),\epsilon\big)$, ossia la funzione è continua in $p$ secondo la definzione \ref{d:continuita}.
\end{proof}
\begin{teorema}
\label{t:comp_to_comp}
Sia la funzione $f\colon E\subseteq (X_1,d_1)\to(X_2,d_2)$ continua in tutto $E$. Se l'insieme $K$ è compatto, allora la sua immagine $f(K)$ è a sua volta un insieme compatto.
\end{teorema}
\begin{proof}
Sia la successione $\{y_n\}\subset f(K)$: allora ogni elemento $y_n$ è immagine di un qualche punto in $E$, ossia $y_n=f(x_n)$, Allora si ha che $\{x_n\}\subseteq K$; poiché $K$ è compatto si può estrarre da $x_n$ una sottosuccessione $x_{n_j}$ che converga ad un valore $\tilde{x}\in K$. La continuità di $f$ quindi implica che $f(x_{n_j})\to f(\tilde{x})$, cioè che $y_{n_j}\to f(\tilde{x})\in f(K)$, e ciò significa che $f(K)$ è compatto.
\end{proof}
Da questo teorema segue un importante teorema per le funzioni a valori reali, definite in qualunque spazio metrico.
\begin{teorema}[Weierstrass]
\label{t:weierstrass}
Se $f\colon E\subseteq (X,d)\to\R$ è una funzione continua in tutto $E$, e tale insieme è compatto, allora $f$ ammette massimo e minimo assoluti, vale a dire $f(E)$ è limitato e ammette un massimo e un minimo.
\end{teorema}
\begin{proof}
Per il teorema \ref{t:comp_to_comp} l'immagine $f(E)$ è un insieme compatto, e dato che $f(E)\subset\R$ esso è anche chiuso e limitato: allora deve avere sia un estremo superiore che inferiore, che sono anche il massimo e il minimo di tale insieme, dato che è chiuso.
\end{proof}
Se una funzione è continua non soltanto in un punto, ma in tutto un insieme, la definizione \ref{d:continuita} deve essere verificata per tutti i punti $p\in D$, indicando con $D$ tale insieme. Allora in generale il raggio $\delta$ dell'intorno di $p$ non dipende solo dal valore di $\epsilon$, ma anche dal punto $p$ in esame. Esistono certe funzioni, però, tali per cui $\delta$ può non essere in relazione con $p$ e allora fissato $\epsilon$, dei punti che distano tra di loro meno di $\delta(\epsilon)$ hanno delle immagini che distano meno di $\epsilon$ per \emph{qualunque} punto dell'insieme considerato. Queste funzioni particolari rispondono alla definizione seguente.
\begin{definizione}
\label{d:continuita_uniforme}
La funzione $f\colon E\subseteq (X_1,d_1)\to(X_2,d_2)$ si dice \emph{uniformemente continua} in $E$ se, fissato $\epsilon$ esiste $\delta$ tale che $\forall x_1,x_2\in E$ per cui $d_1(x_1,x_2)<\delta$ si ha $d_2\big(f(x_1),f(x_2)\big)<\epsilon$.
\end{definizione}
A questo punto ai fini della continuità della funzione non è più importante quale sia il punto $p$ nell'insieme $E$, ma si considera soltanto la distanza tra una qualsiasi coppia di punti. Per le funzioni uniformemente continue infatti il raggio $\delta$ non dipende più dal punto $p$ preso come centro dell'intorno.

\begin{teorema}[Heine-Cantor]
\label{t:heine-cantor}
Se $f\colon E\subseteq (X_1,d_1)\to(X_2,d_2)$ è una funzione continua in $E$ e se tale insieme è compatto, allora $f$ è uniformemente continua in $E$.
\end{teorema}
\begin{proof}
Sia per assurdo che $f$ non sia uniformemente continua, ossia che esista $\epsilon>0$ tale che, per ogni $\delta>0$, esiste almeno una coppia di punti $x_i,y_i\in E$ per cui se $d_1(x_i,y_i)$ allora $d_2\big(f(x_i),f(y_i)\big)\geq\epsilon$.
Posto $\delta=1$, si ha che $\exists x_1,y_1\colon d_1(x_1,y_1)<1$ ma $d_2\big(f(x_1),f(y_1)\big)\geq\epsilon$. Posto poi $\delta=1/2$, si ha che $\exists x_2,y_2\colon d_1(x_2,y_2)<1/2$ ma $d_2\big(f(x_2),f(y_2)\big)\geq\epsilon$. Procedendo in questo modo si costruisce una successione di $\delta$ infinitesima, per cui $\delta=1/n$, si ha che $\exists x_n,y_n\colon d_1(x_n,y_n)<1/n$ ma $d_2\big(f(x_n),f(y_n)\big)\geq\epsilon$. Però l'insieme $E$ è compatto, perciò $\exists x_{n_k}\to\tilde{x}\in E$, e analogamente $y_{n_k}\to\tilde{y}\in E$. Dato che la distanza tra questi due elementi è $d_1(x_{n_k},y_{n_k})$ e tende a 0, i due limiti delle sottosuccessioni devono coincidere, allora $\tilde{x}=\tilde{y}$, quindi $y_{n_k}\to\tilde{x}$. A questo punto anche le rispettive immagini dei limiti delle due sottosuccessioni devono coincidere, perché la funzione è continua, ad un valore $f(\tilde{x})$, e ciò significa che $d_2\big(f(x_{n_k}),f(y_{n_k})\big)\to 0$, il che è però assurdo dato che per ipotesi si ha che quest'ultima distanza non può essere minore di $\epsilon$. La funzione allora è uniformemente continua in $E$.
\end{proof}
Chiaramente l'uniforme continuità della funzione non dipende soltanto dalla sua formula algebrica, ma anche dall'insieme in cui la si studia. Ad esempio, la funzione $f(x)=x^2$ non è uniformemente continua in un qualsiasi intervallo illimitato, ma lo diventa non appena l'intervallo in questione è chiuso e limitato. In generale, come conseguenza, ogni funzione continua in un intervallo $[a,b]\in\R$ è anche uniformemente continua, nello stesso.
Inoltre questo teorema fornisce una condizione soltanto sufficiente: esistono funzioni continua in insiemi non compatti che sono anche uniformemente continue.

Un'altra condizione sufficiente, più semplice da verificare, per verificare la continuità uniforme di una funzione è la condizione di Lipschitz.
\begin{definizione}
\label{d:lipschitz}
Una funzione $f\colon E\subseteq (X_1,d_1)\to(X_2,d_2)$ si dice \emph{lipschitziana}, di costante $M>0$, nell'insieme $E$ se
\[
\forall x,y\in E\text{ si ha che }d_2\big(f(x_1),f(x_2)\big)\leq M\cdot d_1(x,y).
\]
\end{definizione}
Tutte le funzioni lipschitziane sono anche uniformemente continue, con $\delta=\epsilon/M$.

\section{Continuità di funzioni reali a valori reali}
Tutte le funzioni elementari reali sono continue nel loro insieme di definizione. Il prodotto, la somma e la composizione di funzioni continue è ancora a sua volta una funzione continua. Per questa particolare categoria di funzioni sono inoltre dati alcuni importanti teoremi.
\begin{teorema}[di esistenza degli zeri] \label{t:esistenza_zeri}
Sia $f:[a,b]\to\R$ una funzione continua nell'intervallo $[a,b]$. Se $f(a)\cdot f(b)<0$, allora $\exists\tilde{x}\in[a,b]\colon f(\tilde{x})=0$.
\end{teorema}
\begin{proof}
Siano $f(a)<0$ e $f(b)>0$. Si consideri il punto medio di $I_0\defeq[a,b]$ e la rispettiva immagine $f(\frac{a+b}{2})$, e l'intervallo $I_1=[a_1,b_1]$, lungo la metà di $I_0$, tale per cui o $a_1=\frac{a+b}{2}$ o $b_1=\frac{a+b}{2}$ in modo che si abbia ancora che $f(a_1)\cdot f(b_1)<0$. Procedendo in questo modo, bisecando l'intervallo sempre di più, si trova una successione di intervalli $I_n=[a_n,b_n]$ in cui un elemento è sempre compreso nel precedente nell'ordine.
La successione $a_n$ è monotona crescente, inoltre è sempre $f(a_n)<0$, mentre $b_n$ è monotona decrescente e si ha $f(b_n)>0$. La distanza tra i due $a_n$ e $b_n$ è quindi
\[
	\abs{b_n-a_n}=\frac1{2^n}(b-a),
\]
perché ogni intervallo come già detto è ampio la metà del precedente, e questa distanza è infinitesima. Quindi sia $a_n$ che $b_n$ tendono ad un unico limite $\tilde{x}$. Per il teorema di permanenza del segno, risulta che $f(a_n)\to f(\tilde{x})\leq 0$, e allo stesso tempo $f(b_n)\to f(\tilde{x})\geq 0$ perciò non può che essere $f(\tilde{x})=0$.
\end{proof}
\begin{teorema}[dei valori intermedi]
\label{t:valori_intermedi}
Se $f$ è una funzione a valori reali definita e continua su $[a,b]\in\R$, allora essa assume tutti i valori compresi tra il suo massimo e il suo minimo.
\end{teorema}
\begin{proof}
L'intervallo $[a,b]$ è compatto quindi $f$ per il teorema \ref{t:weierstrass} ammette un massimo e un minimo (assoluti). Sia $m$ tale minimo e $M$ il massimo. Traslando la funzione in verticale di una costante $c$ opportuna, in modo che $m<0$ e $M>0$, allora deve esistere per il teorema \ref{t:esistenza_zeri} un punto in cui $f$ si annulla. Per qualsiasi valore di $c$ tale che $m\leq c\leq M$ allora deve esistere in corrispondenza un punto in cui ciò succede, allora $f$ assume tutti i valori compresi tra $m$ e $M$.
\end{proof}
Come conclusione, e conseguenza, di questi ultimi teoremi, resta il teorema seguente, di cui si dà la dimostrazione solo per la seconda parte.
\begin{teorema}[Darboux]
\label{t:darboux}
Sia $I\subseteq\R$ un intervallo di qualunque tipo: se $f\colon I\to\R$ è continua (non costante), allora $f(I)$ è a sua volta un intervallo. Se inoltre $I$ è un intervallo chiuso e limitato, anche la sua immagine è un intervallo chiuso e limitato.
\end{teorema}
\begin{proof}
Se $I$ è un intervallo limitato e chiuso, del tipo $[a,b]$, poiché la funzione è a variabile reale, per il teorema di \ref{t:heine-cantor} Heine-Cantor l'intervallo chiuso è anche compatto, quindi per il teorema \ref{t:weierstrass} di Weiestrass la sua immagine ammette un massimo e un minimo (che sono il minimo e il massimo della funzione per $x\in[a,b]$). Inoltre per il teorema precedente assume tutti i valori tra tali massimo e minimo, quindi l'immagine è un intervallo chiuso.
\end{proof}

\section{Punti di discontinuità}
Sia $f$ una funzione definita a valori reali definita (almeno) nell'intervallo $(a,b)$. La funzione si dice continua in un punto $x_0\in (a,b)$ se i limiti per difetto ed eccesso della funzione per $x\to x_0$ esistono finiti e coincidono entrambi con il valore della funzione in $x_0$.
Un punto si dice \emph{di discontinuità}, banalmente, se in esso la funzione non è continua. Si può anche ampliare la definizione nel caso in cui la funzione non sia definita in tale punto, ma anche soltanto in un intorno destro e sinistro di esso, dato che deve, in ogni caso, sempre essere possibile calcolare il limite per tale punto.
Sia quindi la funzione definita come $f\colon(a,x_0)\cup(x_0,b)\to\R$, almeno; essa ammette in $x_0$ una discontinuità:
\begin{itemize}
\item di \emph{prima specie} se non esiste il limite in $x_0$, ma i limiti sinistro e destro esistono finiti e sono differenti.
\[
\nexists\lim_{x\to x_0} f(x)\text{, ma }\lim_{x\to x_0^-} f(x)=\ell_1\text{ e }\lim_{x\to x_0^+} f(x)=\ell_2\text{, con }\ell_1\neq\ell_2.
\]
\item di \emph{seconda specie} se almeno uno dei limiti sinistro o destro in $x_0$ è infinito o non esiste. La retta verticale di equazione $x=x_0$ inoltre si dice \emph{asintoto verticale} della funzione $f(x)$.
\item di \emph{terza specie}, o eliminabile, se esiste finito il limite per $x_0$ ma in tale punto la funzione non esiste o assume un valore differente dal limite. Questo tipo di discontinuità si può eliminare definendo una nuova funzione come
\[
g(x)=
	\begin{dcases*}
	f(x)& per $x\neq x_0$\\
	\lim_{x\to x_0} f(x)& per $x=x_0$.
	\end{dcases*}
\]
\end{itemize}
\section{Funzioni monotone}
Per definire se una funzione è crescente, o decrescente, c'è bisogno di una relazione d'ordine che permetta di confrontare più grandezze. Questa relazione è presente solo in $\R$ (e nei suoi sottoinsiemi), quindi è possibile confrontare numeri appartenenti al più a $\R\times\R$, vale a dire un numero reale con un numero reale. Solo in questo spazio metrico, quindi, può avere un senso la seguente definizione.
\begin{definizione}
\label{d:monotonia}
Sia $f\colon I\subseteq\R\to\R$, in cui $I$ è un intervallo. La funzione $f$, per ogni $x_1,x_2\in I$ si dice monotona:
\begin{itemize}
\item crescente, se per $x_1<x_2$ si ha che $f(x_1)<f(x_2)$;
\item decrescente, se per $x_1<x_2$ si ha che $f(x_1)>f(x_2)$.
\end{itemize}
\end{definizione}
\begin{teorema}
Sia $f\colon(\alpha,\beta)\to\R$ una funzione monotona con $\alpha,\beta\in\Rex$ e $\alpha<\beta$: allora esistono e sono finiti i limiti per $x\to\alpha^+$ e per $x\to\beta^-$, inoltre se è
\begin{itemize}
\item crescente, allora
\[
\lim_{x\to\alpha^+}f(x)=\inf_{\alpha<\beta}f(x)\text{ e }\lim_{x\to\beta^-}f(x)=\sup_{\alpha<\beta}f(x);
\]
\item decrescente, allora
\[
\lim_{x\to\alpha^+}f(x)=\sup_{\alpha<\beta}f(x)\text{ e }\lim_{x\to\beta^-}f(x)=\inf_{\alpha<\beta}f(x).
\]
\end{itemize}
\end{teorema}
Questo teorema vale in tutti i punti dell'intervallo, inoltre i limiti devono essere finiti poiché tutti i punti di $(\alpha,\beta)$ devono poter essere ordinati. Da questo teorema si deduce che la funzione non può tendere all'infinito all'interno dell'intervallo in cui è monotona, né possono esserci punti in cui i limiti non esistono o sono differenti dal valore della funzione, dato che salterebbe l'ipotesi di monotonia o di intervallo: allora una funzione monotona può presentare punti di discontinuità solamente di prima specie.
\begin{teorema}
Se $f\colon(a,b)\subseteq\R\to\R$ è una funzione monotona, le sue discontinuità sono solo di prima specie, e sono al più numerabili.
\end{teorema}
\begin{proof}
Sia $x_0\in(a,b)$ un punto di discontinuità: allora esiste il limite della $f(x)$ per $x\to x_0$ sia per difetto che per eccesso, e questi sono necessariamente diversi altrimenti sarebbe continua, e finiti, dato che sono limitati da altri valori della funzione, che è monotona.
Sia $f$ crescente: esiste un intervallo $(\ell_1,\ell_2)$, attorno al punto di discontinuità, in cui $\ell_1\defeq\lim_{x\to x_0^-}f(x)$ e $\ell_2\defeq\lim_{x\to x_0^+}f(x)$. Ad un altro punto $\tilde{x}_0$ si associa in modo analogo l'intervallo $(\tilde{\ell}_1,\tilde{\ell}_2)$. Dato che $f$ è monotona, deve essere che $(\ell_1,\ell_2)\cap(\tilde{\ell}_1,\tilde{\ell}_2)=\emptyset$, poiché $\tilde{\ell}_1\geq \ell_2$. Infatti $\ell_2=\inf_{x>x_0}f(x)$ e $\tilde{\ell}_1=\sup_{x<x_0}f(x)$, e dato che $x_0<\tilde{x}_0$ non può mai succedere che $\ell_2>\tilde{\ell}_1$ (al più si ha il caso in cui sono uguali, come per la funzione $f(x)=\partint{x}$).
Ciò significa che $(\ell_1,\ell_2)\cap(\tilde{\ell}_1,\tilde{\ell}_2)=\emptyset$, anche se sono aperti il caso in cui $\ell_2=\tilde{\ell}_1$ non è contemplato.
Tutti questi intervalli trovati sono quindi disgiunti, e per la densità di $\R$ in ognuno è possibile individuare (almeno) un numero razionale e uno irrazionale. Ad ogni punto di discontinuità si associa allora il numero razionale: $\exists p_1\in\Q$ con $p_1\in(\ell_1,\ell_2)$, ossia si applica una funzione che è biiettiva, quindi le discontinuità sono in corrispondenza biunivoca con $\Q$, cioè sono numerabili.
\end{proof}

\section{Funzioni invertibili}
Una funzione iniettiva è sempre invertibile, e dato che le funzioni strettamente monotone sono tutte iniettive, allora sono anche invertibili. Non vale il contrario, ma se la funzione è sia iniettiva che continua in un intervallo, allora è anche strettamente monotona (in tale intervallo).
\begin{teorema}
Sia $f\colon I\to\R$ continua e invertibile nell'intervallo $I$. La sua inversa $f^{-1}\colon f(I)\equiv J\to\R$, dove $J$ è a sua volta un intervallo per il teorema \ref{t:darboux}, è a sua volta continua in $J$.
\end{teorema}
