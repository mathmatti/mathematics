\begin{figure}
	\tikzsetnextfilename{potenziale-lineare}
	\centering
	\begin{tikzpicture}
		\begin{axis}[
				standard,
				enlargelimits,
				height=.5\linewidth, width=\linewidth,
				xlabel=$\xi$,
				xmin=-5, xmax=8.5,
				ymin=-0.5, ymax=0.6,
				xtick={-4,-2,0,2,4,6}, ytick={-0.5,0.5},
				yticklabels={$-\frac12$,$\frac12$}
			]
			\addplot[thick,samples=1000,densely dotted,domain=-5:8] function {airy(-x)};
			\addplot[thick,samples=1000,domain=-5:8] function {(airy(-x))**2};
			\legend{$\psi(\xi)$,$\abs{\psi(\xi)}^2$}
		\end{axis}
	\end{tikzpicture}
	\caption{Soluzione dell'equazione di Schr\"odinger per il potenziale lineare, nella variabile $\xi=\big(\frac{2ma}{\hbar^2}\big)^{1/3}\big(x-\frac{E}{a}\big)$. Il punto $\xi=0$ corrisponde al punto di inversione, in cui $x=\frac{E}{a}$, dove $E$ è l'energia del sistema e $a$ è la costante in $V(x)=-ax$.}
	\label{fig:potenziale-lineare}
\end{figure}
