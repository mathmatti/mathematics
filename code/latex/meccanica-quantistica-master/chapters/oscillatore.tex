\chapter{L'oscillatore armonico}
Il sistema fisico più semplice da studiare in meccanica quantistica, contrariamente alla tradizione, non è la particella libera, in quanto l'operatore impulso (che sarebbe l'unico ad apparire nell'hamiltoniana) è un po' particolare e richiede una trattazione dedicata, che faremo in seguito.
Il sistema più semplice risulta invece essere l'oscillatore armonico, che studiamo qui in una dimensione.
Occupiamoci del problema fondamentale di individuare gli autovalori possibili dell'operatore hamiltoniano, e la loro probabilità associata.
Diamo innanzitutto l'operatore hamiltoniano, in cui sostituiamo già la costante $k$ della forza di richiamo con $m\omega^2$, in modo da evidenziare la frequenza naturale dell'oscillatore ($\omega^2=k/m$):
\begin{equation}
	\op H=\frac1{2m}(\op p^2+m^2\omega^2\op q^2).
	\label{eq:H-oscillatore}
\end{equation}
Chiamiamo $E_0,E_1,\dots,E_n,\dots$ i suoi autovalori, ordinati in modo crescente, ossia con $E_i\leq E_{i+1}$ per ogni $i$, e indichiamo di conseguenza gli autostati corrispondenti con $\ket{E_i}$.
L'autostato $\ket{E_0}$, per cui l'energia è minima, è anche chiamato \emph{autostato fondamentale}.
Cominciamo con un primo importante risultato.
\begin{teorema} \label{t:oscillatore-autovalori-positivi}
	Gli autovalori di $\op H$ di un oscillatore armonico sono tutti positivi.
\end{teorema}
\begin{proof}
	Preso uno stato qualunque $\ket{x}$, il valore di aspettazione dell'energia è
	\begin{equation}
		\avg{H}=\bra{x}\op H\ket{x}=\frac1{2m}(\bra{x}\op p^2\ket{x}+m^2\omega^2\bra{x}\op q^2\ket{x}).
	\end{equation}
	Ora, dato che $\op p$ è hermitiano, $\bra{x}\op p^2\ket{x}=\bra{x}\adj{\op p}\op p\ket{x}$ che è la norma di $\op p\ket{x}$, e lo stesso per la posizione, dunque sono entrambi positivi o nulli: allora $\bra{x}\op H\ket{x}\geq 0$ qualsiasi sia $\ket{x}$.
	Se vale per qualsiasi stato, sarà vero anche per gli autostati $\ket{E_i}$, e se li prendiamo normalizzati allora
	\begin{equation}
		0\leq\bra{E_i}\op H\ket{E_i}=E_i\braket{E_i}{E_i}=E_i
	\end{equation}
	cioè tutti gli autovalori non sono negativi.
	Se prendiamo poi l'autostato fondamentale $\ket{E_0}$, esso non può avere l'autovalore nullo: infatti $E_0=0$ se e solo se $\bra{E_0}\op H\ket{E_0}=0$, vale a dire
	\begin{equation}
		\frac1{2m}(\bra{E_0}\op p^2\ket{E_0}+m^2\omega^2\bra{E_0}\op q\ket{E_0})=0,
	\end{equation}
	ma sono tutti addendi non negativi quindi l'equazione è vera se e solo se $\bra{E_0}\op p^2\ket{E_0}=0$ e contemporaneamente $\bra{E_0}\op q^2\ket{E_0}$.
	Come prima, però, questi due sono la norma di $\op p\ket{E_0}$ e $\op q\ket{E_0}$, e ciò significherebbe che questi due vettori siano nulli, vale a dire $\op q\ket{E_0}=0$ e $\op p\ket{E_0}=0$, e di conseguenza $\ket{E_0}$ sarebbe un autostato simultaneo di $q$ e $p$.
	Questo fatto però viola il rapporto di indeterminazione per cui $\Delta q\Delta p\geq\frac{\hbar}2$ in \emph{qualsiasi} stato: in questo caso invece si avrebbe $\Delta q\Delta p=0$, in quanto sapremmo con precisione posizione e impulso (entrambi nulli).
	Allora è assurdo che $E_0$ sia nullo: se non è nemmeno positivo per quanto dimostrato precedentemente, allora $E_0>0$.
	Dato che $E_0$ è l'autovalore minimo, segue immediatamente che $E_i>0$ per ogni $i$, ossia ogni autovalore di $H$ è positivo.
\end{proof}

Nell'oscillatore armonico riconosciamo anche la simmetria che è presente anche nella controparte classica: i valori di aspettazione della posizione e dell'impulso sono zero, negli autostati di energia.
Infatti, preso un qualunque operatore $\op A$,
\begin{equation}
	\bra{E_i}[\op H,\op A]\ket{E_i}=\bra{E_i}\op H\op A\ket{E_i}-\bra{E_i}\op A\op H\ket{E_i}=E_i^*\bra{E_i}\op p\ket{E_i}-E_i\bra{E_i}\op p\ket{E_i}=0
\end{equation}
in quanto $E_i\in\R$.
Allo stesso tempo, il commutatore con $\op p$ vale
\begin{equation}
	[\op H,\op p]=\frac1{2m}[\op p^2+m^2\omega^2\op q^2,\op p]=\frac1{2m}[\op p^2,\op p]+\frac{m\omega^2}{2}[\op q^2,\op p]=\frac{m\omega^2}2(\op q[\op q,\op p]+[\op q,\op p]\op q)=i\hbar m\omega^2\op q\ne 0
	\label{eq:commutatore-Hp}
\end{equation}
cioè $\op q=[\op H,\op p]/i\hbar m\omega^2$.
Di conseguenza il valore di aspettazione di $q$ da un autostato di energia risulta
\begin{equation}
	\bra{E_i}\op q\ket{E_i}=\frac1{i\hbar m\omega^2}\bra{E_i}[\op H,\op p]\ket{E_i}=0.
\end{equation}
Con lo stesso ragionamento, calcolando $[\op H,\op q]$ si giunge a $\bra{E_i}\op p\ket{E_i}=0$.

Calcoliamo infine lo stato di energia minima: assumendo $\braket{E_0}{E_0}=1$, abbiamo
\begin{equation}
	E_0=\bra{E_0}\op H\ket{E_0}=\frac1{2m}(\bra{E_0}\op p^2\ket{E_0}+m^2\omega^2\bra{E_0}\op q^2\ket{E_0}),
\end{equation}
ma $\bra{E_0}\op p^2\ket{E_0}=\avg{p^2}$, e dato che $\avg{p}=0$ si ha $\Delta p^2=\avg{p^2}-\avg{p}^2=\avg{p^2}$ (analogamente per $q$), quindi\footnote{Sfruttiamo la disuguaglianza $a^2+b^2\ge 2ab$, per $a,b\in\R$: basta considerare $(a-b)^2\ge 0$ per dimostrarla.}
\begin{equation}
	E_0=\frac1{2m}(\avg{p^2}+m^2\omega^2\avg{q^2})=\frac1{2m}(\Delta p^2+m^2\omega^2\Delta q^2)\ge \frac1{2m}2m\omega\Delta p\Delta q=\omega\Delta p\Delta q\ge\frac{\hbar\omega}2,
\end{equation}
e in particolare vale proprio $\hbar\omega/2$ soltanto se $\Delta p=m\omega\Delta q$.
Pertanto lo stato di minima energia, che è allo stesso tempo lo stato di \emph{minima indeterminazione}, è dato da
\begin{equation}
	\begin{cases}
		\Delta p=m\omega\Delta q\\ \Delta p\Delta q=\frac{\hbar}2.
	\end{cases}
	\label{eq:oscillatore-armonico-stato-minima-energia}
\end{equation}

\section{Operatori di creazione e distruzione}
Per calcolare lo spettro dell'hamiltoniano, come faremo nella prossima sezione, ci conviene introdurre due nuovi operatori: definiamo
\begin{equation}
	\op a=\frac1{\sqrt{2\hbar m\omega}}(\op p-im\omega\op q)
	\label{eq:operatore-distruzione}
\end{equation}
e di conseguenza il suo aggiunto
\begin{equation}
	\adj{\op a}=\frac1{\sqrt{2\hbar m\omega}}(\op p+im\omega\op q).
	\label{eq:operatore-creazione}
\end{equation}
Essi sono chiamati rispettivamente operatore di \emph{distruzione} e di \emph{creazione}, per ragioni che vedremo più avanti.
Il loro prodotto è dato da
\begin{multline}
	\adj{\op a}\op a=\frac1{2\hbar m\omega}(\op p+im\omega\op q)(\op p-im\omega\op q)=\frac1{2\hbar m\omega}(\op p^2+im\omega\op q\op p-im\omega\op p\op q+m^2\omega^2\op q^2)=\\=\frac1{2\hbar m\omega}(\op p^2+im\omega[\op q,\op p]+m^2\omega^2\op q^2)=\frac{\op H}{\hbar\omega}+\frac{im\omega[\op q,\op p]}{2\hbar m\omega}=\frac{\op H}{\hbar\omega}-\frac12,
\end{multline}
mentre
\begin{equation}
	\op a\adj{\op a}=\frac{\op H}{\hbar\omega}+\frac12.
\end{equation}
Il commutatore tra i due operatori è dunque $[\op a,\adj{\op a}]=1$.
Possiamo anche ricavare l'hamiltoniano da questi due operatori, trovando che $\op H=\hbar\omega(\adj{\op a}\op a+\frac12)=\hbar\omega(\op a\adj{\op a}-\frac12)$, e di conseguenza
\begin{equation}
	[\op H,\op a]=\hbar\omega[\adj{\op a}\op a, a]=\hbar\omega([\adj{\op a}[\op a,\op a]+[\adj{\op a},\op a]\op a)=-\hbar\omega\op a,
\end{equation}
e anche, dato che il commutatore è antihermitiano, $[\op H,\adj{\op a}]=-\adj{[\op H,\op a]}=\hbar\omega\adj{\op a}$.

Semplificando ulteriormente, definiamo l'operatore \emph{numero} $\op N=\adj{\op a}\op a$, con il quale possiamo riscrivere l'hamiltoniano come
\begin{equation}
	\op H=\hbar\omega\bigg(\op N+\frac12\bigg).
	\label{eq:hamiltoniano-numero}
\end{equation}
L'operatore $\op N$ è evidentemente hermitiano, e vediamo che i suoi autostati sono anche autostati dell'hamiltoniano: se esiste uno stato $\ket{n}$ tale che $\op N\ket{n}=n\ket{n}$, allora
\begin{equation}
	\op H\ket{n}=\hbar\omega\bigg(\op N+\frac12\bigg)\ket{n}=\hbar\omega\bigg(n+\frac12\bigg)\ket{n}.
	\label{eq:autostato-H-N}
\end{equation}
Abbiamo inoltre i commutatori
\begin{equation}
	[\op a,\adj{\op a}]=1,\qquad [\op N,\op a]=-\op a,\qquad [\op N,\adj{\op a}]=\adj{\op a}.
	\label{eq:commutatori-algebra-heisenberg}
\end{equation}
L'insieme di questi tre operatori forma un'algebra detta \emph{algebra di Heisenberg}.

\section{Spettro dell'hamiltoniano}
Sia $\ket{E}$ un autostato di $H$: prendiamo lo stato che risulta applicando l'operatore di distruzione $\op a$ introdotto precedentemente, e questa volta l'azione dell'hamiltoniana è
\begin{equation}
	\op H\op a\ket{E}=(\op H\op a-\op a\op H+\op a\op H)\ket{E}=([\op H,\op a]+\op a\op H)\ket{E}=-\hbar\omega\op a\ket{E}+\op a E\ket{E}=(E-\hbar\omega)\op a\ket{E},
\end{equation}
ossia $\op a\ket{E}$ è ancora un autostato dell'hamiltoniano, ma con l'autovalore (di $H$) diminuito di $\hbar\omega$.
Applicando nuovamente $\op a$ si ottiene un ulteriore autostato con autovalore diminuito di $2\hbar\omega$ rispetto a quello di partenza.
Troviamo quindi una catena di autovalori $\{E,E-\hbar\omega,E-2\hbar\omega,\dots\}$: essa non potrà continuare indefinitamente, perch\'e gli autovalori di $H$ sono tutti positivi.
Infatti si arriverà ad un certo stato $\ket{x}$ tale che $\op a\ket{x}=0$: applicando ancora $\op a$, non troviamo un autovalore diminuito di $\hbar\omega$, ma troviamo l'equazione $\op H\op a\ket{x}=(E-\hbar\omega)\op a\ket{x}$: dato che $\op a\ket{x}=0$, quindi essa si riduce a $0=0$, e non ha senso proseguire con gli autovalori.
Se $E_0$ è l'autovalore minimo e $\ket{E_0}$ è il suo autostato, esso dovrà avere il valore minimo accettabile di energia, e sarà quindi tale che $\op a\ket{E_0}=0$.
Di conseguenza anche la sua norma è nulla, perciò
\begin{equation}
	0=\bra{E_0}\adj{\op a}\op a\ket{E_0}=\bra{E_0}\frac{\op H}{\hbar\omega}-\frac12\ket{E_0}=\Big(\frac{E_0}{\hbar\omega}-\frac12\Big)\braket{E_0}{E_0}
\end{equation}
poich\'e $\bra{E_0}\op H=\adj{\op H\ket{E_0}}=\op H\ket{E_0}=E_0\ket{E_0}$.
Risulta quindi
\begin{equation}
	E_0=\frac{\hbar\omega}2.
	\label{eq:oscillatore-armonico-energia-minima}
\end{equation}

Se ora guardiamo a $\adj{\op a}$, con un ragionamento analogo troviamo in un autostato di $H$ che
\begin{equation}
	\op H\adj{\op a}\ket{E}=([\op H,\adj{\op a}]+\adj{\op a}\op H)\ket{E}=(E+\hbar\omega)\adj{\op a}\ket{E},
\end{equation}
ossia $\adj{\op a}$ porta ogni autostato dell'hamiltoniano in un altro autostato con autovalore (sempre di $H$) aumentato di $\hbar\omega$.
Dunque, se partiamo da $\ket{E_0}$, possiamo applicare indefinitamente l'operatore o esiste un'energia massima che non si può oltrepassare?
Se esistesse un autostato $\ket{E_M}$ (per il quale $\op H\ket{E_M}=E_M\ket{E_M}$) tale che $\adj{\op a}\ket{E_M}=0$, si avrebbe come prima
\begin{equation}
	0=\bra{E_M}\op a\adj{\op a}\ket{E_M}=\bra{E_M}\frac{\op H}{\hbar\omega}+\frac12\ket{E_M}=\Big(\frac{E_M}{\hbar\omega}+\frac12\Big)\braket{E_M}{E_M},
\end{equation}
ma $E_M=-\hbar\omega/2$ contraddice il fatto che tutti gli autovalori di $H$ sono positivi, quindi un tale autostato non può esistere.

Abbiamo trovato dunque che gli autostati di $H$ soddisfano l'equazione
\begin{equation}
	\op H(\adj{\op a})^k\ket{E_0}=(E_0+k\hbar\omega)(\adj{\op a})^k\ket{E_0}=\Big(\frac{\hbar\omega}2+k\hbar\omega\Big)(\adj{\op a})^k\ket{E_0}=\Big(k+\frac12\Big)\hbar\omega(\adj{\op a})^k\ket{E_0},
\end{equation}
ossia che gli autovalori $E_n$ sono dati da
\begin{equation}
	E_n=\Big(\frac12+n\Big)\hbar\omega
	\label{eq:oscillatore-armonico-autovalori-hamiltoniano}
\end{equation}
per $n\in\{0,1,2,\dots\}$.
Questi devono necessariamente essere tutti e soli gli autovalori perch\'e, se ne esistesse uno con un espressione diversa, applicando $\op a$ ripetutamente non si incontrerebbe più l'autovalore $E_0$ (scendendo a ``gradini'' di $\hbar\omega$, verrebbe saltato) e si scenderebbe cos\`i in autovalori negativi, che è assurdo.
Non conoscendo se $H$ sia degenere o meno, non possiamo però affermare che gli autostati $\ket{E_n}$ relativi a tali autovalori sono unici.\footnote{L'hamiltoniano \emph{non è} degenere, quindi sono realmente tutti e soli gli autostati, ma questo non lo dimostriamo ora.}

\section{Autostati di energia}
Abbiamo dunque ora un insieme di autostati $\{\ket{E_n}\}_{n=0}^{+\infty}$ dell'hamiltoniano, tali che
\begin{equation}
	\op H\ket{E_n}=\hbar\omega\Big(n+\frac12\Big)\ket{E_n}.
\end{equation}
Su di essi, gli operatori di creazione e distruzione agiscono come
\begin{equation}
	\op H\op a\ket{E_n}=\hbar\omega\bigg((n-1)+\frac12\bigg)\op a\ket{E_n}=\op H\ket{E_{n-1}}
\end{equation}
dunque $\op a\ket{E_n}=\ket{E_{n-1}}$ (per $n\le 1$, altrimenti $\op a\ket{E_0}=0$ come abbiamo già visto) e analogamente $\adj{\op a}\ket{E_n}=\ket{E_{n+1}}$.

Ci rimane da capire se questo insieme forma un sistema ortonormale completo per lo spazio di Hilbert degli stati.
Per quanto abbiamo visto, questi stati si ottengono tutti applicando ripetutamente $\adj{\op a}$ allo stato fondamentale $\ket{E_0}$, ma in generale $(\adj{\op a})^n\ket{0}=\ket{E_n}$ non è normalizzato.
Per calcolare la sua norma dobbiamo calcolare $\bra{E_0}\op a^n(\adj{\op a})^n\ket{E_0}$: sarebbe comodo se ci trovassimo con dei termini $\bra{E_0}\adj{\op a}$ oppure $\op a\ket{E_0}$, che sono nulli, in modo da semplificare i calcoli: vediamo allora come commutano $\op a$ e $(\adj{\op a})^n$.
Abbiamo già visto che $[\op a,\adj{\op a}]=[\op a,(\adj{\op a})^1]=1=1(\adj{\op a})^{1-1}$, dunque potremmo supporre che $[\op a,(\adj{\op a})^n]=n(\adj{\op a})^{n-1}$.
Dimostriamolo per induzione: se assumiamo che $[\op a,(\adj{\op a})^{n-1}]=(n-1)(\adj{\op a})^{n-2}$, allora troviamo
\begin{equation}
	\begin{split}
		[\op a,(\adj{\op a})^n]&=[\op a,\adj{\op a}(\adj{\op a})^{n-1}]=\\
		&=\adj{\op a}[\op a,(\adj{\op a})^{n-1}]+[\op a,\adj{\op a}](\adj{\op a})^{n-1}=\\
		&=\adj{\op a}(n-1)(\adj{\op a})^{n-2}+(\adj{\op a})^{n-1}=\\
		&=(n-1)(\adj{\op a})^{n-1}+(\adj{\op a})^{n-1}=\\
		&=n(\adj{\op a})^{n-1}.
	\end{split}
\end{equation}
Allora abbiamo
\begin{equation}
	\begin{split}
		\bra{E_0}\op a^n(\adj{\op a})^n\ket{E_0}&=\bra{E_0}\op a^{n-1}\op a(\adj{\op a})^n\ket{E_0}=\\
		&=\bra{E_0}\op a^{n-1}\big(\op a(\adj{\op a})^n-(\adj{\op a})^n\op a+(\adj{\op a})^n\op a\big)\ket{E_0}=\\
		&=\bra{E_0}\op a^{n-1}[\op a,(\adj{\op a})^n]\ket{E_0}+\bra{E_0}\op a^{n-1}(\adj{\op a})^n\op a\ket{E_0}=\\
		&=\bra{E_0}\op a^{n-1}[\op a,(\adj{\op a})^n]\ket{E_0}
	\end{split}
\end{equation}
dove $\bra{E_0}\op a^{n-1}(\adj{\op a})^n\op a\ket{E_0}=0$ poich\'e $\op a\ket{E_0}=0$.
Sostituendo il commutatore con quanto trovato in precedenza, otteniamo
\begin{equation}
	\bra{E_0}\op a^n(\adj{\op a})^n\ket{E_0}=n\bra{E_0}\op a^{n-1}(\adj{\op a})^{n-1}\ket{E_0},
\end{equation}
di conseguenza iteriamo il calcolo fino ad ottenere
\begin{equation}
	\bra{E_0}\op a^n(\adj{\op a})^n\ket{E_0}=n!\braket{E_0}{E_0}.
\end{equation}
Definiamo quindi un nuovo insieme di stati $\{\ket{n}\}_{n=0}^{+\infty}$ con
\begin{equation}
	\ket{n}\defeq\frac1{\sqrt{n!}}(\adj{\op a})^n\ket{E_0},
	\label{eq:sonc-oscillatore-armonico}
\end{equation}
e $\ket{0}\defeq\ket{E_0}$.
Sono evidentemente ancora autostati dell'hamiltoniano, e adesso se $\braket{0}{0}=1$ allora anche $\braket{n}{n}=1$ per ogni $n\in\N$.

Calcoliamo nuovamente l'azione degli operatori di creazione e distruzione su questi nuovi stati: notiamo innanzitutto che, essendo $\ket{n}$ multiplo di $\ket{E_n}$, abbiamo sempre $\op H\ket{n}=\hbar\omega\big(n+\frac12\big)\ket{n}$.
Inoltre $\op a\ket{E_n}=\ket{E_{n-1}}$ (per $n\ge 1$), dunque una relazione simile si avrà anche per gli stati $\ket{n}$: poniamo $\op a\ket{n}=\lambda_n\ket{n-1}$.
Prendendone la norma, abbiamo $\bra{n}\adj{\op a}\op a\ket{n}=\lambda_n^*\lambda_n\braket{n-1}{n-1}=\abs{\lambda_n}^2$.
Allo stesso tempo, si ha $\adj{\op a}\op a=\op N$, e dalla \eqref{eq:hamiltoniano-numero} risulta dunque
\begin{equation}
	\abs{\lambda_n}^2=\bra{n}\adj{\op a}\op a\ket{n}=\bra{n}\bigg(\frac{\op H}{\hbar\omega}-\frac12\bigg)\ket{n}=\bra{n}\bigg(n+\frac12-\frac12\bigg)\ket{n}=n\braket{n}{n}=n.
\end{equation}
Di conseguenza $\lambda_n=e^{i\theta_n}\sqrt{n}$ per un certo $\theta_n\in\R$ arbitrario: scegliamo ovviamente $\theta_n=0$ ottenendo\footnote{Dato che multipli dello stesso \emph{ket} rappresentano il medesimo stato fisico, la scelta della fase è arbitraria poich\'e per ogni $\theta_n$ scelto lo stato risulta comunque normalizzato.}
\begin{equation}
	\op a\ket{n}=\sqrt{n}\ket{n-1}.
\end{equation}
Con un ragionamento del tutto analogo, ipotizzando questa volta che $\adj{\op a}\ket{n}=\lambda_n'\ket{n+1}$, si trova $\adj{\op a}\ket{n}=\sqrt{n+1}\ket{n+1}$.
L'operatore numero allora è tale che
\begin{equation}
	\op N\ket{n}=\adj{\op a}\op a\ket{n}=\sqrt{n}\adj{\op a}\ket{n-1}=\sqrt{n(n-1+1)}\ket{n}=n\ket{n}
\end{equation}
ossia ``conta'' i quanti $\hbar\omega$ di energia, o il livello energetico dello stato (da cui il suo nome).

Gli stati $\ket{n}$ formano dunque un sistema ortonormale completo, quindi possiamo assumerli come base per il nostro spazio degli stati.
Infatti
\begin{equation}
	\begin{split}
		\braket{n}{m}&=\bra{0}\frac{\op a^n}{\sqrt{n!}}\frac{(\adj{\op a})^n}{\sqrt{m!}}\ket{0}=\\
		&=\frac1{\sqrt{n!m!}}\bra{0}\op a^{n-1}\op a(\adj{\op a})^m\ket{0}=\\
		&=\frac1{\sqrt{n!m!}}\bra{0}\op a^{n-1}[\op a,(\adj{\op a})^m]\ket{0}=\\
		&=\frac{m}{\sqrt{n!m!}}\bra{0}\op a^{n-1}(\adj{\op a})^{m-1}\ket{0}=\dots
	\end{split}
	\label{eq:oscillatore-armonico-ortonormalita-autostati}
\end{equation}
diminuendo ad ogni successivo passo le potenze di $\op a$ e $\adj{\op a}$, fino ad esaurirle, perciò
\begin{itemize}
	\item se $n<m$, si giunge a un multiplo di $\bra{0}\op a^{m-n}\ket{0}$, che è nullo;
	\item se $n>m$, si giunge a un multiplo di $\bra{0}(\adj{\op a})^{n-m}\ket{0}$, che è ancora nullo;
	\item se $n=m$, entrambi gli operatori arrivano alla potenza nulla, chè è l'identità, perciò si arriva a
		\begin{equation}
			\frac{m!}{\sqrt{n!m!}}\braket{0}{0}=\frac{m!}{m!}\braket{0}{0}=\braket{0}{0}=1,
		\end{equation}
\end{itemize}
quindi $\braket{n}{m}=\delta_{nm}$.

Usando questo insieme come base di Hilbert per lo spazio degli stati, possiamo dunque rappresentare gli operatori incontrati finora come matrici.
Cominciamo dagli operatori di creazione e distruzione: risulta
\begin{multline}
	(\adj{\op a})_{nm}=\bra{n}\adj{\op a}\ket{m}=\frac1{\sqrt{m!}}\bra{n}\adj{\op a}(\adj{\op a})^m\ket{0}=\frac1{\sqrt{m!}}\bra{n}(\adj{\op a})^{m+1}\ket{0}=\\
	=\frac1{\sqrt{m!}}\bra{n}\sqrt{(m+1)!}\ket{m+1}=\sqrt{\frac{(m+1)!}{m!}}\delta_{n,m+1}=\sqrt{n}\delta_{n,m+1}.
	\label{eq:matrice-operatore-creazione-oscillatore}
\end{multline}
L'operatore di creazione è l'aggiunto di quest'ultimo quindi $(\op a)_{nm}=(\adj{\op a})_{mn}^*$.
Abbiamo dunque le matrici\footnote{Volendo essere rigorosi, il simbolo di uguaglianza è sbagliato in questo caso, dato che le matrici \emph{non sono} i rispettivi operatori, ma \emph{li rappresentano} in un determinato spazio, che qui è $\ell^2(\C)$. Per mantenere una notazione più semplice continueremo comunque con questa notazione.}
\begin{equation}
	\adj{\op a}=
	\begin{bmatrix}
		0		&0			&0			&0		&0		&\cdots\\
		1		&0			&0			&0		&0		&\cdots\\
		0		&\sqrt{2}	&0			&0		&0		&\cdots\\
		0		&0			&\sqrt{3}	&0		&0		&\cdots\\
		0		&0			&0			&2		&0		&\cdots\\
		\vdots	&\vdots		&\vdots		&\vdots	&\vdots	&\ddots
	\end{bmatrix}
	\qeq
	\op a=
	\begin{bmatrix}
		0		&1			&0			&0			&0			&\cdots\\
		0		&0			&\sqrt{2}	&0			&0			&\cdots\\
		0		&0			&0			&\sqrt{3}	&0			&\cdots\\
		0		&0			&0			&0			&2			&\cdots\\
		0		&0			&0			&0			&0			&\cdots\\
		\vdots	&\vdots		&\vdots		&\vdots		&\vdots		&\ddots
	\end{bmatrix}.
	\label{eq:matrici-operatori-scala-oscillatore}
\end{equation}
Per risalire alla posizione e al momento, invertiamo le \eqref{eq:operatore-distruzione} e \eqref{eq:operatore-creazione} ottenendo
\begin{equation}
	\op q=i\sqrt{\frac{\hbar}{2m\omega}}(\op a-\adj{\op a})\qeq\op p=\sqrt{\frac{m\hbar\omega}2}(\op a+\adj{\op a}).
\end{equation}
I due operatori sono allora rappresentati in $\ell^2(\C)$ dalle matrici
\begin{equation}
	\op q=i\sqrt{\frac{\hbar}{2m\omega}}
	\begin{bmatrix}
		0		&1			&0			&0			&0		&\cdots\\
		-1		&0			&\sqrt{2}	&0			&0		&\cdots\\
		0		&-\sqrt{2}	&0			&\sqrt{3}	&0		&\cdots\\
		0		&0			&-\sqrt{3}	&0			&2		&\cdots\\
		0		&0			&0			&-2			&0		&\cdots\\
		\vdots	&\vdots		&\vdots		&\vdots		&\vdots	&\ddots
	\end{bmatrix}
	\qeq
	\op p=\sqrt{\frac{m\hbar\omega}2}
	\begin{bmatrix}
		0		&1			&0			&0			&0		&\cdots\\
		1		&0			&\sqrt{2}	&0			&0		&\cdots\\
		0		&\sqrt{2}	&0			&\sqrt{3}	&0		&\cdots\\
		0		&0			&\sqrt{3}	&0			&2		&\cdots\\
		0		&0			&0			&2			&0		&\cdots\\
		\vdots	&\vdots		&\vdots		&\vdots		&\vdots	&\ddots
	\end{bmatrix}.
	\label{eq:matrici-posizione-impulso}
\end{equation}
È evidente che le matrici di $\op q$ e $\op p$ sono hermitiane, e verificano la relazione $[\op q,\op p]=i\hbar$, dove $i\hbar$ è inteso come multiplo della matrice identità.
È importante inoltre notare che queste due rappresentazioni matriciali degli operatori di posizione e impulso valgono sempre in $\ell^2(\C)$, non solo quando si studia l'oscillatore armonico, a patto ovviamente di rappresentare gli stati nella base degli autostati $\ket{n}$ come definiti in precedenza.

\section{Stati coerenti}
Matematicamente, gli \emph{stati coerenti} sono definiti come gli autostati ``destri'' dell'operatore di distruzione, cioè\footnote{L'attributo \emph{destri} attribuito agli autostati è importante in questo caso, perch\'e l'operatore di distruzione non è hermitiano, quindi se $\op a\ket{\alpha}=\alpha\ket{\alpha}$ non vale la relazione duale $\bra{\alpha}\op a=\alpha\bra{\alpha}$.}
\begin{equation}
	\op a\ket{\alpha}=\alpha\ket{\alpha}
	\label{eq:stato-coerente}
\end{equation}
per un $\alpha\in\C$.
Possiamo sviluppare $\ket{\alpha}$ nella base di autostati $\{\ket{n}\}_{n=0}^{+\infty}$ dell'hamiltoniano, ottenendo una serie $\sum_{n=0}^{+\infty}a_n\ket{n}$ con i coefficienti $\braket{n}{\alpha}\defeq a_n$ da determinare.
Dalla \eqref{eq:stato-coerente} otteniamo la relazione
\begin{equation}
	\op a\sum_{n=0}^{+\infty}a_n\ket{n}=\sum_{n=0}^{+\infty}a_n\op a\ket{n}=\sum_{n=0}^{+\infty}a_n\sqrt{n}\ket{n-1}.
\end{equation}
Il primo addendo è nullo per $\sqrt{n}$, quindi possiamo far iniziare la sommatoria da $n=1$: traslando poi gli indici con $m=n-1$ abbiamo
\begin{equation}
	\sum_{n=0}^{+\infty}a_n\sqrt{n}\ket{n-1}=\sum_{n=1}^{+\infty}a_n\sqrt{n}\ket{n-1}=\sum_{m=0}^{+\infty}a_{m+1}\sqrt{m+1}\ket{m}.
\end{equation}
Allo stesso modo di prima sviluppiamo anche $\alpha\ket{\alpha}=\sum_{m=0}^{+\infty}\alpha a_m\ket{m}$ e uguagliando ogni addendo con il risultato precedente otteniamo\footnote{Un risultato uguale si ottiene moltiplicando la \eqref{eq:stato-coerente} a sinistra per $\bra{n}$, ottenendo $\bra{n}\op a\ket{\alpha}=\alpha\braket{n}{\alpha}$ e introducendo la risoluzione dell'identità $\sum_{m=0}^{+\infty}\ket{m}\bra{m}=1$, in pratica sviluppando nella base $\ket{m}$, come prima, dato che è un sistema ortonormale completo.}
\begin{equation}
	a_{m+1}=\frac{\alpha}{\sqrt{m+1}}a_m.
	\label{eq:ricorrenza-stati-coerenti}
\end{equation}
per $m\in\N_0$.
Per determinarli completamente ci basta dunque fissare il primo valore, dunque poniamo $a_0=1$ e troviamo $a_1=\alpha$, $a_2=\frac{\alpha^2}{\sqrt{2}}$, $a_3=\frac{\alpha^3}{\sqrt{6}}$ e cos\`i via.
Completiamo dimostrando per induzione: se $a_k=\frac{\alpha^k}{\sqrt{k!}}$, allora il coefficiente di posto $k+1$ è
\begin{equation}
	a_{k+1}=\frac{\alpha}{\sqrt{k+1}}a_k=\frac{\alpha}{\sqrt{k+1}}\frac{\alpha^k}{\sqrt{k!}}=\frac{\alpha^{k+1}}{\sqrt{(k+1)!}},
\end{equation}
dunque lo stato coerente si scrive come
\begin{equation}
	\ket{\alpha}=\sum_{n=0}^{+\infty}\frac{\alpha^n}{\sqrt{n!}}\ket{n}.
\end{equation}
Resta soltanto da normalizzarli: dato che
\begin{multline}
	\braket{\alpha}{\alpha}=\sum_{m=0}^{+\infty}\frac{(\alpha^*)^m}{\sqrt{m!}}\bra{m}\sum_{n=0}^{+\infty}\frac{\alpha^n}{\sqrt{n!}}\ket{n}=
	\sum_{m=0}^{+\infty}\frac{(\alpha^*)^m}{\sqrt{m!}}\sum_{n=0}^{+\infty}\frac{\alpha^n}{\sqrt{n!}}\delta_{nm}=\\=
	\sum_{m=0}^{+\infty}\frac{(\alpha^*)^m}{\sqrt{m!}}\frac{\alpha^m}{\sqrt{m!}}=
	\sum_{m=0}^{+\infty}\frac{(\alpha^*\alpha)^m}{m!}=
	\sum_{m=0}^{+\infty}\frac{\abs{\alpha}^{2m}}{m!}=
	e^{\abs{\alpha}^2}
	\label{eq:norma-stato-coerente}
\end{multline}
possiamo rappresentare gli stati coerenti, ora di norma unitaria, con
\begin{equation}
	\ket{\alpha}=\exp\Big(\!\!-\frac{\abs{\alpha}^2}2\Big)\sum_{n=0}^{+\infty}\frac{\alpha^n}{\sqrt{n!}}\ket{n}.
	\label{eq:stato-coerente-normalizzato}
\end{equation}
In un'ulteriore semplificazione, possiamo sostituire $\ket{n}$ con la \eqref{eq:sonc-oscillatore-armonico}, con la quale otteniamo
\begin{equation}
	\ket{\alpha}=\sum_{n=0}^{+\infty}\frac{\alpha^n}{\sqrt{n!}}\ket{n}=\sum_{n=0}^{+\infty}\frac{\alpha^n}{\sqrt{n!}}\frac{(\adj{\op a})^n}{\sqrt{n!}}\ket{0}=\sum_{n=0}^{+\infty}\frac{(\alpha\adj{\op a})^n}{n!}\ket{0}=\exp\big(\alpha\adj{\op a}\big)\ket{0}.
	\label{eq:stato-coerente-exp}
\end{equation}
Notiamo che lo stato fondamentale è anch'esso uno stato coerente, corrispondente ad $\alpha=0$.

\section{Soluzioni dell'equazione di Schr\"odinger}
In una dimensione, con un potenziale della forma $V(q)=\frac12m\omega^2q^2$, l'equazione di Schr\"odinger per la funzione d'onda $\psi$ nella base della posizione è
\begin{equation}
	\frac{\hbar^2}{2m}\psi''(q)+\bigg(E-\frac12m\omega^2 q^2\bigg)\psi(q)=0.
	\label{eq:schrodinger-oscillatore-1d}
\end{equation}
In questa forma però risulta difficile da risolvere, anche perch\'e non conosciamo il valore di $E$.
Per evitarla usiamo, come abbiamo già fatto per ricavare lo spettro dell'hamiltoniano, gli operatori di creazione e distruzione.
Nella base delle coordinate, abbiamo per uno stato $\ket{\psi}$ (con la funzione d'onda $\psi(q)=\braket{q}{\psi}$)
\begin{multline}
	\bra{q}\op a\ket{\psi}=\bra{q}\frac{\op p-im\omega\op q}{\sqrt{2m\hbar\omega}}\ket{\psi}=\frac1{\sqrt{2m\hbar\omega}}(\bra{q}\op p\ket{\psi}-im\omega\bra{q}\op q\ket{\psi})=\\
	=\frac1{\sqrt{2m\hbar\omega}}\bigg[-i\hbar\drv{\psi}{q}(q)-im\omega q\psi(q)\bigg]
\end{multline}
Possiamo dunque rappresentare $\op a$ come l'operatore in $\leb[2]{\R}$
\begin{equation}
	f(q)\mapsto-\frac{i}{\sqrt{2}}\bigg[\sqrt{\frac{\hbar}{m\omega}}\drv{f}{q}(q)+\sqrt{\frac{m\omega}{\hbar}}qf(q)\bigg].
\end{equation}
Possiamo semplificare il tutto con un cambio di variabili: poniamo $y=\big(\frac{m\omega}{\hbar})^{-\frac12}q$, da cui $\drv{}{q}=\drv{y}{q}\drv{}{y}=\big(\frac{m\omega}{\hbar})^{-\frac12}\drv{}{y}$, ottenendo
\begin{equation}
	\op a=-\frac{i}{\sqrt{2}}\bigg(\drv{}{y}+y\bigg)
\end{equation}
ricordando sempre che $y$ (anche come variabile della funzione d'onda) è il riscalamento di $x$ definito poco fa.
Detto ciò, sappiamo che lo stato fondamentale soddisfa l'equazione $\op a\ket{0}=0$, da cui otteniamo l'equazione differenziale per la funzione d'onda $\psi_0$
\begin{equation}
	-\frac{i}{\sqrt{2}}\bigg[\drv{\psi_0}{y}(y)+y\psi_0(y)\bigg]=0\quad\longrightarrow\quad \drv{\psi_0}{y}(y)=-y\psi_0(y)
\end{equation}
che ha come soluzione (accettabile) $\psi_0(y)=A\exp(-y^2/2)$.
Normalizzando la funzione troviamo
\begin{equation}
	1=\norm{\psi_0}_2^2=\int_{-\infty}^{+\infty}\abs{\psi_0(y)}^2\,\dd y=A^2\int_{-\infty}^{+\infty}e^{-\frac{y^2}2}\,\dd y=A^2\sqrt{\pi}
\end{equation}
dunque $\psi_0(y)=\pi^{-\frac14}\exp(-y^2/2)$.

Analogamente, possiamo rappresentare in $\leb[2]{\R}$ nella base della posizione (e nella variabile $y$) anche l'operatore di creazione $\adj{\op a}$ come $-\frac{i}{\sqrt{2}}\big(\drv{}{y}-y\big)$.
Possiamo dunque determinare le funzioni d'onda degli stati più energetici dalla \eqref{eq:sonc-oscillatore-armonico}, trovando\footnote{Notare il fattore $(-i)^n$ che moltiplica i vari membri: abbiamo ridefinito, in un certo senso, l'operatore di creazione come $i\adj{\op a}$. Questo, almeno per l'oscillatore armonico unidimensionale che stiamo studiando, non cambia il significato fisico delle soluzioni: alla fine la funzione d'onda apparirà moltiplicata per $i^n$ (per compensare l'altro fattore), ma essa rimane comunque normalizzata e la densità di probabilità associata non cambia.}
\begin{equation}
	\begin{split}
		\psi_n(y)&=\frac{(-i)^n}{\sqrt{n!}}\bigg[-\frac{i}{\sqrt{2}}\bigg(\drv{}{y}-y\bigg)\bigg]^n\psi_0(y)=\\
		&=\frac1{\sqrt{2^nn!}}\bigg(y-\drv{}{y}\bigg)^n\psi_0(y)=\\
		&=\frac1{\sqrt{2^nn!\sqrt{\pi}}}\bigg(y-\drv{}{y}\bigg)^ne^{-\frac{y^2}2}=\\
		&=\frac1{\sqrt{2^nn!\sqrt{\pi}}}e^{-\frac{y^2}2}e^{\frac{y^2}2}\bigg(y-\drv{}{y}\bigg)^ne^{-\frac{y^2}2}=\\
		&=\frac1{\sqrt{2^nn!\sqrt{\pi}}}e^{-\frac{y^2}2}H_n(y)=h_n(y)
	\end{split}
	\label{eq:autofunzioni-oscillatore-hermite}
\end{equation}
dove $H_n(y)$ è il \emph{polinomio di Hermite} di grado $n$, definito appunto da
\begin{equation}
	H_n(y)=e^{\frac{y^2}2}\bigg(y-\drv{}{y}\bigg)^ne^{-\frac{y^2}2}
	\label{eq:def-polinomio-hermite}
\end{equation}
mentre $h_n(y)$ è la \emph{funzione di Hermite} di grado $n$.
Nella figura \ref{fig:oscillatore-probabilita} sono mostrate le densità di probabilità nella variabile $y$, per alcuni dei primi autostati.
\begin{figure}
	\tikzsetnextfilename{oscillatore-probabilita}
	\centering
	\begin{tikzpicture}
		\begin{axis}[
				standard,
				height=.5\linewidth, width=\linewidth,
				enlargelimits,
				xlabel=$y$,
				xmin=-5, xmax=5,
				ymin=0, ymax=.4,
				ytick=\empty
			]
			\addplot[thick, samples=1000, domain=-5:5] function {(pi)**(-1)*exp(-x**2)}; % n=0
			\addplot[thick, samples=1000, densely dashed, domain=-5:5] function {(2*sqrt(pi))**(-1)*(2*x)**2*exp(-x**2)}; % n=1
			%\addplot[thick, samples=1000, black!40!white, domain=-6:6] function {(2**2*2!*sqrt(pi))**(-1)*(4*x**2-2)**2*exp(-x**2)}; % n=2
			\addplot[thick, samples=1000, densely dotted, domain=-5:5] function {(2**5*5!*sqrt(pi))**(-1)*(32*x**5-160*x**3+120*x)**2*exp(-x**2)}; % n=5
			%\addplot[thick, samples=1000, black!80!white, domain=-6:6] function {(2**10*10!*sqrt(pi))**(-1)*(1024*x**10-23040*x**8+161280*x**6-403200*x**4+302400*x**2-30240)**2*exp(-x**2)}; % n=10
			\legend{$n=0$, $n=1$, $n=5$}
		\end{axis}
	\end{tikzpicture}
	\caption{Distribuzioni di probabilità, in unità $y=\big(\frac{m\omega}{\hbar}\big)^{-\frac12}q$, dell'oscillatore armonico per alcuni dei primi autostati $\ket{n}$ dell'hamiltoniano, ossia gli stati eccitati del sistema per energie $E_n=\hbar\omega\big(n+\frac12\big)$.}
	\label{fig:oscillatore-probabilita}
\end{figure}


\paragraph{Funzioni di Hermite}
Approfondiamo per un momento la trattazione di queste funzioni speciali.
Dei polinomi di Hermite esistono varie definizioni: una delle più importanti è data tramite la funzione generatrice, nella quale i polinomi appaiono nello sviluppo in serie
\begin{equation}
	\exp(2ty-t^2)=\sum_{n=0}^{+\infty}\frac{H_n(y)}{n!}t^n.
	\label{eq:funzione-generatrice-polinomi-hermite}
\end{equation}
Essi soddisfano le relazioni di ricorrenza
\begin{equation}
	\begin{gathered}
		H_{n+1}(y)=2yH_n(y)-2nH_{n-1}(y)\\
		H_n'(y)=2nH_{n-1}(y).
	\end{gathered}
	\label{eq:ricorrenza-polinomi-hermite}
\end{equation}
che si possono ricavare differenziando i due membri dell'equazione precedente rispetto a $t$ o $y$ (rispettivamente) e uguagliando ogni addendo delle somme.
Proviamo a verificare la \eqref{eq:def-polinomio-hermite} sostituendola nella serie di potenze: otteniamo una funzione
\begin{equation}
	\begin{split}
		f(t,y)&=\sum_{n=0}^{+\infty}e^{\frac{y^2}2}\bigg(y-\drv{}{y}\bigg)^ne^{-\frac{y^2}2}\frac{t^n}{n!}=\\
		&=\sum_{n=0}^{+\infty}\frac1{n!}e^{\frac{y^2}2}\bigg[t\bigg(y-\drv{}{y}\bigg)\bigg]^ne^{-\frac{y^2}2}=\\
		&=\exp\bigg(\frac{y^2}2\bigg)\exp\bigg[t\bigg(y-\drv{}{y}\bigg)\bigg]\exp\bigg(-\frac{y^2}2\bigg).
	\end{split}
\end{equation}
Non possiamo scambiare liberamente l'ordine degli esponenziali, dato che alcuni sono operatori di $\leb[2]{\R}$ e potrebbero non commutare.
Cerchiamo però di scindere l'operatore che si trova in mezzo in due esponenziali distinti: innanzitutto abbiamo $\big[y,-\drv{}{y}\big]=1$, dato che
\begin{equation}
	\bigg[y,\drv{}{y}\bigg]f(y)=y\drv{}{y}f(y)-\drv{}{y}\big[yf(y)\big]=y\drv{}{y}f(y)-f(y)-y\drv{}{y}f(y)=-f(y).
\end{equation}
Inoltre, con la formula di Baker-Campbell-Hausdorff, risulta
\begin{equation}
	\exp(ty)\exp\bigg(-t\drv{}{y}\bigg)=\exp\bigg\{t\bigg(y-\drv{}{y}\bigg)+\frac{t^2}2\big[y,-\drv{}{y}\big]+\cdots\bigg\}=\exp\bigg[t\bigg(y-\drv{}{y}\bigg)+\frac{t^2}2\bigg]
\end{equation}
dato che i termini di ordine superiore al secondo contengono tutti commutatori di $\big[y,-\drv{}{y}\big]$ con altri termini, ma quest'ultimo è uno scalare quindi commuta con tutti gli operatori; di conseguenza, questi termini sono tutti nulli.
Allora risulta, invertendo la formula,
\begin{equation}
	\exp\bigg[t\bigg(y-\drv{}{y}\bigg)\bigg]=\exp\bigg(-\frac{t^2}2\bigg)\exp(ty)\exp\bigg(t\drv{}{y}\bigg).
\end{equation}
L'azione di questi esponenziali, in particolare l'ultimo, su una funzione $f$ è\footnote{Assumendo che $f$ sia analitica, in modo che coincida con il suo sviluppo di Taylor.}
\begin{equation}
	\exp\bigg(-t\drv{}{y}\bigg)f(y)=\bigg[\sum_{n=0}^{+\infty}\frac{(-t)^n}{n!}\frac{\dd^n}{\dd y^n}\bigg]f(y)=\sum_{n=0}^{+\infty}\frac{(-t)^nf^{(n)}(y)}{n!}
\end{equation}
che è lo sviluppo in serie di Taylor di $f(y-t)$ centrato in $y$.
Di conseguenza
\begin{equation}
	\begin{split}
		f(y,t)&=\exp\bigg(\frac{y^2}2\bigg)\exp\bigg[t\bigg(y-\drv{}{y}\bigg)\bigg]\exp\bigg(-\frac{y^2}2\bigg)=\\
		&=\exp\bigg(\frac{y^2}2\bigg)\exp\bigg(\frac{t^2}2\bigg)\exp(ty)\exp\bigg(-t\drv{}{y}\bigg)\exp\bigg(-\frac{y^2}2\bigg)=\\
		&=\exp\bigg(\frac{y^2}2\bigg)\exp\bigg(\frac{t^2}2\bigg)\exp(ty)\exp\bigg[-\frac{(y-t)^2}2\bigg]=\\
		&=\exp\bigg[\frac{y^2}2+\frac{t^2}2+ty-\frac{(y-t)^2}2\bigg]=\\
		&=\exp(2ty-t^2)
	\end{split}
\end{equation}
che è proprio la funzione generatrice dei polinomi di Hermite.
La proprietà fondamentale dei polinomi è che formano un insieme ortogonale nello spazio $\leb[2]{\R,e^{-x^2}\dd x}$, ossia
\begin{equation}
	\int_{\R}e^{-x^2}H_n(x)H_m(x)\,\dd x=0\quad\text{se }n\ne m.
	\label{eq:ortogonalita-polinomi-hermite}
\end{equation}
Di conseguenza le funzioni di Hermite $h_n$, per come sono definite, formano invece un insieme \emph{ortonormale} in $\leb[2]{\R}$.\footnote{L'insieme delle funzioni di Hermite è anche \emph{completo}, dunque può essere preso come base di $\leb[2]{\R}$. Il punto chiave per dimostrarlo è che queste funzioni sono anche autofunzioni della trasformata di Fourier, che è un'operatore unitario.}

\paragraph{Soluzione asintotica}
Possiamo costruire una soluzione come di seguito: per grandi valori di $y$, ipotizziamo che l'autostato di ordine $n$ abbia una funzione d'onda asintotica a $y^n\exp(-y^2/2)$.
Dato che $\psi_n\to 0$ per $\abs{y}\to+\infty$, possiamo approssimare l'equazione di Schr\"odinger da $-\psi_n''(y)+y^2\psi_n(y)=(2n+1)\psi_n(y)$ a $-\psi_n''(y)+y^2\psi_n(y)=0$, e nel momento in cui trascuro il secondo membro l'equazione non dipende più da $n$.
Cominciamo quindi a costruire la soluzione: partiamo da una funzione prova del tipo $\psi_n(y)=\exp(-\beta y^k)$.
Risulta $\psi_n'(y)=-\beta ky^{k-1}\exp(-\beta y^k)=-\beta ky^{k-1}\psi_n(y)$ e $\psi_n''(y)=-\beta k(k-1)y^{k-2}\psi_n(y)+\beta ky^{k-1}\psi_n'(y)$ dunque riscriviamo l'equazione di Schrodinger come
\begin{equation}
	-\psi_n''(y)+y^2\psi_n(y)=e^{-\beta y^k}\big[-\beta^2k^2y^{2k-2}+\beta k(k-1)y^{k-2}+y^2\big]\sim e^{-\beta y^k}(-\beta^2k^2y^{2k-2}+y^2)
\end{equation}
e affinch\'e l'espressione tenda a zero per $\abs{y}\to+\infty$ deve essere uguali le potenze di $y$, ossia $k=2$.
A questo punto troviamo
\begin{equation}
	-\psi_n''(y)+y^2\psi_n(y)\sim e^{-\beta y^2}(1-4\beta^2)y^2
\end{equation}
e con gli stessi vincoli ($\psi_n\to 0$) troviamo che $1-4\beta^2=0$, ossia $\beta=\frac12$.
La funzione $\exp(-y^2/2)$ è una buona approssimazione iniziale, per grandi valori di $\abs{y}$, della funzione d'onda del sistema.
Possiamo fare di meglio: aggiungiamo un termine polinomiale, con $\psi_n(y)=y^a\exp(-y^2/2)$.
Otteniamo $\psi_n'(y)=(ay^{a-1}-y^{a+1})\exp(-y^2/2)$ e $\psi_n''(y)=\big[a(a-1)y^{a-2}-(2a+1)y^a+y^{a+2}\big]\exp(-y^2/2)$, perciò
\begin{equation}
	\begin{aligned}
		\big[-a(a-1)y^{a-2}+(2a+1)y^a-y^{a+2}\big]e^{-\frac{y^2}2}+y^ae^{-\frac{y^2}2}y^2&=(2n+1)y^ae^{-\frac{y^2}2}\\
		\big[-a(a-1)y^{a-2}+(2a+1)y^a\big]e^{-\frac{y^2}2}&=(2n+1)y^ae^{-\frac{y^2}2}
	\end{aligned}
\end{equation}
e mantenendo solo i termini di ordine maggiore risulta
\begin{equation}
	(2a+1)y^ae^{-\frac{y^2}2}=(2n+1)y^ae^{-\frac{y^2}2}
\end{equation}
per cui evidentemente $a=n$, perciò la soluzione è la funzione $\psi_n(y)=y^n\exp(-y^2/2)$.
Potremmo ancora proseguire aggiungendo altri termini ``correttivi'', ma ci fermiamo qui.

\section{Approssimazione classica}
Vediamo ora in che modo possiamo riottenere la distribuzione di probabilità classica da quanto visto finora.
In generale, data un'hamiltoniana (classica) $H=\frac1{2m}p^2+V(x)$, l'energia è conservata poich\'e non dipende esplicitamente dal tempo, perciò poniamo $H=E$.
Fissiamo un punto $x'$: la densità di probabilità di misurare la posizione della particella in $x'$ è
\begin{equation}
	\rho(x')=\frac1{t_2-t_1}\int_{t_1}^{t_2}\delta\big(x(t)-x'\big)\,\dd t,
\end{equation}
da normalizzare in modo che
\begin{equation}
	\int_{-\infty}^{+\infty}\rho(x')\,\dd x'=\int_{-\infty}^{+\infty}\frac1{t_2-t_1}\int_{t_1}^{t_2}\delta\big(x(t)-x'\big)\,\dd t\,\dd x'=1.
\end{equation}
Possiamo inoltre scrivere $\dd t=\dd x/\abs{v(x)}$, e data la conservazione dell'energia ricaviamo la velocità in funzione della posizione come
\begin{equation}
	\abs{v(x)}=\sqrt{\frac2{m}[E-V(x)]},
\end{equation}
nelle regioni in cui $V(x)\le E$.
Possiamo prendere $v(x)\ge 0$ se ci restringiamo a studiare solo un semiperiodo, quindi per $t_2-t_1=\frac{T}2$.
\begin{equation}
	\rho(x')=\frac1{t_2-t_1}\int_{x(t_1)}^{x(t_2)}\frac{\delta(x-x')}{\sqrt{\frac2{m}[E-V(x)]}}\Theta\big(E-V(x)\big)\,\dd x.
\end{equation}
Il termine $\Theta\big(E-V(x)\big)$, nullo per $E<V(x)$, assicura che la probabilità sia nulla nelle regioni di spazio in cui non si ha $V(x)\le E$.
Risulta quindi
\begin{equation}
	\rho(x')=\frac2{T}\frac1{\sqrt{\frac2{m}[E-V(x')]}}=\frac2{T}\frac1{\abs{v(x')}}.
\end{equation}
Notiamo che è inversamente proporzionale alla velocità della particella: è come ci si aspetta, poich\'e più veloce è la particella più difficile è trovarla nel punto in cui passa.

Nel caso dell'oscillatore armonico, dunque, abbiamo
\begin{equation}
	\rho(x')=\frac{\omega}{\pi}\frac1{\sqrt{\frac2{m}\big(E-\frac12m\omega^2x'^2\big)}}=\frac1{\pi}\sqrt{\frac{m\omega^2}{2E}}\frac1{\sqrt{1-\frac{m\omega^2x'^2}{2E}}}.
	\label{eq:pdf-classica-oscillatore-armonico}
\end{equation}
Cambiando la variabile con $y=\big(\frac{m\omega^2}{2E}\big)^{\frac12}x'$ abbiamo poi
\begin{equation}
	\rho(x')\,\dd x'=\frac1{\pi}\frac1{\sqrt{1-y^2}}\,\dd y.
\end{equation}
Come già visto nel caso generale, la probabilità è minima in $y=0$, dove la velocità è massima.
Per $E\to 0$, la funzione converge a $\delta(x')$, coerentemente con il fatto che la particella può essere trovata solamente nel punto $x'$.

\paragraph{Quantizzazione nello spazio delle fasi}
Nello spazio delle fasi, la traiettoria descritta dall'oscillatore armonico è un'ellisse, con equazione data dall'hamiltoniana, ossia
\begin{equation}
	\frac{p^2}{2mE}+\frac{x^2}{\frac{2E}{m\omega^2}}=1.
\end{equation}
Essa racchiude un'area pari a
\begin{equation}
	A=\pi \sqrt{2mE}\sqrt{\frac{2E}{m\omega^2}}=\frac{2\pi}{\omega}E
\end{equation}
e sostituendo $E$ con gli autovalori trovati di $\op H$ troviamo il curioso risultato
\begin{equation}
	A=\frac{2\pi}{\omega}\Big(n+\frac12\Big)\hbar\omega=\Big(n+\frac12\Big)\hbar
\end{equation}
che mostra come l'area (in generale, il volume) nello spazio delle fasi sia quantizzata.
Questa fu una delle prima idee per spiegare i fenomeni quantistici, nota come \emph{quantizzazione di Bohr-Sommerfield}.

\section{Oscillatore armonico multidimensionale}
\label{sec:oscillatore-armonico-multidimensionale}
In più dimensioni, l'operatore hamiltoniano dell'oscillatore armonico è analogo alla versione unidimensionale, ma con gli operatori posizione e impulso vettoriali anzich\'e scalari, ossia
\begin{equation}
	\op H=\frac1{2m}(\op{\vec p}^2+m^2\omega^2\op{\vec q}^2)=\frac1{2m}\sum_{i=1}^d(\op p_i^2+m^2\omega^2\op q_i^2)
	\label{eq:hamiltoniano-oscillatore-armonico-multidim}
\end{equation}
dove $d$ è la dimensione del sistema.
Possiamo definire l'hamiltoniano della coordinata $i$-esima $\op H_i=\frac1{2m}(\op p_i^2+m^2\omega^2\op q_i^2)$ cos\`i che $\op H=\sum_{i=1}^d\op H_i$.
Dal commutatore $[\op q_i,\op p_j]=i\hbar\delta_{ij}$ è evidente che gli hamiltoniani di differenti coordinate sono compatibili.

Chiamiamo $\ket{n_1,\dots,n_d}$ lo stato tale per cui $\op H_j\ket{n_1,\dots,n_d}=\hbar\omega(n_j+\frac12)$: sono gli stati in cui i vari $\op H_i$ sono diagonali.
Inoltre
\begin{equation}
	\op H\ket{n_1,\dots,n_d}=\sum_{i=1}^d\op H_i\ket{n_1,\dots,n_d}=\sum_{i=1}^d\Big(n_i+\frac12\Big)\hbar\omega\ket{n_1,\dots,n_d}=\hbar\omega\Big(\!N+\frac{d}2\Big)\ket{n_1,\dots,n_d}
\end{equation}
dove $N\defeq\sum_{i=1}^dn_i$.

Lo stato fondamentale ha $N=0$, e dato che ogni $n_i$ è non negativo l'unico modo per ottenere zero è che $n_i=0$ per ogni $i\in\{1,\dots,d\}$.
Dato che ognuno di questi numeri è univocamente determinato, lo stato fondamentale non è degenere.
Gli stati di energia maggiore però non lo sono: se già $N=1$ si hanno $d$ possibili combinazioni che danno $N=1$.
In generale, il grado di degenerazione dello stato $N$ è il numero di modi in cui si può ottenere $N$ con $d$ interi positivi o nulli, ossia
\begin{equation}
	\frac{d(N+d-1)!}{d!N!}=\frac{(N+d-1)!}{(d-1)!N!}=\binom{N+d-1}{N}.
\end{equation}
Notiamo anche qui che se $N=0$ si ha sempre 1.

Poich\'e il potenziale $V(\op{\vec q})$ è centrale, dato che dipende dalla norma al quadrato della posizione, l'hamiltoniano è invariante per rotazioni, le trasformazioni del gruppo $SO(n)$.

Definiamo il momento angolare orbitale, in analogia con la sua controparte classica, come
\begin{equation}
	\op{\vec L}=\op{\vec q}\times\op{\vec p}
\end{equation}
o per ciascuna componente $\op L_k=\epsilon_{ijk}\op q_i\op p_j$.
Nella definizione non troviamo problemi di commutazione in quanto posizione e impulso si riferiscono a coordinate diverse.
In questo caso, dato che l'hamiltoniano è invariante per rotazioni, si ha $[\op H,\op{\vec L}]=0$.
Definiamo per ciascuna coordinata l'operatore, simile a quello di distruzione, $\op\eta_i=\frac1{\sqrt{2m\hbar\omega}}(\op p_i-im\omega\op q_i)$.
Con questa definizione risulta $\op H=\sum_{i=1}^d\big(\adj{\op\eta_i}\op\eta_i+\frac12\big)$.
Possiamo inoltre ricavare gli operatori di posizione e impulso da $\op\eta_i$ e $\adj{\op\eta_i}$ in modo analogo al caso unidimensionale.
Notiamo subito che tutti questi operatori commutano, anche con i loro aggiunti, cioè $[\op\eta_i,\op\eta_j]=[\adj{\op\eta_i},\adj{\op\eta_j}]=[\adj{\op\eta_i},\op\eta_j]=0$.
Poniamoci per semplicità in un sistema bidimensionale ($d=2$): il momento angolare ha una sola componente $\op L$ che vale
\begin{equation}
	\begin{split}
		\op L&=-i\sqrt{\frac{m\hbar\omega}2}\sqrt{\frac{\hbar}{2m\omega}}\big[(\adj{\op\eta_1}-\op\eta_1)(\adj{\op\eta_2}+\op\eta_2)-(\adj{\op\eta_2}-\op\eta_2)(\adj{\op\eta_1}+\op\eta_1)\big]=\\
		&=-\frac{i\hbar}2(\adj{\op\eta_1}\op\eta_2-\op\eta_1\adj{\op\eta_2}-\adj{\op\eta_2}\op\eta_1+\op\eta_2\adj{\op\eta_1})=\\
		&=i\hbar(\op\eta_1\adj{\op\eta_2}-\adj{\op\eta_1}\op\eta_2)
	\end{split}
\end{equation}
perciò commuta con l'hamiltoniano come
\begin{equation}
	[\op H,\op L]=i\hbar[\op H,\op\eta_1\adj{\op\eta_2}-\adj{\op\eta_1}\op\eta_2]=i\hbar[\op H,\op\eta_1\adj{\op\eta_2}]-i\hbar[\op H,\adj{\op\eta_1}\op\eta_2]=0
\end{equation}
come si può verificare.

Nello stato fondamentale $\ket{0,0}$, il momento angolare è $\op L\ket{0,0}=i\hbar(\adj{\op\eta_2}\op\eta_1-\adj{\op\eta_1}\op\eta_2)\ket{0,0}=0$, cioè è nullo.
Il primo stato eccitato è invece una combinazione lineare di $\ket{1,0}$ e $\ket{0,1}$: prendiamo questi due stati come base e applichiamo $\op L$ ad essi, ottenendo
\begin{equation}
	\op L\ket{1,0}=i\hbar(\op\eta_1\adj{\op\eta_2}\ket{1,0}-\adj{\op\eta_1}\op\eta_2\ket{1,0})=i\hbar\op\eta_1\ket{1,1}=i\hbar\ket{0,1}
\end{equation}
e analogamente $\op L\ket{0,1}=-i\hbar\ket{1,0}$.
Possiamo quindi rappresentare, in questa base, l'operatore con la matrice $i\hbar\begin{psmallmatrix}0&-1\\1&0\end{psmallmatrix}$, che ha $\hbar,-\hbar$ come autovalori a cui corrispondono gli autostati, rispettivamente, $\ket{1,0}+i\ket{0,1}$ e $\ket{1,0}-i\ket{0,1}$: gli autovalori di $\op L$ nello stato con $N=1$ sono dunque $\pm\hbar$.

Definiamo ora gli operatori
\begin{equation}
	\op a_-=\frac{\op\eta_1+i\op\eta_2}{\sqrt{2}},\quad\op a_+=\frac{\op\eta_1-i\op\eta_2}{\sqrt{2}},\quad\op N=\adj{\op a_+}\op a_+-\adj{\op a_-}\op a_-.
\end{equation}
Risulta che $\adj{\op a_+}\ket{0,0}=\frac1{\sqrt{2}}(\ket{1,0}+i\ket{0,1})$, ossia $\adj{\op a_+}$ ``crea'' un quanto con momento angolare $\hbar$, mentre $\adj{\op a_-}$ ne crea uno con momento angolare $-\hbar$; $\op a_+$ e $\op a_-$ invece li ``distruggono''.
Come per il caso unidimensionale, da questi operatori possiamo ritrovare l'hamiltoniano con $\op H=\adj{\op a_+}\op a_++\adj{\op a_-}\op a_-$.
Poich\'e
\begin{equation}
	[\op H,\adj{\op a_+}\op a_-]=0,\quad [\op H,\op L]=0,\quad [\op L,\adj{\op a_+}\op a_-]=2\hbar\adj{\op a_+}\op a_-,
\end{equation}
siamo in presenza di una degenerazione.

Per ``creare'' $N$ quanti, possiamo scegliere di crearne un certo numero con $\adj{\op a_+}$ e i restanti con $\adj{\op a_-}$: scelto $k\in\{0,\dots,N\}$, otteniamo dunque uno stato $(\adj{\op a_+})^k(\adj{\op a_-})^{N-k}\ket{0,0}$.
Per $k=N$, ad esempio, si ha lo stato $(\adj{\op a_+})^N\ket{0,0}$ che ha come autovalore dell'energia $\hbar\omega(N+1)$ e momento angolare orbitale $N\hbar$.
Ci possiamo chiedere a questo punto se il momento angolare può essere reso grande a piacere.
La risposta è no, perch\'e risulta
\begin{equation}
	\adj{\op a_+}\op a_-(\adj{\op a_+})^N\ket{0,0}=\adj{\op a_+}\big[\op a_-(\adj{\op a_+})^N-(\adj{\op a_+})^N\op a_+(\adj{\op a_+})^N\op a_-\big]\ket{0,0}=\adj{\op a_+}[\op a_-,(\adj{\op a_-})^N]\ket{0,0}=0
\end{equation}
e, diversamente, $\adj{\op a_-}\op a_+(\adj{\op a_-})^N\ket{0,0}=\adj{\op a_-}[\op a_+,(\adj{\op a_+})^N]\ket{0,0}=N\adj{\op a_-}(\adj{\op a_+})^{N-1}\ket{0,0}$.
Dallo stato con momento angolare $N$ si scende dunque di due in due con gli autovalori $N-2$, $N-4$\dots fino a quando?
Giunti alla situazione simmetrica, nell'altro verso di rotazione, troviamo che $-N$ è il minimo, infatti $\adj{\op a_-}\op a_+(\adj{\op a_-})^N\ket{0,0}=\adj{\op a_-}[\op a_+,(\adj{\op a_-})^N]\ket{0,0}=0$ e il processo si ferma qui.

