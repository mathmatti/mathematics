\chapter{Evoluzione temporale}
Poich\'e finora ci siamo interessati degli stati di equilibrio dei sistemi  e delle loro energie, ci è servito studiare soltanto una descrizione indipendente dal tempo.
Per analizzare i fenomeni dinamici e variabili nel tempo, però, ci occorre capire come gli stati dei sistemi quantistici evolvono nel tempo: dato uno stato $\ket{\psi(0)}$ noto ad un certo istante iniziale di tempo, come determinare la sua evoluzione $\ket{\psi(t)}$ ad un generico istante?
A questo risponde l'ultimo postulato della teoria quantistica che incontriamo.
\begin{postulato}
    L'evoluzione nel tempo di uno stato $\ket{\psi}$ è unitaria ed è dettata dall'equazione
    \begin{equation}
        i\hbar\drp{}{t}\ket{\psi}=\op H\ket{\psi}.
        \label{eq:schroedinger-dipendente-dal-tempo}
    \end{equation}
\end{postulato}
Questa equazione è nota come \emph{equazione di Schr\"odinger dipendente dal tempo}, e in effetti vedremo che per gli autostati dell'hamiltoniano si riduce all'equazione di Schr\"odinger già vista in precedenza (che è \emph{indipendente} dal tempo).

Partiamo però da un caso più semplice, che è quello in cui l'operatore hamiltoniano non dipende esplicitamente dal tempo.
Nella meccanica classica l'evoluzione temporale è descritta dall'evoluzione delle coordinate canoniche nello spazio delle fasi secondo le equazioni di Hamilton; in meccanica quantistica, essa è descritta da un opportuno \emph{operatore di evoluzione temporale}, che indicheremo con $\op U(t,t_0)$, per il quale $\ket{\psi(t)}=\op U(t,t_0)\ket{\psi(0)}$.
L'operatore $\op U(t,t_0)$ deve rispettare tre principali proprietà:
\begin{itemize}
	\item deve essere unitario, in modo che si conservi la probabilità: se $\adj{\op U(t,t_0)}=U(t,t_0)^{-1}$ allora $\braket{\psi(t)}{\psi(t)}=\bra{\psi(0)}\adj{\op U(t,t_0)}\op U(t,t_0)\ket{\psi(0)}=\braket{\psi(0)}{\psi(0)}$;
	\item deve essere continuo e ridursi all'identità per $t\to t_0$, ossia $U(t_0,t_0)=1$;
	\item si compone con la regola $\op U(t_2,t_1)\op U(t_1,t_0)=\op U(t_2,t_0)$, in modo che se facciamo evolvere il sistema da un tempo $t_0$ a $t_1$ e successivamente da $t_1$ a $t_2$, otteniamo lo stesso risultato che se lo facessimo evolvere direttamente da $t_0$ e $t_2$.
\end{itemize}

Da qui in poi assumiamo che l'hamiltoniano non sia esplicitamente dipendente dal tempo, come sarà il caso in tutti gli esempi che vedremo.
Seguendo l'esempio della meccanica classica, assumiamo anche qui che l'hamiltoniano sia il generatore delle traslazioni temporali infinitesime.
Dato che l'argomento dell'esponenziale deve essere adimensionale, dobbiamo dividere $\op Ht$ per un'opportuna costante: essa è ancora $\hbar$, e l'operatore $\op U(t,t_0)$ è dato dalla formula\footnote{Non mostriamo qui le ragioni per cui la costante dimensionale debba essere proprio la stessa $\hbar$ usata finora.}
\begin{equation}
	\op U(t,t_0)=\exp\bigg[-\frac{i}{\hbar}\op H(t-t_0)\bigg]
	\label{eq:operatore-evoluzione-temporale}
\end{equation}
che mostra come il sistema dipenda in realtà solamente dal tempo trascorso $t-t_0$ (sempre se $\op H$ è indipendente dal tempo) e non dagli specifici istanti di tempo.
Possiamo indicare direttamente con $t$ il tempo trascorso, oppure porre $t_0=0$, ottenendo
\begin{equation}
	\op U(t)=\exp\bigg(-\frac{i}{\hbar}\op Ht\bigg).
\end{equation}
L'operatore di evoluzione temporale commuta dunque con l'hamiltoniano.
Nel caso di una dipendenza esplicita dal tempo dell'hamiltoniano, non esiste in generale una formula per $\op U(t,t_0)$, ed esso non commuta necessariamente con $\op H(t)$.

Per studiare come uno stato varia nel tempo consideriamo la sua derivata temporale all'istante $t$, passando alla derivata di $\op U(t)$, possiamo ottenere l'equazione differenziale
\begin{equation}
	\drp{}{t}\ket{\psi(t)}=\drp{}{t}\op U(t)\ket{\psi(0)}=-\frac{i}{\hbar}\op H\op U(t)\ket{\psi(0)}=-\frac{i}{\hbar}\op H\ket{\psi(t)}\quad\longrightarrow\quad \op H=i\hbar\drp{}{t}.
\end{equation}
Nel caso in cui anche l'hamiltoniano dipenda dal tempo, l'equazione si generalizza nella forma
\begin{equation}
	i\hbar\drp{}{t}\op U(t)=\op H(t)\op U(t)
	\label{eq:schrodinger-tempo}
\end{equation}
che non è altro che la \eqref{eq:schroedinger-dipendente-dal-tempo} applicata all'operatore $U(t)$ anzich\'e allo stato.
Nella rappresentazione delle coordinate troviamo l'equazione differenziale per la funzione d'onda, che è $\bra{q}i\hbar\drp{}{t}\ket{\psi(t)}=\bra{q}\op H(t)\ket{\psi(t)}$, da cui
\begin{equation}
	i\hbar\drp{}{t}\psi(q,t)=H\psi(q,t)
	\label{eq:schrodinger-tempo-coordinate}
\end{equation}
dove $H$ nell'equazione si riferisce all'operatore di $\leb[2]{\R}$ corrispondente all'hamiltoniano.

\section{Stati stazionari}
Esistono degli stati \emph{stazionari}, che non dipendono dal tempo?
Chiedere che $\ket{\psi(t)}$ sia uguale a $\ket{\psi(0)}$ è eccessivo, perch\'e allora $\ket{\psi(0)}$ sarebbe un autostato di $\op U(t)$ con autovalore 1, ossia
\begin{equation}
	\ket{\psi(0)}=e^{-\frac{i}{\hbar}\op Ht}\ket{\psi(0)}=\sum_{k=0}^{+\infty}\frac1{k!}\bigg(-\frac{i}{\hbar}\op Ht\bigg)^k\ket{\psi(0)}=\ket{\psi(0)}+\sum_{k=1}^{+\infty}\frac1{k!}\bigg(-\frac{i}{\hbar}\op Ht\bigg)^k\ket{\psi(0)}
\end{equation}
che vale a dire che $\ket{\psi(0)}$ è un autostato di $\op H$ con autovalore nullo, situazione impossibile per uno stato fisico.
Sappiamo però che due vettori multipli per uno scalare rappresentano lo stesso stato fisico, perciò anche se $\op U(t)\ket{\psi(0)}=c(t)\ket{\psi(0)}$ lo stato fisico rimane invariato, come vogliamo.
Oltre ad apparire nell'espressione dell'operatore $\op U(t)$, l'hamiltoniano gioca un ruolo cruciale nell'evoluzione temporale, come spiegato nel seguente teorema.
\begin{teorema} \label{t:autostati-hamiltoniano-stazionari}
	Uno stato è stazionario se e solo se è un autostato dell'hamiltoniano.
\end{teorema}
\begin{proof}
	Sia $\ket{\psi(0)}$ un autostato di $\op H$: allora $\op H\ket{\psi(0)}=E\ket{\psi(0)}$ per un certo $E\in\R$.
	Sviluppando in serie l'esponenziale, ogni termine ha una potenza di $\op H$ e chiaramente $\op H^k\ket{\psi(0)}=E^k\ket{\psi(0)}$, quindi $\op U(t)\ket{\psi(0)}=\exp\big(-\frac{i}{\hbar}\op Ht\big)\ket{\psi(0)}=\exp\big(-\frac{i}{\hbar}Et\big)\ket{\psi(0)}$.

	Sia ora $\ket{\psi(0)}$ uno stato stazionario, per cui esiste una funzione $c(t)$ tale che $\ket{\psi(t)}=c(t)\ket{\psi(0)}$.
	Allora dall'equazione \eqref{eq:schrodinger-tempo} applicata a $\ket{\psi(0)}$ troviamo
	\begin{equation}
		i\hbar\dot{c}(t)\ket{\psi(0)}=c(t)\op H\ket{\psi(0)}\quad\longrightarrow\quad i\hbar\frac{\dot{c}(t)}{c(t)}\ket{\psi(0)}=\op H\ket{\psi(0)}.
	\end{equation}
	Ora, se $\op H$ non dipende dal tempo, anche il primo membro è una costante.
	Poniamo dunque porre $E\defeq i\hbar\frac{\dot{c}(t)}{c(t)}$ ottenendo l'equazione differenziale $\dot{c}(t)=-\frac{i}{\hbar}Ec(t)$.
	La condizione iniziale è $c(0)=1$, dato che deve valere evidentemente $\ket{\psi(0)}=c(0)\ket{\psi(0)}$.
	Troviamo di conseguenza la soluzione $c(t)=\exp\big(-\frac{i}{\hbar}Et\big)$, vale a dire $\ket{\psi(0)}$ è un autostato di $\op H$ come si può vedere ricalcando il ragionamento della prima parte della dimostrazione.
\end{proof}

Conoscendo lo spettro dell'hamiltoniano e i suoi autostati $\ket{E_n}$, potremmo sviluppare lo stato in questa base scrivendo
\begin{equation}
	\ket{\psi(0)}=\sum_{n=0}^{+\infty}\ket{E_n}\braket{E_n}{\psi(0)}=\sum_{n=0}^{+\infty}a_n\ket{E_n}
\end{equation}
da cui ricaviamo
\begin{equation}
	\ket{\psi(t)}=\op U(t)\sum_{n=0}^{+\infty}a_n\ket{E_n}=\sum_{n=0}^{+\infty}a_n\op U(t)\ket{E_n}=\sum_{n=0}^{+\infty}a_ne^{-\frac{i}{\hbar}E_nt}\ket{E_n},
\end{equation}
quindi i coefficienti dello sviluppo nella base di autostati di $\op H$ evolvono secondo l'equazione $a_n(t)=a_n(0)\exp\big(-\frac{i}{\hbar}E_nt\big)$, poich\'e gli autostati sono stazionari.

\section{Evoluzione delle osservabili}
Finora abbiamo visto gli stati variare nel tempo, mentre le osservabili rimangono costanti: è lo \emph{schema di Schr\"odinger}.
Potremmo invece lasciare fissati gli stati e far variare, secondo le medesime equazioni, le osservabili, seguendo invece lo \emph{schema di Heisenberg}.
Possiamo individuare delle osservabili i cui valori medi rimangono costanti?
Il valore medio, come funzione del tempo, è ancora definito come $\avg{\xi(t)}=\bra{\psi(t)}\op\xi\ket{\psi(t)}$.
Calcolando la sua derivata rispetto al tempo e uguagliandola a zero troviamo, dato che $\bra{\psi(0)}$ e $\ket{\psi(0)}$ sono costanti,
\begin{multline}
	0=\drp{}{t}[\adj{\op U}(t)\op\xi\op U(t)]=\bigg(\drp{}{t}\adj{\op U}(t)\bigg)\op\xi\op U(t)+\adj{\op U}(t)\op\xi\drp{}{t}\op U(t)=\\
	=\frac{i}{\hbar}\big[\adj{\op U}(t)\op H\op\xi\op U(t)-\adj{\op U}(t)\op\xi\op H\op U(t)\big]=\frac{i}{\hbar}\adj{\op U}(t)[\op H,\op\xi]\op U(t)
\end{multline}
ossia $[\op H,\op\xi]=0$.
Dunque $\op\xi$ è una ``costante del moto'' se commuta con $\op H$: in questo caso sono compatibili (se $\op H$ non è degenere) e possiamo scegliere una base di autostati simultanei, che sono quindi autostati di $\op\xi$ stazionari.

Nell'equazione $\bra{\psi(t)}\op\xi\ket{\psi(t)}=\bra{\psi(0)}\adj{\op U}(t)\op\xi\op U(t)\ket{\psi(0)}$ possiamo definire l'operatore variabile nel tempo $\op\xi(t)\defeq\adj{\op U}(t)\op\xi\op U(t)$ ottenendo $\bra{\psi(t)}\op\xi\ket{\psi(t)}=\bra{\psi(0)}\op\xi(t)\ket{\psi(0)}$.
Se $\op\xi$ è hermitiano chiaramente lo è anche $\op\xi(t)$, inoltre vale ancora
\begin{equation}
	\drp{}{t}\op\xi(t)=\frac{i}{\hbar}\adj{\op U}(t)[\op H,\op\xi]\op U(t)=\frac{i}{\hbar}[\op H,\op\xi(t)],
	\label{eq:evoluzione-osservabili-heisenberg}
\end{equation}
detta \emph{equazione di Heisenberg}.
A meno del fattore $i/\hbar$, questa equazione non è altro che la già nota relazione in meccanica classica per l'evoluzione temporale delle osservabili $\dot{f}=\{f,H\}$.

Ad esempio, prendiamo l'hamiltoniano $\op H=\frac1{2m}\op p^2+V(\op q)$ di una particella soggetta, in una dimensione, ad un generico potenziale $V(q)$.
L'evoluzione della posizione è
\begin{equation}
	\drp{}{t}\op q(t)=\frac{i}{\hbar}[\op H,\op q]=\frac{i}{2\hbar m}[\op p^2,\op q]=\frac{i\op p}{\hbar m}[\op p,\op q]=\frac{\op p}{m},
\end{equation}
mentre quella degli impulsi è
\begin{equation}
	\drp{}{t}\op p(t)=\frac{i}{\hbar}[\op H,\op p]=\frac{i}{\hbar}[V(\op q),\op p]=-\drp{V}{q}(\op q).
\end{equation}
Notare la forte somiglianza di queste due equazioni con quelle di Hamilton in meccanica classica.

\section{Corrente di probabilità}
Dato che l'equazione di Schr\"odinger non dipendente dal tempo è a coefficienti reali, se $\psi$ è una sua soluzione anche la coniugata $\psi^*$ lo è, ossia $\psi$ e $\psi^*$ soddisfano le equazioni
\begin{equation}
	-\frac{\hbar^2}{2m}\lap\psi(\vec x)+\big[V(\vec x)-E\big]\psi(\vec x)=0\qeq	-\frac{\hbar^2}{2m}\lap\psi^*(\vec x)+\big[V(\vec x)-E\big]\psi^*(\vec x)=0.
\end{equation}
Moltiplicando (a sinistra) la prima per $\psi^*$, la seconda per $\psi$ e sottraendole otteniamo allora
\begin{equation}
	-\frac{\hbar^2}{2m}\big[\psi^*(\vec x)\lap\psi(\vec x)-\psi(\vec x)\lap\psi^*(\vec x)\big]=0
\end{equation}
da cui
\begin{equation}
	0=-\frac{\hbar^2}{2m}\psi^*(\vec x)\lap\psi(\vec x)-\psi(\vec x)\lap\psi^*(\vec x)=-\frac{\hbar^2}{2m}\div\big(\psi^*(\vec x)\grad\psi(\vec x)-\psi(\vec x)\grad\psi^*(\vec x)\big).
\end{equation}
Definiamo dunque la \emph{corrente di probabilità} $\vec J(\vec x)$ come la quantità
\begin{equation}
	\vec J(\vec x)=-\frac{i\hbar}{2m}\big(\psi^*(\vec x)\grad\psi(\vec x)-\psi(\vec x)\grad\psi^*(\vec x)\big)
	\label{eq:corrente-probabilita}
\end{equation}
che in questo caso soddisfa l'equazione di continuità $\div\vec J(\vec x)=0$.

La corrente di probabilità $\vec J$ si definisce allo stesso modo anche per sistemi variabili nel tempo, secondo l'equazione \eqref{eq:schrodinger-tempo-coordinate}, con le $\psi$ e $\psi^*$ ovviamente a loro volta dipendenti dal tempo.
In tal caso si dimostra che la corrente di probablità soddisfa l'equazione
\begin{equation}
	\div\vec J(\vec x,t)+\drp{\rho}{t}(\vec x,t)=0,
	\label{eq:continuita-corrente-probabilita}
\end{equation}
dove $\rho(\vec x,t)=\abs{\psi(\vec x,t)}^2$, che è la versione più generale dell'equazione di continuità precedente.
Se in particolare $\psi(\vec x,t)$ è un'autofunzione dell'hamiltoniano $\psi_n(\vec x)e^{-iE_nt/\hbar}$, allora $\vec J(\vec x,t)=0$ e la densità di probabilità $\rho$ si conserva nel tempo, come abbiamo già visto.

\section{Teorema di Ehrenfest}
Nel caso classico abbiamo il teorema del viriale che lega i valori medi nel tempo dell'energia cinetica e potenziale.
Ai fini della dimostrazione è necessario il seguente risultato.
\begin{teorema}[Eulero] \label{t:eulero-funzioni-omogenee}
	Se $f\colon\R^n\to\R$ è una funzione omogenea sufficientemente liscia di grado $k$, ossia $f(\lambda\vec x)=\lambda^kf(\vec x)$, allora
	\begin{equation}
		\sum_{i=1}^nx_i\drp{V}{x_i}(\vec x)=kV(\vec x).
		\label{eq:eulero-funzioni-omogenee}
	\end{equation}
\end{teorema}
\begin{proof}
	Derivando l'identità $f(\lambda\vec x)=\lambda^kf(\vec x)$ rispetto alla componente $i$-esima troviamo $\lambda\drp{f}{x_i}(\lambda\vec x)=\lambda^k\drp{f}{x_i}(\vec x)$ ossia
	\begin{equation}
		\drp{f}{x_i}(\lambda\vec x)\lambda^{k-1}f(\vec x).
	\end{equation}
	Sommando su tutte le componenti
	\begin{equation}
		\sum_{i=1}^nx_i\drp{f}{x_i}(\lambda\vec x)=k\lambda^{k-1}f(\vec x)
	\end{equation}
	e basta prendere $\lambda=1$ per ottenere la \eqref{eq:eulero-funzioni-omogenee}.
\end{proof}

\begin{teorema}[del viriale] \label{t:viriale}
	Se il potenziale $V$ di un sistema è una funzione omogenea di grado $k$, allora $k\avg{V}=2\avg{T}$.
\end{teorema}
\begin{proof}
	Prendiamo la parentesi di Poisson $\pois{\sum_iq_ip_i}{\ham}$: risulta
	\begin{equation}
		\sum_i\pois{q_ip_i}{\ham}=\sum_i\big(q_i\pois{p_i}{\ham}+\pois{q_i}{\ham}p_i\big)=\sum_i\bigg(-q_i\drp{\ham}{q_i}+p_i\drp{\ham}{p_i}\bigg)=-kV+2T
	\end{equation}
	sfruttando le equazioni canoniche di Hamilton e il fatto che $T$ è omogenea di grado $2$ mentre $V$ lo è di grado $k$.
	Integrando su un periodo risulta
	\begin{equation}
		0=\frac1{T}\int_0^T\drv{}{t}\bigg(\sum_iq_ip_i\bigg)\,\dd t=\frac1{T}\int_0^T\pois{\sum_iq_ip_i}{\ham}\,\dd t=-k\avg{V}+2\avg{T}
	\end{equation}
	da cui $k\avg{V}=2\avg{T}$.
\end{proof}

Nei sistemi quantistici, l'analogo del teorema del viriale è dato dal teorema di Ehrenfest.
Anche la dimostrazione è molto simile, ma è svolta considerando soltanto gli autostati dell'hamiltoniano.
\begin{teorema}[Ehrenfest] \label{t:ehrenfest}
	Sia $\ket{E}$ un autostato dell'hamiltoniano, tale che $\op H\ket{E}=E\ket{E}$.
	Se il potenziale $V$ è una funzione di $\op q$ omogenea di grado $k$ allora
	\begin{equation}
		k\bra{E}V(\op q)\ket{E}=2\bra{E}T(\op p)\ket{E}
		\label{eq:ehrenfest}
	\end{equation}
	dove $T$ è l'energia cinetica, come funzione di $\op p$ omogenea di grado 2.
\end{teorema}
\begin{proof}
	Abbiamo già visto che per un qualsiasi operatore $\op\xi$ si ha $\ket{E}[\op\xi,\op H]\ket{E}=0$.
	Ponendo $\op\xi=\sum_i\op q_i\op p_i$ allora risulta
	\begin{equation}
		0=\sum_i\bra{E}[\op q_i\op p_i,\op H]\ket{E}=\sum_i\bra{E}\op q_i[\op p_i,\op H]+[\op q_i,\op H]\op p_i\ket{E}=\bra{E}-kV(\op q)+2T(\op p)\ket{E}
	\end{equation}
	da cui la tesi.
\end{proof}

