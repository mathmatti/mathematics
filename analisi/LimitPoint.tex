\documentclass[a4paper,10pt]{article}
\usepackage[utf8]{inputenc}
\usepackage{amsmath}
\usepackage{hyperref}

%opening
\title{Limit point (Punto di accumulazione)}
\author{\href{http://www.baudo.hol.es}{giuseppe baudo}}

\begin{document}

\maketitle

\section{DEFINIZIONE (1)}
Let $X$ be a metric space. All points and sets mentioned below are understood to be elements and subsets of $X$.

A point $p$ is a limit point of the set $E$ if every neighborhood of $p$ contains a point $q \ne p$ such that $q \in E$.

\section{DEFINIZIONE (2)}
Let $S \subseteq R$, and let $x \in R$.

1. $x$ is a limit point or an accumulation point or a cluster point of $S$ if $\forall \delta > 0$, $(x-\delta,x+\delta) \cap S \ \ne \o{}$

\section{NOTAZIONE}

\section{ESEMPIO}

\section{APPROFONDIMENTI}
\begin{itemize}
 \item TITLE
\end{itemize}

\end{document}
