\documentclass[a4paper,10pt]{article}
\usepackage[utf8]{inputenc}
\usepackage{amsmath}
\usepackage{hyperref}

%opening
\title{Analisi matematica}
\author{}

\begin{document}

\maketitle


\section{Introduzione}

\section{Syllabus}
\begin{itemize}
 \item Principio di induzione
 \item Teoria degli insiemi, funzioni, applicazione, prodotto cartesiano.
 \item Topologia
 \item Intorno
 \item Punto di accumulazione
 \item \href{InsiemeCompatto.html}{Insieme compatto}
 \item \href{Function.html}{Funzione}
 \item (Funzioni) Trigonometria
 \item \href{Arcsine.html}{Arcsine}
 \item Valore assoluto, esponenziali, logaritmi, radici, equazioni e disequazioni
 \item \href{./ValoreAssoluto.html}{Valore assoluto}
 \item \href{./FunzioneEsponenziale.html}{Funzione esponenziale}
 \item Dominio di una funzione
 \item Maggiorante di una funzione
 \item Minorante di una funzione
 \item Estremo superiore di una funzione
 \item \href{FunzioneMassimoMinimo.html}{Massimo e minimo di una funzione}
 \item \href{FunzioneContinua.html}{Funzione continua}
 \item \href{Limite.html}{Limite}
 \item Successione
 \item \href{Derivata.html}{Derivata} 
 \item Derivata
 \item Integrale
\end{itemize}

\section{Theorems}
\begin{itemize}
 \item Teorema di Weierstrass. Una funzione continua in un insieme $E$ compatto ha massimo e minimo.
\end{itemize}


\section{Temi d'esame}
\begin{itemize}
 \item \href{http://www.math.unipd.it/~marson/didattica/Analisi1/temiAnalisi1.html}{http://www.math.unipd.it/~marson/didattica/Analisi1/temiAnalisi1.html}
 \item \href{http://www.math.unipd.it/~colombo/didattica/analisi1/}{http://www.math.unipd.it/~colombo/didattica/analisi1/}
 \item \href{http://www.uniba.it/docenti/mininni-michele/attivita-didattica/tracce/istituzioni-di-analisi-matematica-analisi-mat.-1}{http://www.uniba.it/docenti/mininni-michele/attivita-didattica/tracce/istituzioni-di-analisi-matematica-analisi-mat.-1}
 \item \href{http://paola-gervasio.unibs.it/Appelli_AM1/appelli.html}{http://paola-gervasio.unibs.it/Appelli_AM1/appelli.html}
 \item \href{http://calvino.polito.it/~terzafac/Corsi/analisi1/materiale.html}{http://calvino.polito.it/~terzafac/Corsi/analisi1/materiale.html}
 \item \href{http://calvino.polito.it/~lancelotti/didattica/analisi1_new/analisi1_new_temi.html}{http://calvino.polito.it/~lancelotti/didattica/analisi1_new/analisi1_new_temi.html}
 \item \href{http://www.dmi.units.it/~omari/Analisi_matematica_1_(2010-11)/Esercizi/Anex1.pdf}{http://www.dmi.units.it/~omari/Analisi_matematica_1_(2010-11)/Esercizi/Anex1.pdf}
 \item \href{http://users.dma.unipi.it/gobbino/Home_Page/Files/HP_AD/E99_CS.pdf}{http://users.dma.unipi.it/gobbino/Home_Page/Files/HP_AD/E99_CS.pdf}
 \item \href{http://www.dima.unige.it/~demari/Eser.pdf}{http://www.dima.unige.it/~demari/Eser.pdf}
\end{itemize}

\end{document}
