\documentclass[a4paper,10pt]{article}
\usepackage[utf8]{inputenc}
\usepackage{amsmath}
\usepackage{hyperref}

%opening
\title{Tangent}
\author{\href{http://www.baudo.hol.es}{giuseppe baudo}}

\begin{document}

\maketitle

\section{DEFINITION (1)}
La tangente di una funzione in un punto $x_0$ E' la derivata prima della funzione calcolata in quel punto.

\section{DEFINITION (2)}
E' Il rapporto tra seno e coseno

\section{DEFINITION (3)}
E' il rapporto tra ordinata e raggio goniometrico???

\section{DEFINITION (4)}
Tangent function represents the ratio of sides of the right angle triangle. It is also known as circular
function, since their values can be the ratios fo $x$ and $y$ coordinates on a circle of radius $1$.
\section{NOTAZIONE}

\section{ESEMPIO}

\section{LETTURE CONSIGLIATE}
\begin{itemize}
 \item \url{http://www.youmath.it/lezioni/analisi-matematica/le-funzioni-elementari-e-le-loro-proprieta/150-tangente-e-cotangente-di-un-angolo.html}
\end{itemize}

\section{APPROFONDIMENTI}
  \begin{itemize}
   \item \url{https://en.wikipedia.org/wiki/Tangent}
  \end{itemize}


\end{document}
