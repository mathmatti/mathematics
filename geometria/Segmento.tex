\documentclass[a4paper,10pt]{article}
\usepackage[utf8]{inputenc}
\usepackage{amsmath}
\usepackage{hyperref}

%opening
\title{Segmento}
\author{\href{http://www.baudo.hol.es}{giuseppe baudo}}

\begin{document}

\maketitle

\section{DEFINIZIONE}
Un segmento è un vettore applicato in un punto A e terminante in un punto B.

\section{NOTE}
Il concetto di segmento è la corrispondenza 1 a 1 tra lo studio della geometria con gli strumenti tradizionali ed il concetto di vettore.


\section{ESEMPIO}

\section{APPROFONDIMENTI}
\begin{itemize}
 \item \url{http://calvino.polito.it/~casnati/Geometria15BCG/GeometriaNuovo/GeometriaNuovo7.pdf}
\end{itemize}

\end{document}
