\documentclass[a4paper,10pt]{article}
\usepackage[utf8]{inputenc}
\usepackage{amsmath}
\usepackage{hyperref}

%opening
\title{Equazione cartesiana del piano}
\author{\href{http://www.baudo.hol.es}{giuseppe baudo}}

\begin{document}

\maketitle

\section{ENUNCIATO}
Ogni equazione lineare in $x,y e z$ del tipo $ax+by+cz+d=0$ rappresenta, a meno di un fattore moltiplicativo non nullo,
l'equazione cartesiana di un piano nello spazio $S_3$ (chi essere $S_3$???).

\section{DIMOSTRAZIONE}

\section{NOTE}
In realtà qui la questione riguarda un teorema il cui risultato è utilizzatissimo nelle applicazioni pratiche.

Si dimostrerà che ogni equazione di primo grado in $x$, $y$ e $z$ del tipo:
\[
 ax+by+cz+d=0
\]

con $a,b,c,d \in R$ e $a,b,c$ non contemporaneamente tutti uguali a zero, $(a,b,c) \ne (0,0,0)$ rappresenta un piano. 
Viceversa, ogni piano dello spazio è rappresentabile tramite un'equazione lineare in $x,y,z$ del tipo suddetto.


\section{ESEMPIO}

\section{APPROFONDIMENTI}
\begin{itemize}
 \item \url{http://progettomatematica.dm.unibo.it/GeomSpazio3/Sito/Pagine/indiceFRAME.html}
 \item \url{http://calvino.polito.it/~salamon/P/G/alga11.pdf}
 \item \url{http://calvino.polito.it/~casnati/Geometria05BCG/Geometria/Geometria9.pdf}
\end{itemize}

\end{document}
