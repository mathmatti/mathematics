\documentclass[a4paper,10pt]{article}
\usepackage[utf8]{inputenc}
\usepackage{amsmath}
\usepackage{hyperref}

%opening
\title{Polinomio caratteristico di una matrice}
\author{\href{http://www.baudo.hol.es}{giuseppe baudo}}

\begin{document}

\maketitle

\section{DEFINIZIONE}
Sia A una matrice quadrata, t una incognita (la nostra x) e $I_n$ la matrice identità di ordine n. Il polinomio caratteristico (nella incognita $t$) di una matrice è uguale al determinante della matrice:
\[
 A-tI_n
\]

scriviamo:
\[
 p_A(t) := det(A-tI_n)
\]

\section{NOTAZIONE}

\section{ESEMPIO}

\section{APPROFONDIMENTI}
\begin{itemize}
 \item \url{http://www.dm.unibo.it/~ida/NoteGeometria1-25-5-16.pdf}
\end{itemize}

\end{document}
