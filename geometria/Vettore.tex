\documentclass[a4paper,10pt]{article}
\usepackage[utf8]{inputenc}
\usepackage{amsmath}
\usepackage{hyperref}

%opening
\title{Vettore}
\author{\href{http://www.baudo.hol.es}{giuseppe baudo}}

\begin{document}

\maketitle

\section{DEFINIZIONE}
\subsection{Vettore colonna, vettore riga}
Un vettore in uno spazio $n$-dimensionale è un insieme ordinato formato da $n$ valori. 

\subsection{Vettore in geometria mono, bi e tri-dimensionale}
Un vettore è un oggetto che ha una direzione e una lunghezza. In questo caso si dimostrerà che un vettore può essere
rappresentato come da definizione precedente.

\section{Componenti di un vettore}

\section{Rappresentazione canonica}

\section{Lunghezza di un vettore in $R^n$}
The length of a vector $v$ in $R^n$ is the square root of the sum of the squares of tis components.
\[
 |v|=\sqrt{v^2_1+...+v^2_n}
\]

This is a natural generalization of the Pythagorean Theorem.

\section{dot product or scalar product}
The dot product (or inner product or scalar product) of two $n$-component real vectors is the linear combination of their components.
\[
 u \dotfill v = u_1v_1+...+u_nv_n
\]
squares of its components.


\section{NOTAZIONE}

\section{NOTE}
Le due definizioni sono equivalenti nel senso che si possono rappresentare i vettori della definizione 2 come vettori
della definizione 1.

Attenzione alla definizione di vettore libero.

Attenzione all'uguaglianza tra due vettori. Due vettori sono uguali quando hanno la stessa rappresentazione canonica.

\section{APPROFONDIMENTI}
\begin{itemize}
 \item \url{http://joshua.smcvt.edu/linearalgebra/book.pdf}
 \item \url{https://www.math10.com/en/geometry/vectors-operations/vectors-operations.html}
 \item \url{http://www.math.utah.edu/online/2210/notes/ch13.pdf}
 \item \url{http://www.ncert.nic.in/ncerts/l/lemh204.pdf}
\end{itemize}

\end{document}
