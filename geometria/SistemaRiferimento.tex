\documentclass[a4paper,10pt]{article}
\usepackage[utf8]{inputenc}
\usepackage{amsmath}
\usepackage{hyperref}

%opening
\title{Sistema di riferimento}
\author{\href{http://www.baudo.hol.es}{giuseppe baudo}}

\begin{document}

\maketitle

\section{DEFINIZIONE}

\subsection{wikipedia italiano}
Si definisce sistema di riferimento l'insieme dei riferimenti o coordinate utilizzate per individuare la posizione di un oggetto
nello spazio. A seconda del numero di riferimenti usati, si può parlare di: sistema di riferimento monodimensionale, bidimensionale, tridimensionale.

\subsection{from wikipedia in english}
In geometry, a coordinate system is a system which uses one or more numbers, or coordinates, to uniquely determine the position of
a point or other geometric element on a manifold such as Euclidean space...


\section{NOTE}
Come dice il nome stesso, dobbiamo fissare un "punto" dal quale osservare i nostri "oggetti". Questo concetto deriva più dalla fisica
che dalla matematica. Vi sono tanti esempi che posso chiarificare. In qualche modo sistema di riferimento si potrebbe anche chiamare
punto di osservazione.

Da notare che per la definizione italiana, sistema di riferimento e sistema di coordinate sono la stessa cosa, sono sinonimi. Mentre
se cerchiamo reference system in inglese ci troviamo qualcosa come "frame of reference" che tratta argomenti di fisica. Quindi nel mondo
anglosassone il nostro sistema di riferimento prende il nome di coordinate system. Peccato però che le due definizioni sono diverse
non solo nella forma ma anche nella sostanza. La definizione inglese, che secondo me è da preferire, mette bene in risalto l'obiettivo
del "sistema delle coordinate" ovvero quello di rappresentare tramite dei numeri o altre strutture matematiche più complesse (per esempio vettori di $R^n$) la 
posizione degli elementi geometrici (punti, rette, solidi, etc).

\section{ESEMPIO}
The simplest example of a coordinate system is the identification of points on a line with real numbers using the number line.
In this sytem, an arbitrary piont O (the origin) is chosen on a given line. The coordinate of a point P is defined as the signed
distance from O to P, where the signed distance is the distance taken as positive or negative depending on which side of the
line P lies. Each point is given a unique coordinate and each real numebr is the coordinate of a unique point.

\section{APPROFONDIMENTI}

\begin{itemize}
 \item \url{https://it.wikipedia.org/wiki/Sistema_di_riferimento}
 \item \url{https://en.wikipedia.org/wiki/Coordinate_system}
\end{itemize}


\end{document}
